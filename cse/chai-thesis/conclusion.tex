\chapter{Conclusion and Future Work}
\chai \ provides an environment to experiment with interface
modules and game semantics. A developer can make use of the algorithms
present in \chai \ to develop new verification algorithms or to use
the existing algorithms to verify the designs. For instance,
\chai\ provides a host of BDD based algorithms to find the
predecessor and post regions on a set of states and compute a set
of reachable states for error detection. This facility is
currently being used for the development of error detection algorithms
in DVLAB \cite{dvlab}.

The streamlined path for using hardware description languages
within \chai\ makes a plethora of designs coded in high level
languages like Verilog available for verification research with
{\chai}. At DVLAB \cite{dvlab} several publicly available hardware
designs coded in Verilog are used as test suites for development
of early error-detection algorithms. Thus, {\chai} accelerates and
promotes the application of verification research to 
industry designs.

We are working towards a comprehensive manual for \chai\ and
releasing a stable version. 
%% Future work would essentially comprise
%% of supporting {\chai}, maintaining {\mvrm}, streamlining the input
%% language and integrating proven compositional verification
%% research and methodologies back into the \chai\ code-base.


