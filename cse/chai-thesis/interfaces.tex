\chapter{Interfaces}
\section{Interface Modules}
No component is designed in isolation: a component must interact
with an {\em environment\/} (either the user, or other design
components), and produce some useful output or behavior. {\im }
provide a way of modelling both aspects of component development:
the input (or design) assumptions, and the output guarantees (or
output behavior).

%How interfaces are better
% I cut this - as I told you, it doesn't really make sense. Luca
%% In traditional formalisms for component based designs, the
%% designer has to specify the behavior of the system for all
%% possible inputs even if he knows some input values in the given
%% environments are infeasible \cite{cav-paper}. Interfaces provide a
%% way of describing the input assumptions of a component by
%% expressing them as input relations.
%% Consider an environment where more than one component can generate
%% the same output. In such a setting passive modelling is
%% in-applicable. A partial solution is provided by formalisms like
%% process algebras \cite{cav-paper} where a co-operative solution is
%% achieved by making components agree on the output action. Such a
%% co-operation is input based. However there is no-way to express
%% the environment assumptions of a this design other than expressing
%% it hard-coded in the components. Interfaces present an elegant
%% solution by allowing the designer to express such assumptions.

To give a taste of interface modules,
Figure~\ref{fig-counter} illustrates a simple synchronous
interface of a 8-bit $\pm 1$ adder controlled by a binary counter.
The interface formalism is described as an interface module with
state variables partitioned as inputs and outputs. The acceptable
state variable changes are described by transition relations of
input atoms, and possible output state variable changes are
described as transition relations of output atoms. We will
consider parts of this example in coming sections to explain
various aspects of interfaces.

%A gate-counter example
%\begin{comment}
\begin{figure}
\centering
\input{egs/counter.intf}
\hspace{3em}
\input{egs/adder.intf}
\caption{A counter and a $\pm 1$ adder, modelled as a synchronous
interface.} \label{fig-counter}
\end{figure}
%\end{comment}

%Background definitions
\section{Basics}
This section is modelled after \cite{daLect}. In order to formally
define \im \ we present a few relevant terms.

%We present here a few relevant definitions before we formally
%define interface modules.

\paragraph{Variables.}
Consider an infinite global set $\allvars$ of {\em typed
variables,} from which the variables of the modules will be drawn.
Each variable $x \in \allvars$ has an associated {\em domain,} or
set of possible values, which we denote $\domain(x)$. We treat
$\domain(x)$ as a finite set, restricting our attention to
finite-state systems.

\paragraph{States.}
Given a finite set $\vars\subs\allvars$ of variables, a {\em
state\/} $s$ over $\vars$ is a function that associates with each
variable $x \in \vars$ a value $s(x) \in \domain(x)$; we denote by
$\states[\vars]$ the set of all possible states over the variables
$\vars$. Note that, formally, the type of $s \in \states[\vars]$
is $\prod_{x\in\vars} (x \mapsto \domain(x))$. Given a state $s
\in \states[\vars]$ and a subset $\avars \subs \vars$ of
variables, we denote by $s[\avars] \in \states[\avars]$ the
restriction of $s$ to the variables in $\avars$: precisely,
$s[\avars]$ is defined by $s[\avars](x) = s(x)$ for all $x \in
\avars$. For any two sets $\vars,\avars$ of variables, and states
$s \in \states[\vars]$ and $t \in \states[\avars]$, we write $s
\simeq t$ if $s(x) = t(x)$ for all shared variables $x \in \vars
\inters \avars$.

\paragraph{State predicates.}
We assume a logical language $\loglang$ in which assertions about
the values of the variables in $\allvars$ can be written. For
example, if all variables are boolean, then $\loglang$ can be
taken to be predicate logic with the addition of the quantifiers
$\forall$ and $\exists$ over the booleans. We say that a formula
$\phi\in\loglang$ is {\em over\/} a set $\vars$ of variables if it
only involves variables of $\vars$; such a formula is also called
a {\em predicate\/} over $\vars$. We denote by $\Preds[\vars]$ the
set of all formulas over the set of variable $\vars$. Given a
formula $\phi$ over $\vars$ and a state $s \in \states[\vars]$, we
write $s \sat \phi$ to denote the fact that $\phi$ is true under
the interpretation that assigns to every variable $x \in \vars$
the value $s(x)$. In particular, a formula $\phi$ over $\vars$
defines the set of states $\semb{\phi}_\vars = \set{s \in
\states[\vars] \mid s \sat \phi}$.

\begin{examp}{}
Consider the set of boolean variables $\vars = \set{x,y,z}$. The
set $\states[\vars]$ consists of $2^3 = 8$ elements. If we take
$\loglang$ to be propositional logic, then the formula $x \und \no
y$ is satisfied by the two states $(x=\true,y=\false,z=\false),
(x=\true,y=\false,z=\true) \in \states[\vars]$. If we take
$\loglang$ to be quantified boolean formulas, then the formula $
  \exists w \qdot (w \equiv x \und w \equiv \no y \und w \equiv z)
$ is satisfied by the two states $(x=\true,y=\false,z=\true),
(x=\false,y=\true,z=\false) \in \states[\vars]$. \qed
\end{examp}

\paragraph{Transition predicates.}
In order to be able to define {\em relations,} in addition to sets
of states, we introduce the following notation used widely in
model checking research community. For each state variable $x$, we
introduce a new variable $\nx{x}$ (read: ``next $x$''), with
$\domain(x) = \domain(\nx{x})$, that denotes the value of the
state variable $x$ in the successor state. Given a set $\vars
\subs \allvars$ of variables, we let $\nx{\vars} = \set{\nx{x}
\mid x \in \vars}$ be the corresponding set of next variables. We
denote the converse of $\nx{}$ by $\pr{}$ (read: ``previous''):
precisely, we let $\pr{\nx{x}} = x$ for all variables $x$. Given a
predicate $\phi$, we denote by $\nx{\phi}$ the result of replacing
every variable $x$ in $\phi$ with $\nx{x}$, and by $\pr{\phi}$ the
result of replacing every $\nx{x}$ in $\phi$ with $x$; thus,
$\pr{\nx{\phi}} = \phi$.

In a transition predicate the standard variables refer to the
current state, and the next variables refer to the successor
state. Given a predicate $\rho$ over $\vars \union \nx{\avars}$,
and states $s \in \states[\vars]$ and $t \in \states[\avars]$, we
write $(s,t) \sat \rho$ to denote the fact that $\rho$ is true
when every $x \in \vars$ has value $s(x)$, and every $\nx{y} \in
\nx{\avars}$ has value $t(y)$. A transition predicate $\rho \in
\Preds[\vars,\nx \avars]$ defines a relation
\[
  \semb{\rho}_{\vars,\nx{\avars}}
  = \set{(s,t) \in \states[\vars] \times \states[\avars] \mid
         (s,t) \sat \rho}.
\]

\begin{examp}{}
Consider the set of boolean variables $\vars = \set{x,y}$. The
transition predicate $(\nx{x} \equiv y) \und \no \nx{y}$ defines
the transition that copies the value of $y$ into $x$, and sets $y$
to $\false$. \qed
\end{examp}

%TODO: Formally - Formal definition of moore interfaces
\section{Synchronous Interfaces, Formally}
Interfaces as implemented in \chai \ do not depend on next values
of inputs, and hence are termed Synchronous Interfaces. We
formally define them as follows.

\begin{defi}{(Interface Modules)}
An {\em Interface Module\/} $\mp = \tuple{\ivars_\mp, \ovars_\mp,
\rvars_\mp, \iinit_\mp, \oinit_\mp, \itrans_\mp, \otrans_\mp}$
consists of the following elements:

\begin{itemize}

\item A set $\ivars_\mp$ of {\em input variables,} and a set
$\ovars_\mp$ of {\em output variables.} The two sets must be
disjoint: $\ivars_\mp \inters \ovars_\mp = \emptyset$. We indicate
by $\vars_\mp = \ivars_\mp \union \ovars_\mp$ the set of all state
variables of $\mp$.

\item A set $\rvars_\mp$ of {\em reserved variables,} such that
$\ivars_\mp \union \ovars_\mp \subs \rvars_\mp$. The set
$\rvars_\mp$ contains variables that are reserved for use by the
module, and constitute the module {\em name space.}

\item A predicate $\iinit_\mp \in \Preds[\ivars_\mp]$ defining the
legal initial values for the input variables.

\item A predicate $\oinit_\mp \in \Preds[\ovars_\mp]$ defining the
initial values of the output variables.

\item An {\em input transition predicate\/} $\itrans_\mp \in
\Preds[\vars_\mp \union \nx{\ivars_\mp}]$, such that for all $s
\in \states[\vars_\mp]$, there is some $t \in \states[\ivars_\mp]$
such that $(s,t) \sat \itrans_\mp$. The predicate $\itrans_\mp$
specifies what are the legal value changes for the input
variables.

\item An {\em output transition predicate\/} $\otrans_\mp \in
\Preds[\vars_\mp \union \nx{\ovars_\mp}]$, such that for all $s
\in \states[\vars_\mp]$, there is some $t \in \states[\ovars_\mp]$
such that $(s,t) \sat \otrans_\mp$. The predicate $\otrans_\mp$
specifies how the module can update the values of the output
variables. \qed
\end{itemize}
\end{defi}

\noindent Thus, associated with a module is a set of initial
states, a transition relation, and a language. The set of initial
states consists of the states that correspond to both possible
initial values for the output variables, and legal initial values
for the input variables. The transition relation consists of the
state transitions that are both possible for the output variables,
and legal for the input variables. The language of a module
consists of all the possible infinite sequences of states that
satisfy the initial conditions and the transition relations.

%Definition of Trace.

\begin{defi}{(Set of [initial] states, transition relation, trace, and language)}
Consider a module $\mp = \tuple{\ivars_\mp, \ovars_\mp,
\rvars_\mp, \iinit_\mp, \oinit_\mp, \itrans_\mp, \otrans_\mp}$.
%
\begin{itemize}

\item The set of {\em states\/} of $\mp$ is $S_\mp =
\states[\vars_\mp]$.

\item The set of {\em initial states\/} of $\mp$ is $\sinit_\mp =
\set{s \in S_\mp \mid s \sat \iinit_\mp \und \oinit_\mp}$.

\item The {\em transition relation\/} of $\mp$ is $\tr_\mp =
\set{(s,t) \in S_\mp \times S_\mp \mid (s,t) \sat \itrans_\mp \und
\otrans_\mp}$.

\item A {\em path\/} of $\mp$ from $s \in S_\mp$ is an infinite
sequence $s=s_0, s_1, s_2, \ldots$ of $S_\mp$ such that
$(s_k,s_{k+1}) \in \tr_\mp$ for all $k > 0$.

\item A {\em trace\/} of $\mp$ is a path $s_0, s_1, s_2, \ldots$
such that $s_0 \in \sinit_\mp$.

\item The {\em language\/} of $\mp$ is the set $\lang{(\mp)}$
consisting of all traces of $\mp$. \qed
\end{itemize}
\end{defi}

The requirement on $\itrans_\mp$ and $\otrans_\mp$ ensures that
every state in $S_\mp$ has a successor that satisfies both the
input and output transition relations, ensuring that from every
state, there is a transition that is both possible for the module,
and legal for the environment. Note that if $\iinit_\mp \und
\oinit_\mp$ is unsatisfiable, then $\lang{(\mp)} = \emptyset$.

Synchronous interface modules as defined above are an example of
{\em Moore\/} modules, in which the next value of the output and
internal variables can depend on the current state, but not on the
next value of the input variables. For instance, if the state
variables of a module $\mp$ are $x$ (input) and $y$ (output), then
in a transition from $s \in S_\mp$ to $t \in S_\mp$ the next value
$t(y)$ of $y$ can depend on the old values $s(x)$ and $s(y)$, but
not on the new value $t(x)$ of the input variable.

\section{Composition}
Two interface modules are compatible if there is an environment in
which they can work together. Compatible interfaces on composing
lead to a new satisfiable input assumption that ensures that no
local error state is reachable.

Consider the example in Figure \ref{fig-counter}. The $\adder$ has
two control inputs $\iadd$ and $\isub$, data inputs $\id_7
\cdots\id_0$; and data outputs $\od_7 \cdots \od_0$.
\begin{itemize}

\item When $\iadd = \isub = 1$, the adder leaves the input
unchanged: the next value of $\od_7 \cdots \od_0$ is equal to
$\id_7 \cdots\id_0$.

\item When $\iadd = 0$ and $\isub = 1$, the next outputs are given
by $[\nx{o}_7 \cdots \nx{o}_0] = [\id_7 \cdots\id_0] + 1 \mod
2^8$, and $[\nx{o}_7 \cdots \nx{o}_0]$ is the integer encoded in
binary by $\nx{o}_7 \cdots \nx{o}_0$.

\item When $\isub = 0$ and $\iadd = 1$, we have  $[\nx{o}_7 \cdots
\nx{o}_0] = [\id_7 \cdots\id_0] - 1 \mod 2^8$.

\item The adder is designed with the assumption that $\isub$ and
$\iadd$ are not both~$0$: hence, the input transition relation of
$\adder$ states that $\nx{q}_1 \nx{q}_0 \neq 00$.
\end{itemize}

In order to cycle between adding $0, +1, -1$, the control inputs
$\iadd$ and $\isub$ are connected to the outputs $\oq_1$ and
$\oq_0$ of a two-bit count-to-zero counter $\counter$. The counter
has only one input, $\itoone$: when $\itoone = 0$, then $\nx{q}_1
\nx{q}_0 = 11$; otherwise, $[\nx{q}_1 \nx{q}_0] = [\oq_1 \oq_0] -
1 \mod 4$.

When the counter is connected to the adder, the joint system can
take a transition to a state where $\oq_1 \oq_0 = 00$, violating
the adder's input assumptions. In spite of this, the counter and
the adder are compatible, since there is a way to use them
together: to avoid the incompatible transition, it suffices to
assert $\itoone = 0$ early enough in the count-to-zero cycle of
the counter. To reflect this, when we compose $\counter$ and
$\adder$, we synthesize for their composition $\counter \| \adder$
a new input assumption, that ensures that the input assumptions of
both $\counter$ and $\adder$ are satisfied.

To determine the new input assumption, we \textbf{solve a game}
between Input, which chooses the next values of $\itoone$ and
$\id_7\cdots\id_0$, and Output, which chooses the next values of
$\oq_0,\oq_1$, and $\od_7 \cdots \od_0$. The goal of Input is to
avoid a transition to $\oq_1 \oq_0 = 00$. At the states where
$\oq_1 \oq_0 = 01$, Input can win if $\itoone = 0$, since
% at the next clock cycle
we will have $\nx{q}_1 \nx{q}_0 = 11$; but Input cannot win if
$\itoone = 1$. By choosing $\nx{\itoone} = 0$, Input can also win
from the states where $\oq_1 \oq_0 = 10$. Finally, Input can
always win from
% the states where
$\oq_1 \oq_0 = 11$, for all $\nx{\itoone}$. Thus, we associate
with $\counter \| \adder$ a new input assumption encoded by the
transition relation requiring that whenever $\oq_1 \oq_0 = 10$,
then $\nx{\itoone} = 0$. The input requirement $\oq_1 \oq_0 \neq
00$ of the adder gives rise, in the composite system, to the
requirement that the reset-to-1 occurs early in the count-to-zero
cycle of the counter. \qed

\subsection{Composition and Compatibility, Formally}
Two Synchronous interfaces $\mp$ and $\mq$ are {\em composable\/}
if $\ovars_\mp \inters \ovars_\mq = \emptyset$. If $\mp$ and $\mq$
are composable, we merge them into a single interface $\mr$ as
follows. We let $\ovars_\mr = \ovars_\mp \union \ovars_\mq$ and
$\ivars_\mr = (\ivars_\mp \union \ivars_\mq) \setm \ovars_\mr$.
The output behavior of $\mr$ is simply the joint output behavior
of $\mp$ and $\mq$, since each interface is free to choose how to
update its output variables: hence, $\oinit_\mr = \oinit_\mp \und
\oinit_\mq$ and $\otrans_\mr = \otrans_\mp \und \otrans_\mq$. On
the other hand, we cannot simply adopt the symmetrical definition
for the input assumptions. A syntactic reason is that $\iinit_\mp
\und \iinit_\mq$ and $\itrans_\mp \und \itrans_\mq$ may contain
variables in $(\ovars_\mr)'$. But a deeper reason is that we may
need to strengthen the input assumptions of $\mr$ further, in
order to ensure that the input assumptions of $\mp$ and $\mq$
hold. If we can find such a further strengthening $\iinit$ and
$\itrans$, then $\mp$ and $\mq$ are said to be {\em compatible,}
and $\mr = \mp \| \mq$ with $\iinit_\mr$ and $\itrans_\mr$ being
the weakest such strengthening; otherwise, we say that $\mp$ and
$\mq$ are incompatible, and $\mp\|\mq$ is undefined. Hence,
informally, $\mp$ and $\mq$ are compatible if they can be used
together under some assumptions.

\begin{defi}{(Compatibility and composition of synchronous interfaces)}
\label{def-moore-compatible} For any two synchronous interfaces
$\mp$ and $\mq$, we say that $\mp$ and $\mq$ are {\em
composable\/} if $\ovars_\mp \inters \ovars_\mq = \emptyset$. If
$\mp$ and $\mq$ are composable, let $\ovars_\mr = \ovars_\mp
\union \ovars_\mq$, $\ivars_\mr = (\ivars_\mp \union \ivars_\mq)
\setm \ovars_\mr$, $\vars_\mr = \ovars_\mr \union \ivars_\mr$,
$\oinit_\mr = \oinit_\mp \und \oinit_\mq$, and $\otrans_\mr =
\otrans_\mp \und \otrans_\mq$.

The interfaces $\mp$ and $\mq$ are {\em compatible\/} (written
$\mp\compat\mq$) if they are composable, and if there are
predicates $\iinit$ on $\ivars_\mr$ and $\itrans$ on $\vars_\mr
\union (\ivars_\mr)'$ such that (i)~$\iinit$ is satisfiable;
(ii)~$\forall \vars_\mr . \exists (\ivars_\mr)' . \itrans$ holds;
(iii)~ for all $s_0, s_1, s_2, \ldots \in \traces(\ivars_\mr,
\ovars_\mr, \iinit, \oinit_\mr, \itrans, \otrans_\mr)$ we have
$s_0 \sat \iinit_\mp \und \iinit_\mq$ and, for all $k \geq 0$,
$(s_k, s_{k+1}) \sat \itrans_\mp \und \itrans_\mq$.

The {\em composition\/} $\mr = \mp \| \mq$ is defined if and only
if $\mp\compat\mq$, in which case $\mr$ is obtained by taking for
the input predicate $\iinit_\mr$ and for the input transition
relation $\itrans_\mr$ the weakest predicates such that the above
condition holds. \qed
\end{defi}

\section{Composition Algorithm}
\label{sec:algo}
% The definition illustrates how composition composes not only the output
% behavior, but also the input assumptions.
% The algorithm for computing the predicates $\iinit_\mr$ and
% $\itrans_\mr$ is based on the following idea.
% Two Moore interfaces $\mp$ and $\mq$ are compatible if the choice for
% the next values of $\ivars_\mr$ can be restricted to ensure that the
% input assumptions of $\mp$ and $\mq$ are satisfied.
To compute $\mp \| \mq$, we consider a game between Input and
Output \cite{cav-paper}. At each round of the game, Output chooses
new values for the output variables $\ovars_\mr$ according to
% the output transition relation
$\otrans_\mr$; simultaneously and independently, Input chooses
(unconstrained) new values for the input variables $\ivars_\mr$.
The goal of Input is to ensure that the resulting behavior
satisfies $\iinit_\mp \und \iinit_\mr$ at the initial state, and
$\itrans_\mp \und \itrans_\mq$ at all state transitions. If Input
can win the game, then $\mp$ and $\mq$ are compatible, and the
most general strategy for Input will give rise to $\iinit_\mr$ and
$\itrans_\mr$; otherwise, $\mp$ and $\mq$ are incompatible.
%
% Possible place for example
%
The algorithm for computing $\iinit_\mr$ and $\itrans_\mr$
proceeds by computing iterative approximations to $\itrans_\mr$,
and to the set $\ctrl$ of states from which Input can win the
game. We let $\ctrl_0 = \true$ and, for $k \geq 0$:
%
\begin{equation}
  \label{eq-comp}
  \htau_{k+1} = \forall (\ovars_\mr)' .
        \bigl(\otrans_\mr \imply (\itrans_\mp \und \itrans_\mq \und \ctrl'_k)\bigr)
  \eqspa
  \ctrl_{k+1} = \ctrl_k \und \exists (\ivars_\mr)' . \htau_{k+1} .
\end{equation}
%
Note that $\htau_{k+1}$ is a predicate on $\ovars_\mr \union
\ivars_\mr \union (\ivars_\mr)'$. Hence, $\htau_{k+1}$ ensures
that, regardless of how $\ovars_\mr$ are chosen, from
$\ctrl_{k+1}$ we have that (i)~for one step, $\itrans_\mp$ and
$\itrans_\mq$ are satisfied; and (ii)~the step leads to $\ctrl_k$.
% \mynote{cut?}
% Another way of understanding (\ref{eq-compa})
% is to note that the computation of $\ctrl_{k+1}$ can be rewritten as:
% $
% \exists (\ivars_\mr)' .
% \forall (\ovars_\mr)' .
% (\otrans_\mr \imply (\itrans_\mp \und \itrans_\mq \und \ctrl'_k)),
% $
% which is the classical controllability fixpoint.
Thus, indicating by $\ctrl_* = \lim_{k \go \infty} \ctrl_k$ and
$\htau_* = \lim_{k \go \infty} \htau_k$ the fixpoints of
(\ref{eq-comp}) we have that $\ctrl_*$ represents the set of
states from which Input can win the game, and $\htau_*$ represents
the most liberal Input
strategy % (i.e., the most liberal transition relation for $\ivars_\mr$)
for winning the game. This suggests us to take $\itrans_\mr =
\htau_*$. However, this is not always the weakest choice, as
required by Definition~\ref{def-moore-compatible}: a weaker choice
is $\itrans_\mr = \no \ctrl_* \oder \htau_*$, or equivalently
$\itrans_\mr = \ctrl_* \imply \htau_*$. Contrary to $\itrans_\mr =
\htau_*$, this weaker choice ensures that the interface $\mr$ is
non-blocking. We remark that the choices $\itrans_\mr = \htau_*$
and $\itrans_\mr = \ctrl_* \imply \htau_*$ differ only at
non-reachable states. Since the state-space of $\mr$ is finite, by
monotonicity of (\ref{eq-comp}) we can compute the fixpoint
$\ctrl_*$ and $\htau_*$ in a finite number of iterations.
% and since for all $k \geq 0$
% we have $\ctrl_{k+1} \imply \ctrl_k$ and $\htau_{k+1} \imply \htau_k$, by
% iterating (\ref{eq-comp}) we eventually reach
% $k$ for which $\ctrl_{k+1} \equiv \ctrl_k$ and $\htau_{k+1} \equiv
% \htau_k$, so that the fixpoints can be computed in a finite number of
% iterations.
Finally, we define the input initial condition of $\mr$ by $
  \iinit_\mr = \forall \ovars .
  ( \oinit_\mr \imply (\iinit_\mp \und \iinit_\mq \und \ctrl_*) )
$. The following algorithm summarizes these results.

\begin{algo}{}
Given two composable Synchronous interfaces $\mp$ and $\mq$, let
$\ctrl_0 = \true$, and for $k > 0$, let the predicates $\ctrl_k$
and $\htau_k$ be as defined by (\ref{eq-comp}). Let $\htau_* =
\lim_{k \go \infty} \htau_k$ and $\ctrl_* = \lim_{k \go \infty}
\ctrl_k$; the limits can be computed with a finite number of
iterations, and let $\iinit_* = \forall \ovars . \bigl( \oinit_\mr
\imply (\iinit_\mp \und \iinit_\mq \und \ctrl_*) \bigr)$. Then the
interfaces $\mp$ and $\mq$ are {\em compatible} iff $\iinit_*$ is
satisfiable; in this case their composition $\mr = \mp \| \mq$ is
given by
\[
\begin{array}{rlcrlcrl}
\ovars_\mr  & = \ovars_\mp \union \ovars_\mq & \hspace*{2em} &
\otrans_\mr & = \otrans_\mp \und \otrans_\mq & \hspace*{2em} &
\oinit_\mr  & = \oinit_\mp \und \oinit_\mq \\
%
\ivars_\mr  & = (\ivars_\mp \union \ivars_\mq) \setm \ovars &
\hspace*{2em} & \itrans_\mr & = \ctrl_* \imply \htau_* &
\hspace*{2em} & \iinit_\mr  & = \iinit_*. \hspace{4em} \qed
\end{array}
\]
\end{algo}


\noindent
\paragraph{Implementation Technique.}
\begin{comment}
To obtain an efficient implementation, it is important to use a
conjunctively decomposed representation for the Binary Decision
Diagrams (BDDs) representing the transition relations, and to
derive the new input transition relation for a composed interface
again in conjunctively decomposed form. In fact, the (output)
transition relation $\otrans$ of a module is best represented not
as a single BDD (Binary Decision Diagram), but as a list of BDDs
$\otrans_1, \otrans_2, \ldots, \otrans_n$, such that $\otrans =
\bigwedge_{i=1}^n \otrans_i$: the BDDs $\otrans_1, \otrans_2,
\ldots, \otrans_n$ are often small, and can be constructed
efficiently while parsing the input. For example, in {\em reactive
modules\/} \cite{RM96journal} each $\otrans_i$, for $1 \leq i \leq
n$, can be obtained by parsing a single {\em atom,} which
describes the updates of only a handful of variables;
%: hence, even as the number $|\vars|$ of total variables
%grows, $\max_i |\vars_i|$ is likely to stay constant.
a similar consideration holds for the input language of e.g.\ SMV
\cite{SMV96}. Furthermore, highly efficient {\em image computation
techniques\/} have been implemented for computing formulas of the
form $\exists \vars . \bigwedge_{i=1}^n \otrans_i$; such
techniques are at the heart of symbolic reachability algorithms
\cite{ranjan97}. The input and output transition relations of
Synchronous interfaces in conjunctively decomposed form are
represented as lists of BDDs.
\end{comment}
%
To obtain an efficient implementation, both the input and the
output transition relations should be represented
% of Moore interfaces
using a conjunctively decomposed representation, where a relation
$\trans$ is represented by a list of BDDs $\trans_1, \trans_2,
\ldots, \trans_n$ such that $\trans = \wedge_{i=1}^n \trans_i$.
% Often, such conjunctive representations can be much smaller than
% representing $\trans$ by a single BDD.
% Furthermore, highly efficient {\em image computation techniques\/}
% have been implemented for computing formulas of the form
% $\exists \vars . \bigwedge_{i=1}^n \otrans_i$, that are at
% the heart of symbolic reachability.
When computing $\mr = \mp \| \mq$, the list for $\otrans_\mr$ can
be readily obtained by concatenating the lists for $\otrans_\mp$
and $\otrans_\mq$. Moreover,
% by rearranging the computation of (\ref{eq-comp}),
% we can obtain a conjunctively decomposed representation of
% $\itrans_\mr$ from the conjunctive decompositions for $\mp$ and $\mq$.
assume that $\otrans_\mr$ is represented as $\bigwedge_{i=1}^n
\otrans_i$, and that $\itrans_\mp \und \itrans_\mq$ is represented
as $\bigwedge_{j=1}^m \otrans_j$. Given $\ctrl_k$, from
(\ref{eq-comp}) the conjunctive decomposition is
$\bigwedge_{j=1}^{m+1} \htau_{k+1,j}$ for $\htau_{k+1}$ by taking
$  \htau_{k+1,m+1} = \no \exists (\ovars_\mr)' .
  (\otrans_\mr \und \no \ctrl'_{k})$
and, for $1 \leq j \leq m$, by taking $   \htau_{k+1,j}  = \no
\exists (\ovars_\mr)' .
    (\otrans_\mr \und \no \itrans_j)$.
% \[
%   \htau_{k+1,j}  = \no \exists (\ovars_\mr)' .
%   ((\bigwedge_{i=1}^n \otrans_i) \und \no \itrans_j)
%   \eqspa
%   \htau_{k+1,m+1} = \no \exists (\ovars_\mr)' .
%   ((\bigwedge_{i=1}^n \otrans_i) \und \no \ctrl'_{k}).
% \]
% The existential quantification in these expressions can be
% performed using image computation techniques.
Also $\ctrl_{k+1} = \exists (\ivars_\mr)' . \bigwedge_{j=1}^{m+1}
\htau_{k+1,j}$. All these operations can be performed using image
computation techniques. On reaching $k$ such that $\ctrl_k \equiv
\ctrl_{k+1}$, the BDDs $\htau_{k,1}, \ldots, \htau_{k,m+1}$ form a
conjunctive decomposition for $\htau_*$. Since the two transition
relations $\htau_*$ and $\ctrl_* \imply \htau_*$ differ only for
the behavior at non-reachable states, we can take directly
$\itrans_\mr = \htau_*$, obtaining again a conjunctive
decomposition.
