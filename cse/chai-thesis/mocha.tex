\chapter{The Starting Point- \mocha}
\chai \ is based on \mocha \, it extends the functionality of
\mocha \ and adds an input language to suit interfaces.

\mocha \ is an interactive environment for modular verification of
heterogeneous systems. \mocha \ relies on the modelling framework
of reactive modules unlike traditional use of state transition
graphs. Its input language is machine readable variant of reactive
modules. In \chai \ we extend \rm \ to interface modules as
explained in next chapter.

\section{Mocha}
%Some lines on mocha implementation

%This section describes Mocha in summary detail.
\mocha \ supports the following functionalities:
\begin{itemize}
\item Simulation, including games between the user and the
simulator \item Enumerative and symbolic invariant checking and
error-trace generation \item Compositional refinement checking
\item ATL model checking \item reachability analysis of real-time
systems
\end{itemize}

The basic structuring units, or the molecules of a system, are
{\em reactive modules}. The modules have a well-defined interface
given by a set of \emph{external (or input)} variables and a set
of {\em interface (or output)} variables. A module may also have a
set of {\em private} variables. All variables are typed, and
\mocha supports a standard set of finite and infinite types, such
as booleans and integers. A module is built from \emph{atoms},
each grouping together a set of \emph{controlled} (interface or
private) variables with exclusive updating rights. \emph{Updating}
is defined by two nondeterministic guarded commands: an
\emph{initialization} command and an \emph{update} command.  In
these commands unprimed variables, such as $x$, refer to the old
value of the corresponding variable, and primed variables, such as
$x'$, refer to the new value of the corresponding variable.  An
atom is said to \emph{await} another atom if its initialization or
update commands refer to primed variables that are controlled by
the other atom. The variables change their values over time in a
sequence of \emph{rounds}. The first round consists of the
execution of the initialization command of each atom, and the
subsequent rounds consist of the execution of the update command
of each atom,  in an order consistent with the await dependencies.
A round of an atom is therefore a \emph{subround} of the module.
If no guard of the update command is enabled, then the atom idles,
i.e., the values of the variables do not change.  %If the update
%command of an atom has a branch with a true guard and no updating
%action, then it may at any time either take a transition or idle.
%Such an atom is called \emph{lazy}, and is useful for modelling
%asynchronous interaction.
%
%%%Use a better example ?
%For example, consider the specification of a village telephone
%system that contains four telephones. The specification consists
%of two modules: the first one models the environment, i.e., the
%users, and the second one models the system. A phone is either
%on-hook or off-hook, and the module {\tt UserSpec}
%nondeterministically toggles at most one telephone between on-hook
%and off-hook.
%\begin{example}{}
%\begin{verbatim}
%type hookType is {on, off} module UserSpec is
%  interface h1,h2,h3,h4: hookType;
%lazy atom ToggleHook
%  controls h1,h2,h3,h4
%  reads h1,h2,h3,h4
%  init
%    [] true -> h1' := on; h2' := on; ...
%  update
%    [] h1 = on -> h1' := off;
%    [] h1 = off -> h1' := on;
%    ...
%  endatom
%endmodule
%\end{verbatim}
%\end{example}

Modules can be \emph{composed} if they have disjoint sets of
interface variables, and their union of atom sets does not contain
a circular await dependency. Given a specification {\tt
SystemSpec} of a system and model of user behavior as {\tt
UserSpec}, specification module {\tt Spec} is defined as:
\vspace*{-2ex}
\begin{footnotesize}
\begin{verbatim}
module Spec is UserSpec || SystemSpec
\end{verbatim}
\end{footnotesize}
\vspace*{-2ex} For encapsulation \rem{} allows the {\em hiding} of
interface variables, and for instantiation it allows the {\em
renaming} of interface and external variables. Hiding and parallel
composition permit hierarchical descriptions of complex systems.

\section{Atoms and Reactive Modules, Formally}
%Define guarded commands
%Modules collect variables
This section describes the syntax and semantics of Reactive
Modules.

%Atom definition
\begin{defi}{atom}\it
  {\em [{\bf Atom}]}
  Let $\vars$ be a finite set of typed variables.
  An {\em $\vars$-atom\/} $\atom$ consists of a declaration and a body.
  The atom declaration consists of a set $\cvars{\atom}\subseteq\vars$ of
  {\em controlled variables}, a set $\mrvars{\atom}\subseteq\vars$ of
  {\em read variables}, and a set
  $\wvars{\atom}\subseteq\vars\bs\cvars{\atom}$ of {\em awaited variables}.
  The atom body consists of an executable {\em initial action\/}
  $\init{\atom}$ from $\wvars{\atom}'$ to $\cvars{\atom}'$ and an executable
  {\em update action\/} $\update{\atom}$ from
  $\mrvars{\atom}\cup\wvars{\atom}'$ to~$\cvars{\atom}'$.
\end{defi}

\mypar In the initialization round, the initial action of the atom
$\atom$ assigns initial values to the controlled variables as a
nondeterministic function of the initial values of the awaited
variables. In each update round, the update action of $\atom$
assigns updated values to the controlled variables as a
nondeterministic function of the latched values of the read
variables and the updated values of the awaited variables. Hence,
in each round, the $\atom$-subround can take place only after all
awaited variables have already been updated. The variable $\var$
{\em awaits\/} the variable~$\avar$, written
$\var\awaits_\atom\avar$, if $\var\in\cvars{\atom}$ and
$\avar\in\wvars{\atom}$.

\begin{table}
\begin{center}
\begin{tabular}{|c|c|c|}
  \hline
  $\priv{\ro}$ &$\intf{\ro}$&$\extl{\ro}$\\
  \hline
  \multicolumn{2}{|c|}{$\cvars{\ro}$} & \\
  \hline
  &\multicolumn{2}{c|}{$\ovars{\ro}$}\\
  \hline
  \multicolumn{3}{|c|}{$\rovars{\ro}$}\\
  \hline
\end{tabular}
\end{center}
\caption{Reactive Module variables}
\end{table}

\begin{defi}{module}\it
  {\em [{\bf Module}]}
  A {\em (reactive) module\/} $\ro$ consists of a declaration and a body.
  The module declaration is a finite set $\rovars{\ro}$ of typed variables
  that is partitioned as shown in Table~3.1.
  The module body is a set $\roatoms{\ro}$ of $\rovars{\ro}$-atoms such that
  (1)~$(\cup_{\atom\in\roatoms{\ro}}\,\cvars{\atom})=\cvars{\ro}$;
  (2)~for all atoms $\atom$ and $\aatom$ in~$\roatoms{\ro}$,
    $\cvars{\atom}\cap\cvars{\aatom}=\emptyset$;
    and
  (3)~the transitive closure
    $\awaits_\ro\,=(\cup_{\atom\in\roatoms{\ro}}\awaits_\atom)^+$ is
    asymmetric.
\end{defi}

\mypar The first two conditions ensure that the atoms of $\ro$
control precisely the variables in $\cvars{\ro}$, and that each
variable in $\cvars{\ro}$ is controlled by precisely one atom. The
third condition ensures that the await dependencies among the
variables in $\rovars{\ro}$ are acyclic. A linear order
$\atom_0,\ldots,\atom_{k-1}$ of the atoms in $\roatoms{\ro}$ is
{\em consistent\/} if for all $0\le i<j<k$, the awaited variables
of $\atom_i$ are disjoint from the controlled variables
of~$\atom_j$. The asymmetry of $\awaits_\ro$ ensures that there
exists a consistent order of the atoms in~$\roatoms{\ro}$.

%Partial ordering in atoms
