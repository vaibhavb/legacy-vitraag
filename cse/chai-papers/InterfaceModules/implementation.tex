\section{Implementation Issues}

We have developed and implemented symbolic algorithms for composition and 
compatibility and refinement checking of bidirectional A/G interfaces.

Our tool is implemented in Java. It uses a front-end parser generated by 
the SableCC parser generator, and the Java interface to the CUDD BDD Package
from JMocha.

In our current implementation the assumptions and guarantees are in terms of 
{\em port variables} and are represented and manipulated symbolically, 
while the {\em states} of the interfaces are represented explicitly, which
in our experience is not a problem because interfaces usually have very few 
states. 

To achieve a purely symbolic implementation one can encode states using 
boolean variables. Let $s_{curr}$ and $s_{next}$ be the set of current and
next state variables, and $p$ be the set of port variables. Functions 
$\lambda_{o}$ and $\lambda_{o^+}$ over variables in $s_{curr}$ and $p$ 
represent the set of variables that are outputs or availables in a certain
state. The transitions and guard conditions are represented by the formula
$\tau$ over variables in $s_{curr}$, $s_{next}$, and $p$. The input assumptions
and output guarantees are represented by $\Phi(s_{curr},p)$ and 
$\Psi(s_{curr},p)$ respectively. On composition, the set of current and next 
state variables of the composition is the union of those of its components
respectively, and the $\lambda$, $\tau$ and $\Psi$ functions are 
conjunctively composed as expected. A problem arises when we need to 
quantify a formula over a set of variables, where the set of variables 
is a function of the state, in course of finding the input assumption 
function $\Phi(s_{curr},p)$ of the composition. As an example, the 
expression for initialising the input assumptions can be written as: \\
$\Phi_C = [\bigwedge_{i \in [p]} \lambda_o(s_{curr},p) \Rightarrow 
(\forall p_1,p_2,...p_n. \Psi_C \Rightarrow (\Phi_M \land \Phi_N))]$ \\
where $C$ is the compostion of $M$ and $N$ and $\{p_1,p_2,....p_n\} \subseteq P_C$
is the set of ports that are assigned to $true$ by valuation $i$.

The formula we thus obtain is of size exponential in the number of ports. A 
similar situation arises when we need to find out if some valuation $i$
of the variables in $s_{curr}$ exists that makes $\Phi(s_{curr},p)_i$ 
unsatisfiable but $\Psi(s_{curr},p)_i$ satisfiable, where 
$\Phi(s_{curr},p)_i$ for instance, stands for the formula obtained from
$\Phi(s_{curr},p)$ by substituting variables in $s_{curr}$ with values 
according to $i$. Here again, the formula we need to evaluate involves 
a conjunction of terms, each obtained for a particular valuation on 
$s_{curr}$, and is thus exponential in the number of state variables, 
and thus polynomial(actually linear) in the number of states in the 
explicit representation, which is quite satisfactory.


