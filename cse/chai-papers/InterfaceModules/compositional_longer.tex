\section{Compositional Verification} 

The goal of compositional methods is to prove that an implementation
satisfies a specification by reasoning separately on the various
components of the implementation. 
When the components $\mp$ and $\mq$ of a design are implemented as
$\mp'$ and $\mq'$, to prove the correctness of the implementation it
is necessary to prove the refinement $(\mp' \| \mq) \refi (\mp \| \mq)$.
The simplest approach to proving this refinement consists in proving
separately that $\mp' \refi \mp$ and $\mq' \refi \mq$. 
Unfortunately, this approach seldom works: usually, the
implementation $\mp'$ refines the specification $\mp$ only 
when it receives suitable inputs from an 
appropriate environment such as, hopefully, $\mq'$;
similarly for $\mq'$ and $\mq$. 
Various improvements to this basic rule have been proposed, based on
the idea of using an environment $\env_{\mp'}$ to restrict the inputs
to $\mp'$, and proving $\mp' \| \env_{\mp'} \refi \mp$ (and
symmetrically for $\mq'$). 
For instance, the following two rules have been proposed: 
\[
  \frac{\mp' \| \mq \refi \mp; \hspace{1em} \mq' \| \mp \refi \mq}{%
	\mp' \| \mq' \refi \mp \| \mq} \hspace{1em} (\mbox{\rm AG-SPEC})
  \hspace{3em} 
  \frac{\mp' \| \env_{\mp'} \refi \mp; \hspace{1em} 
	\mq' \| \env_{\mq'} \refi \mq}{%
	\mp' \| \mq' \refi \mp \| \mq} \hspace{1em} (\mbox{\rm AG-IMPL})
\]
In Rule~(AG-SPEC), the environment for $\mp'$ is the specification
$\mq$ \cite{RM96journal,FMCAD98}; 
in Rule~(AG-IMPL), the environment $\env_{\mp'}$ for $\mp'$ describes 
$\mq'$ and is either provided by the user, or is obtained 
with approximate automatic methods \cite{concur99freddy}. 
Since interfaces can represent input restrictions directly, instead of
providing an environment $\env_{\mp'}$ for $\mp'$, we can restrict the
input assumptions of $\mp'$ to reflect the inputs that can occur in
its actual environment $\mq'$. 
Given two compatible interfaces $\mp'$ and $\mq'$, we call the 
{\em adaptation of $\mp'$ to the environment $\mq'$\/}
%(denoted $\adapt{\mp'}{\mq'}$) 
a strenghening of the input assumptions of $\mp'$ that reflects the
inputs that $\mp'$ can receive from $\mq'$ in $\mp'\|\mq'$. 

\begin{defi}{(adaptation to environment)}
\label{def-adapt} 
Given two compatible interfaces $\mp$ and $\mq$, an 
{\em adaptation of $\mp$ to $\mq$\/} is any interface 
$\wh{\mp}$ that differs from $\mp$ only for the input assumptions, and
such that
(i)~$\mp \refi \wh{\mp}$, 
(ii)~$\wh{\mp} \compat \mq$ iff $\mp \compat \mp$, 
and (iii)~if $\mp \compat \mq$, then for all interfaces $\mr$,  
we have that 
$(\wh{\mp} \| \mq) \compat\mr$ iff $(\mp\|\mq)\compat\mr$.
\end{defi}

\noindent
Condition~(i) ensures that the input invariants of $\wh{\mp}$
do not weaken those of $\mp$.
Conditions~(ii) and~(iii) together require that $\mp\|\mq\|\mr$ is
defined iff $\wh{\mp}\|\mq\|\mr$ is defined: hence, the input
strengthening cannot rule out any input that $\mp$ will receive when
in the environment of $\mq$. 
There are several ways of adapting an interface $\mp$ to its
environment $\mq$. 
The following proposition describes a method for Moore interfaces that
is informed by the technique of \cite{concur99freddy} for the
automated construction of environments.
Given a set $\vars$ of variables, a predicate $\init$ on $\vars$, 
and a predicate $\trans$ on $\vars \union \vars'$, 
define $\reach(\vars,\init,\trans)$ to be the fixpoint of the
reachability computation $R_0 = \init$, and 
$R'_{k+1} = R'_k \oder \exists \vars . (R_k \und \trans)$, for $k \geq 0$.
The proposition strengthens the input assumptions of $\mp$ on the
basis of both the output {\em and the input\/} behavior of $\mq$. 

\begin{prop}{} 
\label{prop-adapt} 
Given two Moore interfaces $\mp$ and $\mq$, 
let $\avars = \vars_\mq \setm \vars_\mp$,
and let $\wh{\mp}$ be obtained
from $\mp$ by replacing $\iinit_\mp$ with 
$\iinit_\mp \und \exists \avars . \oinit_\mq$, 
and by replacing $\itrans_\mp$ with 
$\itrans_\mp \und \exists (\avars \union \avars') . 
(\reach(\vars_\mq,\oinit_\mq \und \iinit_\mq,\otrans_\mq \und
\itrans_\mq) \und \otrans_\mq)$.
Then, $\wh{\mp}$ is an adaptation of $\mp$ to $\mq$. 
\end{prop}

\noindent
The subtlety of the proposition lies in the fact that, to adapt 
$\mp$, we use a reachability predicate for $\mq$ that is computed
under the assumption that the input assumptions of $\mq$ are
respected, even though $\mp$ itself may violate them. 
The correctness of the proposition depends on the fact that, if $\mp$
violates the input assumptions of $\mq$, then so does the adaptation
$\wh{\mp}$, so that in Definition~\ref{def-adapt} it will be neither
$\mp \compat \mq$ nor $\wh{\mp} \compat \mq$. 
The following theorem states that, in proving a compositional
refinement, we can adapt each interface to its environment.
In the theorem, and in the discussion below, we use the notation 
$\adapt{\mp}{\mq}$ to denote an interface that is an adaptation of
$\mp$ to $\mq$. 

\begin{theo}{(compositional refinement and adaptation)} 
\label{theo-comp-refi} 
For all interfaces $\mp, \mq, \mp', \mq'$ such that 
(i)~$\mp \compat \mq$, 
(ii)~$\mp'$ and $\mq'$ are composable (but not necessarily
compatible),
(iii)~$(\ovars_{\mp'} \inters \ivars_\mq) \subs (\ovars_\mp \inters \ivars_\mq)$,
and 
(iv)~$(\ovars_{\mq'} \inters \ivars_\mp) \subs (\ovars_\mq \inters \ivars_\mp)$,
the following verification rule is valid: 
\[
  \frac{\adapt{\mp'}{\mq'} \refi \adapt{\mp}{\mq}; \hspace{1em} 
        \adapt{\mq'}{\mp'} \refi \adapt{\mq}{\mp}}{
	\mp' \compat \mq' \tand \ 
	\mp' \| \mq' \refi \mp \| \mq} \hspace{1em} (\mbox{\rm AG-INTF})
\]
\end{theo}

\noindent
The proviso over the variables is necessary to ensure that the
compatibility of $\mp$ and $\mq$ implies that of $\mp'$ and $\mq'$,
once the refinement is proved. 
The theorem states a circular assume-guarantee principle: in fact, 
as illustrated by Proposition~\ref{prop-adapt}, the adaptation of
$\mp'$ on $\mq'$ can be constructed using the fact that the input
assumptions of $\mq'$ are respected, even though the compatibility of
$\mp'$ and $\mq'$ is a conclusion of the rule, rather than a premise. 
It is easy to see that rule (AG-INTF) generalizes (AG-IMPL); to see
that it also generalizes (AG-SPEC), 
note that rule (AG-SPEC) can be restated as 
$(\mp' \refi \adapt{\mp}{\mq}; \mq' \refi \adapt{\mq}{\mp})/
(\mp' \| \mq' \refi \mp \| \mq)$. \mynote{Hey this is a cool notation!}
In fact, our alternating notion of refinement requires the
implementation to match the specification only when the implementation
is subject to the same inputs as the specification. 
Hence, if we restrict the input assumptions of the specification (as
in $\adapt{\mp}{\mq}$), we need to check that the refinement holds only
when the implementation $\mp'$ is subject to similarly restricted
inputs: thus, checking the alternating refinement 
$\mp' \refi \adapt{\mp}{\mq}$ corresponds to checking the regular
refinement $\mp' \| \mq \refi \mp$ (where $\mp$, $\mq$, and $\mp'$ are
interpreted as regular modules, rather than interfaces). 
A symmetrical argument holds for the other premise. 
Thus, Theorem~\ref{theo-comp-refi} highlights how the essence of
compositional refinement checking lies in studying both implementation
and specification components adapted to their actual environment. 





