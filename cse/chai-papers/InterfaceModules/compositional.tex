\subsection{Compositional verification} 

In compositional verification, one attempts to verify that an
implementation $\mp' \| \mq'$ refines a specificaton $\mp \| \mq$ 
by reasoning separately about $\mp'$ and $\mq'$. 
The simplest approach consists in proving that $\mp' \refi \mp$ and 
$\mq' \refi \mq$, but this seldom succeeds: usually, $\mp'$ refines
$\mp$ only in an environment (such as, hopefully, $\mq'$) that
provides it with the appropriate input. 
This approach has been refined by allowing the use of an environment
$\env_{\mp'}$ to restrict the possible inputs to $\mp$; 
then, in order to prove $\mp' \| \mq' \refi \mp \| \mq$, we show 
that both $\mp' \| \env_{\mp'} \refi \mp$ and 
$\mq' \| \env_{\mq'} \refi \mq$ hold. 
This approach is an instance of {\em assume-guarantee\/} reasoning 
\cite{Stark85,AbLa95,RM96,McMillan97}.
In \cite{RM96}, the environment $\env_{\mp'}$ is taken to be
$\mq$ (rule AG-SPEC):  
the specification itself is used to constrain the implementation; 
in \cite{concur99freddy}, $\env_{\mp'}$ is derived automatically from
$\mq$ (rule AG-IMPL), so that it reflects the implementation
(and symmetrically for $\env_{\mq'}$). 

In our interface formalisms we can restrict the possible inputs to a
component directly, without resorting to auxiliary environments, and
we can restate the above rules in a more direct and general way.
In the composition $\mp \| \mq$, the inputs of $\mp$ that are connected
to $\mq$ will only see output values that can be produced by $\mq$,
rather than arbitrary values; 
we can represent this constraint by an additional input restriction
for $\mp$. 
There are several possible ways of deriving these input restrictions 
from $\mq$;
for lack of space, we discuss here only 
an approach based on \cite{concur99freddy}. 
The approach, presented for Moore interfaces, can be readily adapted
to bidirectional A/G interfaces. 
Given two Moore interfaces $\mp$ and $\mq$, let $R$ be the predicate
defining the set of reachable states of $\mq$, computed as usual as
the fixpoint of the reachability computation $R_0 = \iinit_\mq \und
\oinit_\mq$, and $R'_{k+1} = R'_k \oder \exists \vars_\mq .  (R_k \und
\itrans_\mq \und \otrans_\mq)$, for $k \geq 0$.
%
Let $\avars$ be a set of variables that we wish to abstract away, 
let $\avars_\mq = (\vars_\mq \setm \vars_\mp) \inters \avars$
be the subset of those variables that appear in $\mq$ only, 
and let $\avars_\mp = \vars_\mq \setm (\vars_\mp \union \avars)$
be the variables of $\mq$ that we wish to include as inputs in $\mp$. 
%
Define $\adapt{\mp}{\mq}{\avars}$  to be the interface obtained
from $\mp$ by replacing $\ivars_\mp$ with 
$\ivars_\mp \union \avars_\mp$, 
%
by replacing $\iinit_\mp$ with 
$\iinit_\mp \und \exists \avars_\mq . \oinit_\mq$, 
%
and by replacing $\itrans_\mp$ with 
$\itrans_\mp \und \exists (\avars_\mq \union \avars_\mq') . 
(\reach(\vars_\mq,\oinit_\mq \und \iinit_\mq,\otrans_\mq \und
\itrans_\mq) \und \otrans_\mq)$.
%
Thus, the interface $\adapt{\mp}{\mq}{\avars}$ is obtained by
restricting the input assumptions of $\mp$ to reflect the possible
outputs of $\mq$. 
We can then state the following verification rule. 

\begin{theo}{(compositional refinement and adaptation)} 
\label{theo-comp-refi} 
For all interfaces $\mp, \mq, \mp', \mq'$ such that 
(i)~$\mp \compat \mq$, 
(ii)~$\mp'$ and $\mq'$ are composable (but not necessarily
compatible),
(iii)~$(\ovars_{\mp'} \inters \ivars_\mq) \subs (\ovars_\mp \inters \ivars_\mq)$,
and 
(iv)~$(\ovars_{\mq'} \inters \ivars_\mp) \subs (\ovars_\mq \inters \ivars_\mp)$,
and for all sets of variables $\avars_1$, $\avars_2$, $\avars_3$,
$\avars_4$, the following verification rule is valid: 
\[
  \frac{\adapt{\mp'}{\mq'}{\avars_1} \refi \adapt{\mp}{\mq}{\avars_2}; \hspace{1em} 
        \adapt{\mq'}{\mp'}{\avars_3} \refi \adapt{\mq}{\mp}{\avars_4}}{
	\mp' \compat \mq' \tand \ 
	\mp' \| \mq' \refi \mp \| \mq} \hspace{1em} (\mbox{\rm AG-INTF})
\]
\end{theo}

\noindent
In the rule, the input assumptions of both implementation and the
specification interfaces have been restricted to reflect their actual
environment. 
The proviso over the variables is necessary to ensure that the
compatibility of $\mp$ and $\mq$ implies that of $\mp'$ and $\mq'$,
once the refinement is proved. 
The theorem states a circular assume-guarantee principle: in fact,
$\adapt{\mp'}{\mq'}{\avars_1}$ is constructed using the fact that the
input assumptions of $\mq'$ are respected, even though the
compatibility of $\mp'$ and $\mq'$ is a conclusion of the rule, rather
than a premise; symmetrically for $\adapt{\mq'}{\mp'}{\avars_3}$.


It is easy to see that rule (AG-INTF) generalizes (AG-IMPL); to see
that it also generalizes (AG-SPEC), 
note that rule (AG-SPEC) can be restated as 
$(\mp' \refi \adapt{\mp}{\mq}{\emptyset}; 
  \mq' \refi \adapt{\mq}{\mp}{\emptyset})/
 (\mp' \| \mq' \refi \mp \| \mq)$.
In fact, our alternating notion of refinement requires the
implementation to match the specification only when the implementation
is subject to the same inputs as the specification. 
Hence, if we restrict the input assumptions of the specification (as
in $\adapt{\mp}{\mq}{\emptyset}$), we need to check that the
refinement holds only when the implementation $\mp'$ is subject to
similarly restricted inputs: thus, checking the alternating refinement  
$\mp' \refi \adapt{\mp}{\mq}{\emptyset}$ corresponds to checking the
regular refinement $\mp' \| \mq \refi \mp$ (where $\mp$, $\mq$, and
$\mp'$ are interpreted as regular modules, rather than interfaces). 
A symmetrical argument holds for the other premise. 





