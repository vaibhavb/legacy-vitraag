\chapter{Introduction}
\chai\ is an extension of \mocha\ verification environment.
Here we describe \chai\ as the realm of interface verification.

\chai\ is intended as a vehicle for the development of new
verification algorithms and approaches.  It adopts a software
architecture similar to \vis \cite{BHS96}, a symbolic model-checking
tool from UC Berkeley.  Written in C with Tcl/Tk and Tix
\cite{tix-http}, \chai\ can be easily extended in two ways: designers
and application developers can customize their application or design
their own graphical user interface by writing Tcl scripts; algorithm
developers and researchers can develop new verification algorithms by
writing C code, or assembling any verification packages through C
interfaces.  For instance, \chai\ incorporates the \vis packages
for image computation and multi-valued function manipulation, as well
as various BDD packages, to provide state-of-the-art verification
techniques.