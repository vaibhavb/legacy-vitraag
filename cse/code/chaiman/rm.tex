\label{chap:rem}
% should be chapter. 

The input language that \mocha\  uses for language description is that
of {\em reactive modules.}  The language of reactive modules is
vaguely similar to a programming language.  However, \rem\ have a
different emphasis than most programming languages.  Programs written
in ordinary programming languages are usually meant to describe
procedures or processes in full detail, to enable their efficient
execution. On the other hand, it may not be desirable (or indeed
possible) to encode all the details of a system into a \rem\
description.  Hence non-determinism plays in \rem\ a more important
role than in ordinary programming languages, as it enables to abstract
from such details. \rem\ also provides extensive facilities for
the modular description of a system, and for modeling both synchronous
and asynchronous types of behavior. 

The structure of a \rem\ description resembles that of a conventional
(imperative) programming language: to the statements of the language
correspond {\em atoms,} and to the procedures correspond {\em reactive
modules\/} (modules, for short).  A complete description consists of
one or more modules.  

%The structure of a \rem\ description is depicted
%in Table~\ref{table-rem-structure}.



\section{The input file}

\paragraph{Structure of the file.}

A file contanining a \rem\ description is usually given a name ending
with the {\tt .rm} extension, which is what \mocha\  assumes by default.
Keywords in \rem\ can be separated by any type of {\em
whitespace,}\index{white space} where whitespace can be a space, a
new-line, or a tab.  Comments\index{comments} in \rem\ are introduced
with the character sequence {\tt --}: anything from {\tt --} to the
end of the line is considered as comment, and disregarded by \mocha.

\paragraph{Splitting a description in multiple files.}

A long (or short) \rem\ description can be split into multiple
files\index{files!multiple} for ease of handling.  Each module must be
contained in a single file.  The complete description can be loaded
into \mocha\ simply by loading the separate files one after the other,
as if reading a single file in successive steps.

\paragraph{Typographical conventions.}

In this manual, we write in typewriter font the keywords, such as
\ATOM, that have to be entered as specified, and we write in italics
the items, such as {\it var,} $x$, for which you have to substitute
the appropriate values.  We use also the following special symbols: 
%
\begin{itemize}
  \item $\ldots$ denotes an arbitrary number of items in a list, as in
  $x_1$\COMA\ \ldots \COMA$x_n$, where we assume that $n \geq 1$.  The separator
  between list elements, when present (``,'' in this example)
  indicates the separator that has to be used between list 
  items.

  \item $|$ denotes alternative: in \TRUE\ $|$ \FALSE, you can write
  either \TRUE\ or \FALSE.

  \item $[\cdot]$ denotes an option: in [ \atomname\ ], the input
  \atomname\ can be provided or omitted.

  \item round parentheses in roman typeface, as in $( \cdot )$, are
  used for grouping. 
\end{itemize}
%
%The complete grammar of \rem\ is provided in Chapter~\ref{chap:grammar}.


\section{Atoms}

The state of the system is described by a set of {\em state
variables:\/} each system state corresponds to an assignment of values
to the variables.  The behavior of the system consists in an {\em
initial round,}\index{round!initial} which initializes
the variables to their initial values, followed by an infinite
sequence of {\em update rounds,}\index{round!update} which assign new
values to the variables, thus describing the evolution of the system's
state.  You can also think of the initial round and the update
rounds as the two elements that describe a transition system: the
possible outcomes of the initial round correspond to the
initial states of the transition system, and the update rounds define
the transition relation. Atoms and modules are used to specify the
initial and update update rounds for all the variables.


\subsection{Atoms}

\begin{table}\boxed{
\begin{tabbing}
\atomid\ \gramdef\ \= [\LAZY] \ATOM\ [\atomname] \\
\> \ind \CONTROLS\ $\varc_1$\COMA\ \ldots \COMA$\varc_i$ \\
\> \ind [\READS\ $\varr_1$\COMA\ \ldots \COMA$\varr_j$] \\
\> \ind [\AWAITS\ $\vara_1$\COMA\ \ldots \COMA$\vara_k$] \\
\> \protect[\INIT \\
\> \ind [$\gcommand_1$ \\
\> \ind $\cdots$ \\
\> \ind $\gcommand_m$]] \\
\> \UPDATE \\
\> \ind $\gcommand_1$ \\
\> \ind $\cdots$ \\
\> \ind $\gcommand_n$ \\
\> \ENDATOM \\[3ex]
$\varc$ \gramdef\ \ident[\adeclare] \qquad $\varr$ \gramdef\ $\varc$ \qquad $\vara$ \gramdef\ $\varc$ \\[3ex]
\atomname\ \gramdef\ \ident
\end{tabbing}}
\caption{Atom syntax}
\label{table-atom-syntax}
\end{table}

An {\em atom\/}\index{atom} is the basic unit used to describe the
initial condition and transition relation of a group of related
variables.  The syntax of an \emph{atom} is given in Table~\ref{table-atom-syntax}.
%\footnote{Here, I
%simply wrote vars, and postponed the discussion of bitfields. I would
%like to write it later, for simplicity, and I also need help.}
The keyword \LAZY\ is optional. Its meaning will be described in
Section~\ref{sec-lazyatoms}.  The atom has an (optional) name
\atomname\ \index{atom!name} which is used only as a reminder of the atom's purpose.
The \ident\ is an alphanumerical string which begins with a letter.
The \ident\ may also contain ``{\tt \_}'' (underscores) and ``{\tt .}'' (dots).

An atom has three types of variables:
the {\em controlled\/} variables $\varc_1, \ldots, \varc_i$,
the {\em read\/} variables $\varr_1, \ldots, \varr_j$, and
the {\em awaited\/} variables $\vara_1, \ldots, \vara_k$.
An admissible identifier for a variable is an \ident\ followed
by the optional expression \adeclare\ which restricts the declaration
to particular elements of an array structure.
The exact meaning will be described in Section~\ref{sec-arrays}.

\paragraph{Controlled variables.}\index{variable!controlled}

The controlled variables of the atom represent the
variables for which the atom can establish the value at the next
round.  Each variable is controlled by at most one atom: this insures
that no conflict can arise between different atoms trying to update
the same variable to different values. 


\paragraph{Read variables.}\index{variable!read}

The read variables of an atom are the variables whose current value
can be read by an atom in order to decide the next values of the
controlled variables.  Precisely, if a variable $x$ is not read, then
the new values of the controlled variables do not depend on the
current value of $x$. 


\paragraph{Awaited variables.}\index{variable!awaited}

The awaited variables of an atom are the variables whose {\em next
value\/} can be read by an atom in order to decide the next value of
the controlled variables.  A variable cannot be both awaited and
controlled: otherwise, in order to determine the next value of the
variable, you would have to already know this next value --- such a
circularity can be often problematic.  In order to avoid all such
circularity problems, \mocha\ keeps track of the {\em awaits
relation\/}\index{awaits relation} $\awaits$: if an atom controls a
variable $x$ and awaits a variable $y$, then $x \awaits y$ holds.
\mocha\ then computes a global relation $\awaits_g$ by taking the
union of the $\awaits$ relations for all the atoms.  To avoid await
circularities, \mocha\ checks that the relation $\awaits_g$ is {\em
acyclical.}


\paragraph{Guarded commands.}\index{guarded command}\index{command!guarded}

The $\gcommand$ statements following the \INIT\
keyword\index{init@\INIT} specify the values of the controlled
variables at the end of the initial round;\index{round!initial} the
$\gcommand$ statements following the \UPDATE\
keyword\index{update@\UPDATE} specify the values of the controlled
variables at the end of an update round.\index{round!update} The
syntax of guarded commands is presented in
Table~\ref{table-gcommand-syntax}.  A guarded command statement
consists of two parts: a {\em guard,}\index{guard} that is a boolean
expression specifying when the guarded command can be executed, and a
list of {\em commands,}\index{command} used to specify the next value
of the controlled variables.  The special guard
\DEFAULT\ will be discussed in Section~\ref{sec-default}.


\begin{table}\boxed{\begin{tabbing}
\gcommand\ \= \gramdef\ \choice\ \guard\ \GUARDARROW\ 
        [$\command_1$\SEMICOLON\ \ldots \SEMICOLON$\command_m$] \\[3ex]
\guard\ \gramdef\ \boolexpr\ $|$ \DEFAULT
\end{tabbing}}
\caption{Guarded command syntax\index{guarded command}\index{command!guarded}}
\label{table-gcommand-syntax}
\end{table}


\subsection{Simple examples of atoms}


The atom of Figure~\ref{fig-atomsimple} controls a variable {\tt x},
which we assume to be of type integer (we will explain later how to
specify types for variables).  The atom specifies that the variable
{\tt x} has initially value $0$, and that the value of {\tt x} is
either decremented or incremented by~1 at each round.  Thus, variable
{\tt x} performs an (unbounded) random walk.  Note that {\tt x} must
also be listed among the read variables: otherwise, the value of {\tt
x} at the end of the round would not be able to depend on the current
value of {\tt x}!

\begin{figure}
\begin{verbatim}
atom incrdecr
  controls x
  reads x
init
  [] true -> x' := 0
update 
  [] true -> x' := x + 1
  [] true -> x' := x - 1
endatom
\end{verbatim}
\caption{Atom {\tt incrdecr}}
\label{fig-atomsimple}
\end{figure}

In the atom, the guards of all the guarded commands are always true
--- in fact, they are equal to the boolean constant \TRUE.  Given a
variable $x$, you can access the current value of $x$ (i.e.\ the value
of $x$ at the beginning of the round) by writing $x$; you can access
the {\em next\/} value of $x$ (i.e.\ the value of $x$ at the end of
the round) by writing $x'$.  The guarded commands for the initial
round\index{round!initial} cannot access the current (unprimed) value
of the variables, since such value has not been defined yet --- the
purpose of the initial round is to define such initial value!

The guards of guarded commands need not be mutually exclusive: the
guarded command that is executed is selected non-deterministically
among the ones whose guards are true.  Hence, according to the above
atom each state has two successor states: one in which the value of
{\tt x} is increased by~1, and one in which the value of {\tt x} is
decreased by~1.

You cannot feed the above atom as input to \mocha: the smallest unit of
input to the \mocha\  parser is a module, not an atom.  To give the atom
as input to \mocha, you can embed it into a minimal module as shown in
Figure~\ref{fig-randomwalk}. 
In this module declaration, variable {\tt x} is declared of type {\em
int\/} (the \mocha\  shorthand for integer), and is listed as an {\em
interface\/} variable of the module, indicating that the module can
modify it, and it is visible from outside of the module as well (so
other modules can look at its value). 

\begin{figure}
% parses ok 
\begin{verbatim}
module randomwalk
  interface x : int 

  atom incrdecr
    controls x
    reads x 
  init
    [] true -> x' := 0
  update 
    [] true -> x' := x + 1
    [] true -> x' := x - 1
  endatom
endmodule 
\end{verbatim}
\caption{Module {\tt randomwalk}}
\label{fig-randomwalk}
\end{figure}

\subsubsection{Unit-delay vs.\ zero-delay}
\label{sec-unitdelay}

\begin{figure}
% parses ok 
\begin{verbatim}
module DelayedAnd
  interface x1, x2, y : bool

  atom randx
    controls x1, x2
  init
    [] true -> x1' := nondet; x2' := nondet
  update 
    [] true -> x1' := nondet; x2' := nondet
  endatom

  atom delayedAnd
    controls y
    reads x1, x2
  init
    [] true -> y' := nondet
  update 
    [] true -> y' := x1 & x2
  endatom
endmodule 
\end{verbatim}
\caption{Module {\tt DelayedAnd}}
\label{fig-delayedand}
\end{figure}

The module {\tt DelayedAnd} of Figure~\ref{fig-delayedand} has three
interface variables {\tt x1}, {\tt x2}, and {\tt y}, of type \BOOL,
which is the \mocha\ keyword for boolean.  Atom {\tt randx} updates
{\tt x1} and {\tt x2} non-deterministically at each round: the keyword
\NONDET\index{nondet@\NONDET} is used to assign to a variable a
non-deterministic element of the variable domain (in this case, the
domain of {\tt x1} and {\tt x2} is $\{\TRUE, \FALSE\}$).  This
non-deterministic update simulates random inputs to our {\sc and} gate.
Atom {\tt delayedAnd} assigns to the next value of {\tt y} the logical
{\sc and} of {\tt x1} and {\tt x2}.  Abbreviating \TRUE\ and \FALSE\
with {\sc t} and {\sc f}, we can represent one of the possible
behaviors of module {\tt DelayedAnd} as follows: 

\begin{center}
\begin{tabular}{r|rrrrrrrrrrrc}
Round: 
 & 0 (initial) & 1 & 2 & 3 & 5 & 6 & 7 & 8 & 9 & 10 & $\cdots$ \\ \hline
{\tt x1}:  & {\sc t} & {\sc t} & {\sc f} & {\sc t} & {\sc t} & {\sc f} & {\sc t} & {\sc f} & {\sc t} & {\sc t} & $\cdots$ \\
{\tt x2}:  & {\sc t} & {\sc f} & {\sc t} & {\sc t} & {\sc t} & {\sc f} & {\sc f} & {\sc t} & {\sc t} & {\sc f} & $\cdots$ \\
{\tt y}:   & {\sc f} & {\sc t} & {\sc f} & {\sc f} & {\sc t} & {\sc t} & {\sc f} & {\sc f} & {\sc f} & {\sc t} & $\cdots$ \\ \hline
\end{tabular}
\end{center}

\noindent
Under index $i$, for $i > 0$, we list the current values of the
variables at round $i$. Thus, at round~6, the current value of {\tt
x1} and {\tt x2} is~{\sc f}, and the current value of {\tt y} is~{\sc
f}.  As you can see from this example, variable {\tt y} contains the
logical {\sc and} of {\tt x1} and {\tt x2} with {\em one round of
delay:\/} in fact, according to module {\tt DelayedAnd} the {\em
next\/} value of {\tt y} is the {\sc and} of the {\em current\/}
values of {\tt x1} and {\tt x2}.  To eliminate this delay, you can
await the next values of {\tt x1} and {\tt x2}, and use their next
(rather than the current) values to compute the next value of {\tt y}.
This is what module {\tt SynchAnd} of Figure~\ref{fig-synchand} does.
One of the many behaviors of module module {\tt SynchAnd} is as
follows:  

\begin{center}
\begin{tabular}{r|rrrrrrrrrrrc}
Round: 
 & 0 (initial) & 1 & 2 & 3 & 5 & 6 & 7 & 8 & 9 & 10 & $\cdots$ \\ \hline
{\tt x1}:  & {\sc t} & {\sc t} & {\sc f} & {\sc t} & {\sc t} & {\sc f} & {\sc t} & {\sc f} & {\sc t} & {\sc t} & $\cdots$ \\
{\tt x2}:  & {\sc t} & {\sc f} & {\sc t} & {\sc t} & {\sc t} & {\sc f} & {\sc f} & {\sc t} & {\sc t} & {\sc f} & $\cdots$ \\
{\tt y}:   & {\sc t} & {\sc f} & {\sc f} & {\sc t} & {\sc t} & {\sc f} & {\sc f} & {\sc f} & {\sc t} & {\sc f} & $\cdots$ \\ \hline
\end{tabular}
\end{center}

\noindent 
The syntax of boolean expressions is presented in Table~\ref{table-booleanexpr-syntax}.
An admissible identifier for a boolean variable is an \ident\ with an optional \nextop\
followed by the optional expression \arefer.
The presence of an \arefer\ expression indicates the dereferencing of an array structure to a particular element.
The exact meaning will be described in Section~\ref{sec-arrays}.
The syntax of \intrangeexpr\ and \natexpr\ expressions will be explained in Section~\ref{sec-nat-int}.
The construct \eventvar\QUESTIONMARK\ will be explained in Section~\ref{sec-eventvars}.
The equivalence test between enumeration types will be explained in Section~\ref{sec-enum}.

The meaning of the boolean operators, together with their precedence,
are listed in Table~\ref{table-boolop-precedence}.
Finally, the syntax of the commands is given in Table~\ref{table-command-syntax}.
The \FORALL\ construct as well as the syntax of \aassign\ expressions will be described in Section~\ref{sec-arrays}.
The syntax of \bitexpr\ expressions will be described in Section~\ref{sec-bitvector}.
The \eventvar\EXCLAMATION\ construct will be described in Section~\ref{sec-eventvars}.

\begin{figure}
% parses ok 
\begin{verbatim}
module SynchAnd
  interface x1, x2, y : bool

  atom randx
    controls x1, x2
  init
    [] true -> x1' := nondet; x2' := nondet
  update 
    [] true -> x1' := nondet; x2' := nondet
  endatom

  atom delayedAnd
    controls y
    awaits x1, x2
  init
  update 
    [] true -> y' := x1' & x2'
  endatom
endmodule 
\end{verbatim}
\caption{Module {\tt SynchAnd}}
\label{fig-synchand} 
\end{figure}

\begin{table}\boxed{\begin{tabbing}
\boolexpr\ \gramdef\ \= \booleavar\ $|$ \eventvar \QUESTIONMARK\ $|$ \LPARA \boolexpr \RPARA\ $|$ \NOT \boolexpr \\[1ex]
                     \> $|$ $\boolexpr_1$ \binbool\ $\boolexpr_2$ \\[1ex]
                     \> $|$ \IF\ \boolexpr\ \= \THEN\ $\boolexpr_1$ \\[1ex]
                     \>\> \ELSE\ $\boolexpr_2$ \\[1ex]
                     \> $|$ \compare\ $|$ \TRUE\ $|$ \FALSE \\[3ex]
\booleavar\ \gramdef\ \ident[\nextop][\arefer] \\[3ex]
\eventvar\ \gramdef\ \ident \\[3ex]
\numexpr\ \gramdef\ \intrangeexpr\ $|$ \natexpr \\[3ex]
\compare \gramdef\ \= $\numexpr_1$ \compareOp\ $\numexpr_2$ \\[1ex]
                   \> $|$ ($\element_1$ $|$ $\enumvar_1$) \EQ\ ($\element_2$ $|$ $\enumvar_2$) \\[3ex]
\enumvar\ \gramdef\ \ident[\nextop][\arefer] \\[3ex]
\binbool\ \gramdef\ \AND\ $|$ \OR\ $|$ \EQUIV\ $|$ \IMPLY \\[3ex]
\compareOp\ \gramdef\ \LT\ $|$ \LE\ $|$ \EQ\ $|$ \GE\ $|$ \GT
\end{tabbing}}
\caption{Boolean expression syntax\index{expression!boolean}\index{operator!comparison}}  
\label{table-booleanexpr-syntax}
\end{table}

\begin{table}\boxed{\begin{tabbing}
Operator: \hspace*{4em} \= Meaning: \\[1ex]
\NOT \> negation \\
\AND \> conjunction \\
\OR  \> disjunction \\
\IMPLY \> implication \\
\EQUIV \> equivalence \\[3ex]
Precedence: \\[1ex]
\NOT \> Highest \\
\AND, \OR \> Medium \\
\IMPLY, \EQUIV \> Lowest 
\end{tabbing}}
\caption{Meaning and precedence of boolean operators}\index{operator!boolean!precedence}
\label{table-boolop-precedence}
\end{table}

\begin{table}\boxed{\begin{tabbing}
\command\ \gramdef\ \= \FORALL\ $\vari$ $\var$\nextop\LSQBRAC$\vari$\RSQBRAC\ \ASSIGN\ \expr\\[1ex]
                    \> $|$ $\var$\nextop[\aassign] \ASSIGN\ \expr\ $|$ \NONDET \\[1ex]
                    \> $|$ \eventvar \EXCLAMATION\\[3ex]
\expr\ \gramdef\ \boolexpr\ $|$ \numexpr\ $|$ \bitexpr \\[3ex]
$\var$ \gramdef\ \ident\ \qquad $\vari$ \gramdef\ \ident
\end{tabbing}}
\caption{Command syntax\index{command}\index{assignment}}
\label{table-command-syntax}
\end{table}


\subsubsection{Rounds and sub-rounds}

To understand the semantics of module {\tt SynchAnd} of
Figure~\ref{fig-synchand}, you can imagine each round as consisting of
one or more {\em sub-rounds}.\index{round!sub-} During each subround,
one of the atoms updates the variables it controls.  The order in
which the atoms execute their subrounds is arbitrary, except that if
an atom \atom\ awaits some variable that is controlled by another atom
\aatom, then the sub-round of atom \atom\ must precede that of atom
\aatom.  The order of execution of the sub-rounds does not change the
set of possible successor states, provided the above constraint is
respected (can you see why?).


\subsubsection{Omitting the \INIT\
keyword}\index{round!initial}\index{init@\INIT}

Atom delayedAnd of Figure~\ref{fig-delayedand} shows that you can omit
the guarded commands following the \INIT\ keyword: in this case, the
initial round is equivalent to the update round, so that atom {\tt
delayedAnd} can be equivalently written as follows: 

\begin{verbatim}
  atom delayedAnd
    controls y
    awaits x1, x2
  init
    [] true -> y' := x1' & x2'
  update 
    [] true -> y' := x1' & x2'
  endatom
\end{verbatim}

You can also omit the \INIT\ keyword entirely.  In this case, the
controlled variables of the atom are initialized non-deterministically,
as if they were idle.  Table~\ref{table-initialconv} summarizes these
conventions. 

\begin{table}\boxed{
\begin{itemize}

\item {\em The keyword \INIT\ is missing.}  Then, the controlled
variables of the atom are initialized non-deterministically. 

\item {\em The keyword \INIT\ is present, but no initial guarded
command is present.}  Then, the initial round is assumed to coincide
with the update round. 

\end{itemize}}
\caption{Rules for omitting the \INIT\ keyword, or the initial guarded
commands}
\label{table-initialconv}
\end{table}


\subsection{Guards}

Using guards, you can constrain the random walk of variable {\tt x} of
module {\tt randomwalk} (Figure~\ref{fig-randomwalk}) to the interval
$[0, 10]$ as shown in Figure~\ref{fig-randomwalk010}. 

\begin{figure}
% parses ok 
\begin{verbatim}
module randomwalk010
  interface x : (0..10)

  atom incrdecr010
    controls x
    reads x 
  init
    [] true -> x' := 0
  update 
    [] x < 10  -> x' := x + 1
    [] x > 0 -> x' := x - 1
  endatom
endmodule 
\end{verbatim}
\caption{Module {\tt randomwalk010}}
\label{fig-randomwalk010}
\end{figure}

The variable {\tt x} has a type that ranges over the integers from
~0 to~10, included (type declarations will be discussed in
Section{\ref{sec-types}).  The guards are used to specify when the
corresponding command can be executed.  In the above example, when
${\tt x} = 0$, only the first command 
%
\begin{quote}
{\tt [] x < 10 -> x' := x + 1}
\end{quote}
%
can be executed, so that {\tt x} can be incremented but not
decremented.  Similarly, for ${\tt x} = 10$ only the second guarded
command can be executed.  For $0 < {\tt x} < 10$, both guarded
commands can be executed, and {\tt x} can be both incremented and
decremented. 

Guards can also depend on the {\em next\/} value of a variable.  The
module {\tt CountUp} of Figure~\ref{fig-countup} contains two atoms:
an atom {\tt toggle}, whose boolean output {\tt x} changes at
arbitrary points in time, and the atom {\tt counter}, which counts
with its output {\tt count} the number of ``positive fronts'' (changes
from \FALSE\ to
\TRUE) of {\tt x}. 

\begin{figure}
% parses ok 
\begin{verbatim}
module CountUp
  interface x : bool; count : int

  atom toggle 
    controls x
    reads x
  update 
    [] true -> x' := ~x 
    [] true -> x' := x 
  endatom

  atom counter
    controls count
    reads x, count
    awaits x
  init
    [] true -> count' := 0
  update
    [] ~(x <=> x') -> count' := count + 1
    [] x <=> x' -> count' := count
  endatom
endmodule 
\end{verbatim}
\caption{Module {\tt CountUp}}
\label{fig-countup}
\end{figure}

\subsubsection{Omitting guards}\index{guard!omitting}

\begin{table}\boxed{
A variable $x$ that is idle in the initial round is updated
to a non-deterministic value in its domain.  A variable $x$ that is
idle in an update round is updated as follows:

\bigskip

\begin{enumerate}

\item if $x$ is both controlled and read, then the value of $x$ is
left unchanged;

\item \label{enum-sleep2} 
if $x$ is controlled but not read, then the value of $x$ at the
next round is chosen non-deterministically from the domain of $x$. 

\end{enumerate}}
\caption{Updating idle variables}\index{variable!idle}
\label{table-sleeprules}
\end{table}

If no guard is true, then the atom {\em idles\/}\index{atom!idle} for
one round: this means that all the variables controlled by the atom
are idle,\index{variable!idle} and their value is updated according
to Table~\ref{table-sleeprules}.  The reason for the seemingly strange
Rule~\ref{enum-sleep2} is that if a variable is not read, its current
value is not available to the atom.  Therefore, the atom is unable to
insure that the updated value for the variable is equal to its current
value.  To help prevent errors, \mocha\ issues a warning whenever a
variable is controlled but not read.  Using this feature, you can
rewrite the previous module {\tt CountUp} as shown in
Figure~\ref{fig-countupprime}. 
Note that atom {\tt counterPrime} is idle whenever variable {\tt x}
does not change its value. 

\begin{figure}
% parses ok 
\begin{verbatim}
module CountUpPrime
  interface x : bool; count : int

  atom toggle 
    controls x
    reads x
  update 
    [] true -> x' := ~x 
    [] true -> x' := x 
  endatom

  atom counterPrime
    controls count
    reads x, count
    awaits x
  init
    [] true -> count' := 0
  update
    [] ~(x <=> x') -> count' := count + 1
  endatom
endmodule 
\end{verbatim}
\caption{Module {\tt CountUpPrime}}
\label{fig-countupprime}
\end{figure}



\subsubsection{The \DEFAULT\ guard}\index{guard!default@\protect\DEFAULT}
\label{sec-default}

If you want a guarded command to be executed when the guards of all
other commands are false, you can use the keyword \DEFAULT\ as
guard. For example, suppose that you want to count the number of
consecutive times (including the current one) in which the random walk
of module {\sc randomwalk010} (Figure~\ref{fig-randomwalk010}) is
greater than~5.  You can do this as shown in
Figure~\ref{fig-randomwalkabove}. 

\begin{figure}
% parses ok 
\begin{verbatim}
module randomwalkabove
  interface x : (0..10); count : int

  atom incrdecr010
    controls x
    reads x 
  init
    [] true -> x' := 0
  update 
    [] x < 10  -> x' := x + 1
    [] x > 0 -> x' := x - 1
  endatom

  atom countconsec 
    controls count
    reads count
    awaits x
  init
    [] true -> count' := 0
  update 
    [] x' > 5  -> count' := count + 1
    [] default -> count' := 0
  endatom
endmodule 
\end{verbatim}
\caption{Module {\tt randomwalkabove}}
\label{fig-randomwalkabove}
\end{figure}


\subsubsection{Omitting variable updates}\index{variable!omitting}

If an atom controls more than one variable, it is possible to omit
some of the variable updates from the guarded commands.  If the
guarded command is selected, the variables whose updates are omitted
are idle,\index{variable!idle} and their value is updated according to
Table~\ref{table-sleeprules}.  This convention often enables a
considerable space saving.  Module {\tt GrayCode} of
Figure~\ref{fig-modulepc} generates outputs {\tt x} and {\tt y}, that
change cyclically following the Gray code sequence $00$, $01$, $11$,
$10$.  The module uses a variable {\tt pc} that is {\em
private\/}\index{variable!private} to the module: it is visible only
from within the module itself.

\begin{figure}
% parses ok 
\begin{verbatim}
module GrayCode
  interface x, y : (0..1)
  private pc : (0..3)

  atom gray
    controls x, y, pc
    reads    x, y, pc
  init
    [] true -> x' := 0; y' := 0; pc' := 0
  update
    [] pc = 0 -> pc' := 1; x' := 1
    [] pc = 1 -> pc' := 2; y' := 1
    [] pc = 2 -> pc' := 3; x' := 0
    [] pc = 3 -> pc' := 0; y' := 0
  endatom
endmodule
\end{verbatim}
\caption{Module {\tt GrayCode}}
\label{fig-modulepc}
\end{figure}



\subsubsection{Lazy atoms}\index{atom!lazy}\index{lazy@\LAZY}
\label{sec-lazyatoms}
A {\em lazy atom\/} is an atom whose controlled variables can remain
unchanged during any update round.  Lazy atoms can be declared with
the help of the keyword \LAZY\ (see Table~\ref{table-atom-syntax}).
Precisely, using the keywork \LAZY\ is equivalent to adding the
update guarded command with empty body: 
%
\begin{verbatim}
[] true -> 
\end{verbatim}
%
to the atom.  When the \LAZY\ keyword is used, the atom must
read\index{variable!read} all the variables it controls.  Using
keyword \LAZY, you can encode module {\tt CountUpPrime}
(Fig~\ref{fig-countupprime}) in the alternative way of
Figure~\ref{fig-countupsecond}. 
Since atom {\tt toggleSecond} can sleep at any time, the guarded
command 
%
\verb#[] true -> x' := x#
%
is no longer necessary.  However, to insure that variable {\tt x}
retains its value when the atom sleeps, it is now necessary to declare
it as read (see Table~\ref{table-sleeprules}).

\begin{figure}
% parses ok 
\begin{verbatim}
module CountUpSecond 
  interface x : bool; count : int

  lazy atom toggleSecond
    controls x
    reads x
  update 
    [] true -> x' := ~x 
  endatom

  atom counterPrime
    controls count
    reads x, count
    awaits x
  init
    [] true -> count' := 0
  update
    [] ~(x <=> x') -> count' := count + 1
  endatom
endmodule 
\end{verbatim}
\caption{Module {\tt CountUpSecond}}
\label{fig-countupsecond}
\end{figure}


\begin{comment}
\subsubsection{Event atoms}\index{atom!event}\index{event@\EVENT}

\begin{figure}
% parses ok 
\begin{verbatim}
module ExampleEvent
  interface y, z : int
  external x : bool

  event atom sometimesUp
    controls y, z
    reads x, y, z
    awaits x
  update 
    [] ~(x = x') -> y' := y + 1 
    [] true      -> z' := z - 1 
  endatom 
endmodule
\end{verbatim}
\caption{Module {\tt ExampleEvent}}
\label{fig-exampleevent}
\end{figure}

An {\em event atom\/} is an atom whose controlled variables can remain
unchanged at any update round in which the values of the variables
that are both read and awaited by the atom do not change.  Event atoms
are declared using the keyword \EVENT\ (see
Table~\ref{table-atom-syntax}). If  $\var_1, \ldots, \var_m$ are
the variables that are both read and awaited by the module, then using
the keyword \EVENT\ is equivalent to adding the guarded command 
%
\begin{quote}
$\var_1 {\tt =} \var'_1 {\tt \&} \cdots {\tt \&} \var_m {\tt =} \var'_m
{\tt ->}$
\end{quote}
%
to the update commands of a module.  When the keywork \EVENT\ is used,
the atom must read\{index{variable!reading} all the variables it
controls.  The example of Figure{fig-exampleevent} illustrates the use
of the \EVENT\ keyword
\mynote{I cannot really think of any good use for it; can you?}
In the module {\tt ExampleEvent}, variable {\tt x} is declared
to be {em external.}\index{variable!external}.  This indicates that
the module can read (and await) the value of variable {\tt x}, but
cannot modify it.  Atom {\tt sometimesUp} behaves as follows: 
%
\begin{itemize}
\item if the value of {\tt x} is unchanged, then the atom can either
sleep, or it can decrement {\tt z}; 

\item if the value of {\tt x} is changed, then the atom can either
increment {\tt y}, or it can decrement {\tt z}. 
\end{itemize}
%
The above example illustrates a common pitfall when using the \EVENT\
keyword: while the keyword allows an atom to sleep when the variables
that are both read and awaited don't change, the keyword does not
force the atom to take any particular action when these variables do
change.
\end{comment}

\section{Modules}\index{module}

A {\em module\/} is a collection of atoms, together with a declaration
of the variables that occur in the module.  There are two types of
modules: {\em simple modules,}\index{module!simple} obtained by
specifying directly the atoms composing the module, and {\em composite
modules,}\index{module!composite} obtained by combining or modifying
existing modules.  

\subsection{Simple Modules}

\begin{table}
\boxed{
\begin{tabbing}
\simplemodule\ \gramdef\ \= \MODULE\ \modulename \\[1ex]
\> \ind $\interfacedecl_1$ \\
\> \ind $\cdots$ \\
\> \ind $\interfacedecl_m$ \\[1ex]
\> \ind $\atomid_1$ \\
\> \ind $\cdots$ \\
\> \ind $\atomid_n$ \\[1ex]
\> \ENDMODULE \\[3ex]
\interfacedecl\ \gramdef\ \= (\PRIVATE\ $|$ \INTERFACE\ $|$ \EXTERNAL) \\[1ex]
\> $\vardecl_1$\SEMICOLON\ \ldots \SEMICOLON$\vardecl_k$ \\[3ex]
\vardecl\ \gramdef\ $\var_1$\COMA\ \ldots \COMA$\var_i$ \COLON\ $\typeid$ \\[3ex]
\modulename\ \gramdef\ \ident
\end{tabbing}}
\caption{Simple module syntax}
\label{table-simple-module-syntax}
\end{table}

The syntax of simple modules is given in
Table~\ref{table-simple-module-syntax}.  After the keyword \MODULE, you must
provide the module name \modulename.  The module name is followed by a
list of variable declarations, which is followed in turn by the
list of atoms composing the module.  The keyword \ENDMODULE\ concludes
the module specification.

Associated with each module are three sets of variables: the {\em
private variables,} the {\em interface variables,} and the {\em
external variables.}  These sets of variables have the following
meaning: 
%
\begin{itemize}

\item The private variables\index{variable!private} are the
variables that are controlled by some atom of the module, and that
cannot be read or awaited by other modules.  The value of a private
variable is thus local to the module.

\item The interface variables \index{variable!interface} are the
variables that are controlled by some atom in the module, and that can
be read or awaited by atoms in other modules.  These variables cannot
however be controlled by atoms of other modules, according to the
general rule stating that a variable can be controlled by at most one
atom. 

\item The external variables \index{variable!external} are the
variables whose value can be read or awaited by the atoms in the
module.  These variables cannot be controlled by any atom in the
module.

\end{itemize}
%
Each of the private and interface variables of the module must be
controlled by some atom.  External variables, on the other hand, do
not need to be read nor awaited by any atom.  Again, the awaits
relation\index{awaits relation} $\awaits_g$ computed for the module
must be acyclical.

The three sets of private, interface and external variables, together
with the type of each variable, are specified using the syntax given
in Table~\ref{table-simple-module-syntax}.  The symbol \typeid\ in the table
indicates a {\em type declaration:\/}\index{type!declaration} two
possible declarations are {\tt int} (for integer type) and {\tt bool}
(for boolean type).  Type declarations will be covered in
Section~\ref{sec-types}.

You have seen examples of module declarations in the previous section
as well as the tutorial chapter;
additional examples will be presented in Section~\ref{sec:more_examples}. 



\subsection{Composite Modules}

There are three operations defined on modules in \mocha: {\em variable
hiding,}\index{variable!hiding}, {\em variable
renaming,}\index{variable!renaming} and {\em parallel
composition.}\index{parallel composition} 
These operations create new {\em composite\/} modules, which can in
turn be combined into other composite modules. 


\subsubsection{Hiding Variables}\index{variable!hiding}  

The hiding of interface variables allows you to construct module
abstractions of varying degrees of detail.  For instance, after
composing two modules, it may be appropriate to convert some interface
variables to private variables, so that they are used only for the
interaction of the component modules, and are no longer visible to the
environment of the compound module.

Given a module (simple or composite) with name $P$, denote with
$\external(P)$, $\private(P)$, $\interface(P)$ the set of external,
private, and interface variables of $P$, respectively.  Given any list
of interface variables $\var_1, \ldots, \var_n \in \interface(P)$, you 
can {\em hide\/} $\var_1, \ldots, \var_n$ using the construct:
%
\begin{quote}
  $Q$ \ASSIGN\ \HIDE\  $\var_1$\COMA\ \ldots \COMA$\var_n$ \IN\
      $P$ \ENDHIDE
\end{quote}
%
The resulting module $Q$ is identical to $P$, except that: 
%
\begin{eqnarray*}
  \private(Q) & = & \private(P) \cup \{ \var_1, \ldots, \var_n \} \\
  \interface(Q) & = & \interface(P) \setminus \{ \var_1, \ldots, \var_n \} . 
\end{eqnarray*}
%
In other words, the effect of hiding the interface variables $\var_1,
\ldots, \var_n$ in $P$ is to make them private, so that other modules
cannot access their content.


\subsubsection{Variable Renaming}\index{variable!renaming}

The renaming operation is useful for creating different instances of a 
module, and for avoiding name conflicts. 
Let $P$ be the name of a module, and let $\var$ be a variable
(external, private, or interface) of a module. You can rename variable
$\var$ to $\avar$ with the construct: 

\begin{quote}
$Q$ \ASSIGN\ $P$ \LSQBRAC\ $\var$ \ASSIGN\ $\avar$ \RSQBRAC 
\end{quote}

The resulting module $Q$ is identical to $P$, except that variable
$\var$ has been renamed to $\avar$.  The new variable $\avar$ is of
the same type of $\var$, and it does not need to be explicitly
declared: \mocha\ infers its type and class (external, private, or
interface) from the renaming operation.
You can also rename several variables at once: the construct

\begin{quote}
$Q$ \ASSIGN\ $P$ \LSQBRAC\ $\var_1$\COMA\ \ldots \COMA$\var_m$ \ASSIGN\ $\avar_1$\COMA\ \ldots \COMA$\avar_m$ \RSQBRAC
\end{quote}
%
{\em simultaneously\/} renames the variables $\var_1, \ldots, \var_m$
to $\avar_1, \ldots, \avar_m$.  This simultaneous renaming is
especially useful to exchange variables. Note that the variables $\avar_1,
\ldots, \avar_m$ must be all distinct --- even if they are external
variables of the module of the same type, in which case renaming them
to the same name would make sense (it would correspond to drawing the
various inputs from the same source).
For instance, if a module $P$
has two variables $x_1$ and $x_2$ and we want to exchange them, the following 
\begin{quote}
$Q$ \ASSIGN\ $P$ \LSQBRAC\ $\var_1$ \ASSIGN\ $\var_2$ \RSQBRAC
\LSQBRAC\ $\var_2$ \ASSIGN\ $\var_1$ \RSQBRAC
\end{quote}
leaves $Q$ and $P$ being identical except that in $Q$ the variable $\var_2$ is mapped to $\var_1$.
The correct way to exchange them is 
\begin{quote}
$Q$ \ASSIGN\ $P$ \LSQBRAC\ $\var_1$, $\var_2$ \ASSIGN\ $\var_2$, $\var_1$ \RSQBRAC
\end{quote}
%, as will be illustrated in Section~\ref{sec-ex-moduleops}.  

\mocha\ propagates the variable renames to the awaits
relation\index{awaits relation} $\awaits$ in the obvious way.

\subsubsection{Parallel composition}\index{parallel composition}

You can use parallel composition to combine two modules into a single
module whose behavior captures the interaction between the two
component modules.  Two modules with names $P$ and $Q$ are {\em
compatible\/}\index{module!compatible} if the following conditions
hold: 
%
\begin{itemize}
\item the sets of interface variables of $P$ and $Q$ are
disjoint, i.e.\ $\interface(P) \cap \interface(Q) = \emptyset$;

\item the global awaits relation\index{awaits relation} $\awaits_g$
for the two modules is acyclic. 

\end{itemize}
%
If $P$ and $Q$ are compatible, you can form their parallel composition
(and give it name $R$) with the construct:
%
\begin{quote}
$R$ \ASSIGN\ $P$ \PAR\ $Q$
\end{quote}
%
The external, private, and interface variables of $R$ are given by: 
%
\begin{eqnarray*}
  \private(R) & = & \private(P) \cup \private(Q) \\
  \interface(R) & = & \interface(P) \cup \interface(Q) \\
  \external(R) & = & [\external(P) \cup \external(Q)] \setminus \interface(R)
\end{eqnarray*}
%
Note that $\external(R)$ is equivalent to
$[\external(P) \setminus \interface(Q)] \cup [\external(Q) \setminus \interface(P)]$
since the sets of interface variables of $P$ and $Q$ are disjoint.

\subsubsection{Summary of module declarations}\index{module!declaration}

Table~\ref{table-composite-module-syntax} summarizes the possible module declarations.
The non-terminal \program\ is the start symbol of the grammar for \rem.
The expression \typedefinition\ will be described in Section~\ref{sec-types}.
Note that while the name\index{atom!name} of an
atom is used only for mnemonic purposes, the name\index{module!name}
of a module is used by \mocha\ to construct more complex modules.  In
fact, the names of modules will also be mentioned by verification
commands, as you will see later. 

\begin{table}
\boxed{
\begin{tabbing}
\program\ \gramdef\ \moduledecl\ $|$ \typedefinition\ $|$ (\moduledecl\ $|$ \typedefinition) \program \\[3ex]
\moduledecl\ \gramdef\ \simplemodule\ $|$ \modulename\ \ASSIGN\ \compositemodule \\[3ex]
\compositemodule\ \gramdef\ \= \modulename \\[1ex]
  \> $|$ \LPARA \compositemodule \RPARA \\[1ex]
  \> $|$ \HIDE\ $\var_1$\COMA\ \ldots \COMA$\var_n$ \IN\
         \compositemodule\ \ENDHIDE \\[1ex]
  \> $|$ \compositemodule\ \LSQBRAC\ $\var_1$\COMA\ \ldots \COMA$\var_m$ \ASSIGN\ $\avar_1$\COMA\ \ldots \COMA$\avar_m$ \RSQBRAC \\[1ex]
  \> $|$ $\compositemodule_1$ \PAR\ $\compositemodule_2$
\end{tabbing}}
\caption{Syntax of module declarations}
\label{table-composite-module-syntax}
\end{table}

As an example, we give two different approaches to constructing an
{\sc or} gate.  The first approach, sometimes called the {\em
behavioral\/} approach, specifies the behavior of the {\sc or} gate
directly: 

% parses ok 
\begin{verbatim}
module BehavOr
  external in1, in2: bool
  interface out: bool

  atom controls out awaits in1, in2
  update
      [] true -> out' := in1' | in2'
  endatom
endmodule
\end{verbatim}

\noindent
In the second approach, shown in Figure~\ref{fig-structuralor}, we
assume that all we have at our disposal is two-input {\sc nand} gates,
and we create an {\sc or} gate by connecting these {\sc nand} gates in
the appropriate way.  This approach is sometimes called the {\em
structural\/} approach, because it describes a system by specifying
its structure, rather than directly its behavior. Note that we cannot
obtain a {\sc not} gate simply by renaming the two inputs {\tt in1}
and {\tt in2} of the {\sc nand} gate to the same name {\sc norin}: we
must introduce a splitter module {\sc split}, which makes two signals
out of the same signal.

\begin{figure}
% parses ok 
\begin{verbatim}
module nand 
  external in1, in2 : bool
  interface out : bool

  atom controls out awaits in1, in2
  init update
    [] true -> out' := ~ (in1' & in2')
  endatom
endmodule

module split
  external in : bool
  interface out1, out2 : bool

  atom controls out1, out2 awaits in 
  init update
    [] true -> out1' := in'; out2' := in' 
  endatom 
endmodule

StructOr := hide in1a, in1b, in2a, in2b, cin1, cin2 in 
     split [in, out1, out2 := in1, in1a, in1b]
  || split [in, out1, out2 := in2, in2a, in2b]
  || nand  [in1, in2, out  := in1a, in1b, cin1] 
  || nand  [in1, in2, out  := in2a, in2b, cin2] 
  || nand  [in1, in2, out  := cin1, cin2, out]
  endhide
\end{verbatim}
\caption{Modules for structural {\sc or}}
\label{fig-structuralor}
\end{figure}

%\mynote{someone with the source code please put here a correct version
%of the three-bit binary counter}



\section{Types and Expressions}
\label{sec-types}


The simplest types of \mocha\ are boolean, integer and natural.  Other
types include events, ranges, enumerations, arrays, and bitfields.
Boolean variables are declared with the help of the keyword \BOOL, and
the syntax and semantics of boolean expressions has already been
described in Tables \ref{table-booleanexpr-syntax}
and~\ref{table-boolop-precedence}.


\subsection{Naturals and integers}
\label{sec-nat-int}

\begin{table}\boxed{\begin{tabbing}
\intrangeexpr\ \gramdef\ \= \intconst\ $|$ \somevar\ \\[1ex]
  \> $|$ \LPARA \intrangeexpr \RPARA\ $|$ \MINUS \intrangeexpr\ \\[1ex]
  \> $|$ $\intrangeexpr_1$ \PLUS\ $\intrangeexpr_2$ \\[1ex]
  \> $|$ $\intrangeexpr_1$ \MINUS\ $\intrangeexpr_2$ \\[1ex]
  \> $|$ \IF\ \boolexpr\ \= \THEN\ $\intrangeexpr_1$ \\[1ex]
  \>\> \ELSE\ $\intrangeexpr_2$ \\[3ex]
\intconst\ \gramdef\ \const \\[3ex]
\somevar\ \gramdef\ \ident[\nextop][\arefer]
\end{tabbing}}
\caption{Syntax of integer and range expressions}\index{expression!integer}\index{expression!range}
\label{table-intrange-expressions}
\end{table}

\begin{table}\boxed{\begin{tabbing}
\natexpr\ \gramdef\ \= \intconst\ $|$ \somevar\ \\[1ex]
  \> $|$ \LPARA \natexpr \RPARA\ $|$ $\natexpr_1$ \PLUS\ $\natexpr_2$ \\[1ex]
  \> $|$ \IF\ \boolexpr\ \= \THEN\ $\natexpr_1$ \\[1ex]
  \>\> \ELSE\ $\natexpr_2$
\end{tabbing}}
\caption{Syntax of natural expressions}\index{expression!natural}
\label{table-nat-expressions}
\end{table}

Integer variables are declared using the keyword
\INT;\index{type!integer}\index{int@\INT} their domain ranges over the
integers.  Natural variables are declared using the keyword
\NAT;\index{type!natural}\index{nat@\NAT} their domain ranges over the
non-negative integers.  The syntax of integer (and range) expressions
is given in Table~\ref{table-intrange-expressions}; the syntax of
natural expressions is given in Table~\ref{table-nat-expressions}.
A \const\ is a numerical string.
Natural variables can be assigned to integer ones, and they can be
used in integer expressions; the converse is not permitted. 


\subsection{Events}
\label{sec-eventvars}

In \mocha, events are represented by toggling the value of boolean
variables.  A variable that is used to represent events can be
using the keyword \EVENT\index{type!event}\index{event@event}.
This has two advantages.
First, it makes it easier to toggle the variable, and detect the
toggling.  Second, and more importantly, it notifies \mocha\ that the
value of the variable is irrelevant: the only thing that matters is
whether the value has been toggled or not.  Therefore, the presence of
an additional event variable does not increase the size of the state
space.  This will be explained more in detail in
Section~\ref{sec-efficiency}. 

If {\tt x} is an event variable, then we can toggle it with the
command {\tt x!},\index{!@{\tt !}} and we can test whether it has been
toggled with the boolean expression {\tt x?}\index{!@{\tt !}} (see
Table~\ref{table-booleanexpr-syntax}).  Aside from these two
operations, there are no expressions of event type, and events cannot
be compared using the operators \LT, \GT, etc.  The simple module {\tt
EventCount} of Figure~\ref{fig-eventcount} consists of two atoms: an
atom {\tt generate} that generates events, and an atom {\tt counts}
that counts them. Note that the atoms that refer to event variables
must also read them, to be able to distinguish when they are toggled.

\begin{figure}
% parses ok 
\begin{verbatim}
module EventCount
  external e : event
  external c : nat

  lazy atom generate
    controls e
    reads e
  update
    [] true -> e!
  endatom

  atom counts
    controls c
    reads c, e
    awaits e
  init
    [] true -> c' := 0
  update
    [] e? -> c' := c + 1
  endatom
endmodule
\end{verbatim}
\caption{Module {\tt EventCount}}
\label{fig-eventcount}
\end{figure}


\subsection{Range types}\index{type!range}

A range declaration in \mocha\ has the form {\tt (0..\maxrange)},
where \maxrange\ is a non-negative integer constant.  The expressions
using range types share the same syntax of integer expressions, given
in Table~\ref{table-intrange-expressions}.  In these expressions, you
must follow the restrictions:
%
\begin{enumerate}

\item You cannot mix different range types (or range types and \INT\
or \NAT) in the same expression.

\item You cannot assign an expression of a range type to a different
range type, or to an \INT\ or \NAT.

\end{enumerate}
%
Arithmetic operations are performed modulo \maxrange+1 on the range
type, thus insuring that their value belongs to the same range type of
the operands.  There is no type-casting operator in \mocha, making the
translation between different range types cumbersome.  A module that
translates a variable of type {\tt (0..3)} to integers is given in
Figure~\ref{fig-rangetranslation}: you see that you don't want to be
doing this all the time.

\begin{figure}
% parses ok 
\begin{verbatim}
module RangeTranslation
  external x : (0..3)
  interface y : int

  atom translate
    controls y
    awaits x
  init update
    [] x' = 0 -> y' := 0
    [] x' = 1 -> y' := 1
    [] x' = 2 -> y' := 2
    [] x' = 3 -> y' := 3
  endatom
endmodule
\end{verbatim}
\caption{Module {\tt RangeTranslation}}
\label{fig-rangetranslation}
\end{figure}

\subsection{Enumeration type}\index{type!enumeration}
\label{sec-enum}

To declare a variable of enumeration type, you simply list between
curly braces the possible values of the variable, separated by commas.
The only operations you can perform on an enumeration type are
checking for equality (using operator \EQ), and assignments.  The
minimal module {\tt UpDownWalk} of Figure~\ref{fig-updownwalk}
contains two atoms: one that generates an infinite sequence of ups and
downs, and another that counts up or down, as dictated by the former
atom.

\begin{figure}
% parses ok 
\begin{verbatim}
module UpDownWalk
  private ud : { up, down }
  interface c : int

  atom generate
    controls ud
  update
    [] true -> ud' := nondet
  endatom

  atom count
    controls c
    reads c
    awaits ud
  init
    [] true -> c' := 0
  update
    [] ud' = up   -> c' := c + 1
    [] ud' = down -> c' := c - 1
  endatom
endmodule
\end{verbatim}
\caption{Module {\tt UpDownWalk}}
\label{fig-updownwalk}
\end{figure}

A value in an enumeration cannot be used in more than one distinct
range type.  For example, in Figure~\ref{fig-recyclenames} it was
necessary to capitalize the values {\tt Water} and {\tt Pasta} for the
range of {\tt z} in order to avoid using twice the values for the
range of {\tt x}.  On the other hand, note that it is possible to use
the same enumeration value in {\em the same\/} range type multiple
times, as illustrated in Figure~\ref{fig-enumassign}. 

\begin{figure}
% parses ok 
\begin{verbatim}
module RecycleNames
  external x : { water, bread, pasta } 
  interface z : { Water, Pasta, Bananas }

  atom copy 
    controls z
    awaits x
  init update 
    [] x' = water -> z' := Water
    [] x' = pasta -> z' := Pasta
  endatom
endmodule 
\end{verbatim}
\caption{Module {\tt RecycleNames}}
\label{fig-recyclenames}
\end{figure}

\begin{figure}
% parses ok 
\begin{verbatim}
module EnumAssign
  external x : { water, bread, pasta } 
  interface y : { water, bread, pasta }

  atom copy 
    controls y
    awaits x
  init update 
    [] true -> y' := x'
  endatom
endmodule 
\end{verbatim}
\caption{Module {\tt EnumAssign}}
\label{fig-enumassign}
\end{figure}

\subsection{Naming types}\index{type!equality}\index{type!name}

Equality of types in \mocha\ is defined structurally (as opposed to
being based on inheritance).  This means that two variables have the
same type iff they are declared of structurally equivalent types, even
though the type declarations may be distinct.  For example, the module
{\tt EnumAssign} of Figure~\ref{fig-enumassign} is correct: {\tt y'}
can be assigned {\tt x'}.  You can
also associate a name with enumeration and range
types\index{type!name} (but not with integer, natural, boolean, and
event types).  To do so, you include the declaration
%
\begin{quote}
\TYPEDEF\ \typename\ \COLON\ \complextype\index{type@\TYPEDEF}
\end{quote}
%
in the input file, before the modules that use type \typename.  The
type \complextype\ must be an enumeration, range, array, or bitvector
type.  Using this declaration makes it easier to write structurally
equal types, and also makes the code more readable.  Using this type
declaration, you can rewrite the example of
Figure~\ref{fig-enumassign} as shown in
Figure~\ref{fig-enumassignbis}. 

\begin{figure}
% parses ok 
\begin{verbatim}
type menu : { water, bread, pasta } 

module EnumAssignBis
  external x : menu
  interface y : menu

  atom copy 
    controls y
    awaits x
  init update 
    [] true -> y' := x'
  endatom
endmodule 
\end{verbatim}
\caption{Module {\tt EnumAssignBis}}
\label{fig-enumassignbis}
\end{figure}



\subsection{Arrays}\index{type!array}\index{array}
\label{sec-arrays}

\mocha\ also has array and
bitvector\index{type!bitvector} types.  Arrays are
essentially as in ordinary programming languages, except that
multidimensional arrays are not provided, and other minor limitations
are present.  Each array variable has to be controlled by the same
module, but different array elements can be controlled by different
atoms: hence, you can think of arrays as a bus of wires or signals,
not necessarily controlled jointly. 
The syntax for declaring an array type is: 

\begin{quote} 
  \ARRAY\ \indextype\ \OF\ \elementtype
\end{quote}

Type \indextype\ is the type of the array index, and can be
an enumeration type or a range type (perhaps the most common case).
Type \elementtype\ is the type of the array elements, and can be a
boolean, integer, natural, enumeration, range, or bitvector type.
An example of an array declaration is 

\begin{quote} \tt
  interface x : array (0..10) of bool
\end{quote}

Table~\ref{table-array-access-syntax} describes the constructs
to express dereferencing of arrays in the context of variable declarations in atoms
as well as in expressions and assignments.

You can declare that an atom controls the whole array {\tt x} by
writing {\tt controls x}.  If the atom controls only some elements of
{\tt x}, you can specify these simply by listing them, as in {\tt
controls x[0], x[2], x[4]}.  You can specify which elements are read
or awaited in a similar way. 
Note that the array elements that are controlled, read, or awaited
must be specified using constants or elements of enumeration types only: you cannot write {\tt controls
x[1 + 2]} or {\tt awaits x[a]} if {\tt a} does not denote an element of an enumeration type.
If an array is controlled by the module (i.e.\ if it is declared as
{\tt interface} or {\tt private} in the module), then all array
elements must be controlled by some atom of the module.

You can refer to array elements by writing the variable name followed
by a range expression in square brackets, as in {\tt x[4]} or {\tt x[z
+ 1]}.  In an expression, you can refer to an element of an array of
type \ARRAY\ \indextype\ \OF\ \elementtype\ whenever you can refer to
a variable of type \elementtype.

You can assign values to individual array elements by using
constants to specify them or use the forall assignment to all elements
of the array together.
Moreover, you must specify the array element being
updated using constants only: for example, you can write {\tt x'[3] :=
true}, but not {\tt x'[y] := true} or even {\tt x'[3+1] := true}.
Thus, there is no direct way to update entry {\tt a} of array {\tt
x}. But, indirectly the same effect can be achieved by using the
forall assignment statement as described below. 

The forall assignment is used to assign values simultaneously to all
array entries. An example of the forall assignment is 

\begin{quote} \tt
forall i x'[i] := if (i = a) then y else x[i] fi
\end{quote}

{\tt i} is the index variable that is bound by the forall
statement. The effect is to assign to entry {\tt i} the value of the
expression on the right side with the value of {\tt i} subsituted for {\tt
i}. Notice also that the above example is just another way of saying 

\begin{quote} \tt
x'[a] := y
\end{quote}

Note that forall assignments may only occur in
\command\ expressions as given in Table~\ref{table-command-syntax}.

Modulo arithmetic is used for indexing whenever the index type of an
array is a range type, e.g., if array {\tt x} has index type {\tt
(0..3)}, {\tt x[3+3]} refers to {\tt x[2]}. An array {\tt x} can be
indexed only by a variable of identically the same type as its index
type. For example, an array with index type {\tt (0..7)} cannot be
indexed with a var of type int, or type {\tt (0..5)}.

An atom can read or await parts of an array, but then no expression in
the atom can index the array by a variable. Similarly, if an atom
controls only part of the array it can only update array elements
specified by constants.

\begin{table}\boxed{\begin{tabbing}
\adeclare\ \gramdef\ \LSQBRAC\intconst\ $|$ \element\RSQBRAC \\[3ex]
\arefer\ \gramdef\ \LSQBRAC\numexpr\ $|$ \enumvar\ $|$ \element\RSQBRAC \\[3ex]
\aassign\ \gramdef\ \adeclare\
\end{tabbing}}
\caption{Array access syntax}
\label{table-array-access-syntax}
\end{table}

\subsection{Bitvectors}\index{type!bitvector}\index{bitvector}
\label{sec-bitvector}

\begin{table}\boxed{\begin{tabbing}
\bitexpr\ \gramdef\ \= \intconst\ $|$ \bitvar \\[1ex]
                    \> $|$ \LPARA \bitexpr \RPARA\ $|$ \MINUS \bitexpr\ \\[1ex]
                    \> $|$ $\bitexpr_1$ \PLUS\ $\bitexpr_2$ \\[1ex]
                    \> $|$ $\bitexpr_1$ \MINUS\ $\bitexpr_2$ \\[1ex]
                    \> $|$ \NOT \bitexpr \\[1ex]
                    \> $|$ $\bitexpr_1$ \binbool\ $\bitexpr_2$ \\[1ex]
                    \> $|$ \IF\ \boolexpr\ \= \THEN\ $\bitexpr_1$ \\[1ex]
                    \>\> \ELSE\ $\bitexpr_2$ \\[3ex]
\bitvar\ \gramdef\ \ident[\nextop][\arefer]
\end{tabbing}}
\caption{Bitvector expression syntax\index{expression!bitvector}}  
\label{table-bitexpr-syntax}
\end{table}

Bitvectors are essentially arrays of booleans, except that you can
perform arithmetic and logical operations on a bitvector as a whole,
and  as a consequence each bitvector variable has to be controlled
by a single atom. The index type of a bitvector is always a range
type. An example of a declaration is 

\begin{quote} \tt
interface x : bitvector 8
\end{quote}

Here, {\tt x} has eight elements indexed by the range type {\tt
(0..7)}. The rules for reading or awaiting parts of bitvector
variables are identical to those for arrays. It is not possible for
different atoms to control parts of the same array; the whole
bitvector has to be controlled by the same atom. The forall assignment
can be used with bitvectors also. 

The operations available for bitvectors are arithmetic operations of addition and subtraction
(modulo $2^k$ for length $k$ bitvectors), and bitwise logical
operations \AND, \OR, \NOT, \IMPLY, and \EQUIV.
Table~\ref{table-bitexpr-syntax} provides the syntax of \bitexpr\ expressions.
Two bitvectors can be added (similarly
for other operations) only if they have the same lengths. Constants
can also be used as an argument to these operations. In such
expressions, the constant is interpreted as a constant bitvector of
the length of the other argument. Let {\tt x}, {\tt y}, {\tt z} be bitvectors of
length three.

\begin{quote} \tt
z' := (x \& y) + 5
\end{quote}

The numeral 5 in the above expression will be interpreted as the
constant bitvector {\tt 101}.












\subsection{Summary of type declarations}\index{type}
\label{sec-summary-type}

Table~\ref{table-type-defs} provides a summary of the syntax of the
types available in \mocha.  The \element\ symbol indicates an
alphanumerical string (beginning with a letter) denoting an element of
an enumeration type. The following restrictions apply:
%
\begin{itemize}

\item 
If the type \indextype\ of an array index is specified by a
previously-defined type name \typename, then \typename\ cannot refer
to another array type.  

\item 
If the type \elementtype\ of an array element is specified through a
previously defined type name \typename, then \typename\ cannot refer
to another array or bitvector type.  In particular, this rules out
arrays of arrays\index{array!multidimensional} (and thus
multidimensional arrays) and arrays of bitvectors. 

\end{itemize}

\begin{table}\boxed{
\begin{tabbing}
\typeid \gramdef\ 
        \= \BOOL\ $|$ \INT\ $|$ \NAT\ $|$ \EVENT\ $|$ \typename \\[1ex]
        \> $|$ \enumtype\ $|$ \rangetype \\[1ex]
        \> $|$ \ARRAY\ \indextype\ \OF\ \elementtype \\[1ex]
        \> $|$ \BITVECTOR\ \sizeconst \\[3ex]
%
\typename\ \gramdef\ \ident \\[3ex]
%
\enumtype\ \gramdef\ \LBRACE$\element_1$\COMA\ \ldots \COMA$\element_n$\RBRACE \\[3ex]
%
\element\ \gramdef\ \ident \\[3ex]
%
\rangetype\ \gramdef\ \LPARA\ZERO\ {\tt ..} \sizeconst\RPARA \\[3ex]
%
\typedefinition\ \gramdef\ 
        \= \TYPEDEF\ \typename\ \COLON\ \enumtype \\[1ex]
        \> $|$ \TYPEDEF\ \typename\ \COLON\ \rangetype \\[1ex]
        \> $|$ \TYPEDEF\ \typename\ \COLON\ \ARRAY\ \indextype\ \OF\ \elementtype \\[1ex]
        \> $|$ \TYPEDEF\ \typename\ \COLON\ \BITVECTOR\ \sizeconst \\[3ex]
%
\indextype\ \gramdef\ \= \enumtype\ $|$ \rangetype \\[3ex]
%
\elementtype\ \gramdef\ \= \BOOL\ $|$ \INT\ $|$ \NAT\ $|$ \typename \\[1ex]
        \> $|$ \enumtype\ $|$ \rangetype \\[1ex]
        \> $|$ \BITVECTOR\ \sizeconst \\[3ex]
%
\sizeconst\ \gramdef\ \= \intconst \\[1ex]
        \> $|$ \LPARA \sizeconst\ \RPARA \\[1ex]
        \> $|$ $\sizeconst_1$ \MINUS\ $\sizeconst_2$ \\[1ex]
        \> $|$ $\sizeconst_1$ \PLUS\ $\sizeconst_2$ \\[1ex]
        \> $|$ $\sizeconst_1$ \STAR\ $\sizeconst_2$ \\[1ex]
        \> $|$ $\sizeconst_1$ \SLASH\ $\sizeconst_2$ \\[1ex]
        \> $|$ $\sizeconst_1$ \PERCENT\ $\sizeconst_2$
\end{tabbing}}
\caption{Syntax of types and type definitions}
\label{table-type-defs}
\end{table}

For arithmetic expressions on \sizeconst: $*$, $/$, and $\%$ have
equal precedence, that is higher than $+$ and $-$ which have equal
precedence. Equal precedence operations are left-associative. 
        
\subsection{Finite and infinite types in
verification}\index{type!finite}\index{type!infinite} 

A type is {\em finite\/} if it consists of finitely many possible
values, and it is {\em infinite\/} otherwise.  Examples of finite
types are boolean, enumeration, range, and event types.  
The infinite types are integer and natural, as well as the composite
types (arrays) built from integers or naturals. 
\mocha\ can deal much better with finite types than with infinite
ones.  In particular, if a \rem\ description consists of
only finite types, then \mocha\ can use both
enumerative\index{model-checking!enumerative} and symbolic
model-checking\index{model-checking!symbolic} for the verification. 
If also infinite types are used, then only enumerative
model-checking\index{model-checking!enumerative} can be used.  This
can prevent the use of some of the most efficient verification methods
implemented in \mocha. 

\section{Macro expansion}\index{macro expansion}
The parser for \rem\ provides a limited macro-expansion facility. 

\mypar
The declaration:

\mypar
{\tt \#define name string} 

\mypar
defines a macro {\tt name} that refers to the string {\tt string}. Any
expression that involves {\tt \$name} will be evaluated after first
substituting {\tt string} for it. 

\mypar
For instance the code fragment:
\begin{verbatim}
#define BITWIDTH 4
type valType      : bitvector  $BITWIDTH
\end{verbatim}
defines valType to be a bitvector of 4 bits. 

\mypar
The other macro allowed is the {\tt foreach} macro. 

\mypar
\begin{verbatim}
#foreach i = (1 .. $BITWIDTH + 1)
type valType_$i : bitvector $i
#endforeach
\end{verbatim}

\mypar
The above code fragment defines four types: {\tt valType\_1}, {\tt valType\_2},
{\tt valType\_3}, and {\tt valType\_4}. The index string may be of
index\_type (see Table \ref{table-type-defs}), but cannot be a
type\_name; it has to be the type itself. That is, 

\mypar
\begin{verbatim}
type food : {pasta, food, water}
#foreach i = food
#endforeach
\end{verbatim}
is not allowed, but {\tt foreach i = \{pasta, food, water\}} is fine. 

\section{Efficiency Considerations}\index{model-checking!efficiency}
\label{sec-efficiency}

The efficiency of the symbolic model-checking
methods\index{model-checking!symbolic}  of \mocha\ depends
crucially on the size of the state space\index{state space}, i.e.\ on
the total number of system states.  A {\em system
state\/}\index{state!system} is simply an assignment of values to the
variables (external, interface, and private) of the \rem\
description.  Several factors affect the size of the state space.
First, if the description includes variables of an infinite type, then
the state space is infinite, and the symbolic model-checking cannot be
performed.  

If all the variables are finite, the size of the state space is
proportional to the (product of) size constants used to dimension
arrays and range types, and to the number of values for enumeration
types.  While this, and similar facts, are well-known in
model-checking, the distinction between read and awaited variables
enables \mocha\ to limit the size of the state space.  In particular,
variables that are awaited but not read, and event variables, do not
contribute to the size of the state space. 


\subsection{Awaited, but not read, variables}

If a variable is never read, but only
awaited,\index{variable!read}\index{variable!awaited} then the
variable does not contribute to the size of the state space.  For
example, the variables {\tt in1a}, {\tt in1b}, {\tt in2a}, {\tt in2b},
{\tt cin1}, {\tt cin2} of Figure~\ref{fig-structuralor} do not
contribute to the size of the state space: they are not read by any of
the modules of the figure, and since they are hidden, we are sure that
no other module can read them. Such variables are also called {\em
history-free}\index{variable!history-free}.  The variables {\tt in1}, {\tt in2} and
{\tt out} may or may not contribute to the size of the state space,
depending on whether or not they are read by some other module. 

Variables that are not read, but only awaited, are often used to model
wires of hardware components.  The reason why these variables are not
read is that these variables, like wires, have no memory, and thus
their state needs not be remembered.  In fact, when computing the
transition relation\index{transition relation} of the system, the
values of these variables in the current state are not used to
determine the successor state. 

Variables that are read are also called {\em
history-dependent}\index{variable!history-dependent} variables.
%\mynote{what if the specification
%refers to these variables? Does \mocha\ do the proper thing then?} 


\subsection{Event variables}\index{variable!event}

Event variables also do not contribute to the size of the state space,
even though they must be both read and awaited by all atoms using
them.  The reason is that they current value (which would contribute
to the size of the state space) is not relevant: all that matters is
whether they retain their value, or whether they are toggled, when
going from one state to the next.  Hence, as in the previous case, the
values of event variables at the current state are not used to
determine the successor state, and hence they need not be remembered.



% ================================================================ 
% here's an aborted example 
\begin{comment}
module Memory
  atom 
    controls m, out
    reads m, write1, write2, read1, read2, in1, in2 
    awaits write1, write2, read1, read2
  update
    [] write1? -> m'   := in1
    [] write2? -> m'   := in2
    [] read1?  -> out' := m 
    [] read2?  -> out' := m 
  endatom     

-- Arbiter for who has the permission to change xl

  lazy atom

    controls grant1, grant2, released1, released2, last, busy

    reads    grant1, grant2, released1, released2, last, busy,
             rel1, rel2, req1, req2

  init
    [] true                              -> busy' := 0; last' := 1
  update
    [] req1 & ~req2 & busy = 0           -> grant1!; busy' := 1; last' := 1
    [] req2 & ~req1 & busy = 0           -> grant2!; busy' := 2; last' := 2
    [] req1 & req2 & busy = 0 & last = 1 -> grant2!; busy' := 2; last' := 2
    [] req1 & req2 & busy = 0 & last = 2 -> grant1!; busy' := 1; last' := 1
    [] rel1 & busy = 1                   -> busy' := 0; released1!
    [] rel2 & busy = 2                   -> busy' := 0; released2!
  endatom

module Process1 
  atom 
  init 
    [] true -> ready' := false 
  update 
    [] pc = 0 
\end{comment}

\section{More examples}
\label{sec:more_examples}

\subsection{Synchronous message-passing protocols}

\begin{figure}
{\tt
 \TYPE\ \msgType\ : \bool \\
 \TYPE\ \sendCtrlType\ : \set{\produce, \send} \\
 \TYPE\ \recCtrlType\ : \set{\receive, \consume} \\
 \MODULE\ \SSender\\
 \qu \EXTERNAL\ \ready\ : \EVENT\\
 \qu \INTERFACE\ \transmit\ : \EVENT; \msgS, \msgP\ : \msgType\\
 \qu \PRIVATE\ \status\ : \sendCtrlType; \doneP\ : \EVENT\\
 \qu \ATOM\ \CONTROLS\ \status, \transmit, \msgS \\
 \qqu \READS\ \status, \transmit, \msgS, \doneP, \msgP, \ready\ \AWAITS\ \doneP, \ready\\
 \qu \INIT\\
 \qu \begin{chtab}
     \TRUE & \status' := \produce
 \end{chtab}\\
 \qu \UPDATE\\
 \qu \begin{chtab}
     \status=\produce \AND \doneP? &\status' := \send\\
     \status=\send \AND \ready? & \transmit!; \msgS' := \msgP; \status' := \produce
 \end{chtab}\\
 \qu \ENDATOM \\
 \qu \LAZY\ \ATOM\ \AProd\ \CONTROLS\ \doneP, \msgP\ \READS\ \status, \doneP, \msgP\\
 \qu \UPDATE\\
 \qu \begin{chtab}
     \status=\produce & \doneP!; \msgP' := \NONDET
 \end{chtab}\\
 \qu \ENDATOM \\
 \ENDMODULE \\\\
 \MODULE\ \SReceiver\\
 \qu \EXTERNAL\ \transmit\ : \EVENT; \msgS\ : \msgType\\
 \qu \INTERFACE\ \ready\ : \EVENT; \msgC\ : \msgType\\
 \qu \PRIVATE\ \status\ : \recCtrlType; \doneC\ : \EVENT; \msgR\ : \msgType\\
 \qu \ATOM\ \CONTROLS\ \status, \msgR\ \\
 \qqu \READS\ \status, \transmit, \doneC\ \AWAITS\ \transmit, \msgS, \doneC\\
 \qu \INIT\\
 \qu \begin{chtab}
     \TRUE & \status' := \receive
 \end{chtab}\\
 \qu \UPDATE\\
 \qu \begin{chtab}
     \status=\receive\ \AND\ \transmit? & \msgR' := \msgS'; \status' := \consume\\
     \status=\consume\ \AND\ \doneC? & \status' := \receive
 \end{chtab}\\
 \qu \ENDATOM \\
 \qu \LAZY\ \ATOM\ \CONTROLS\ \ready\ \READS\ \status, \ready\\
 \qu \UPDATE\\
 \qu \begin{chtab}
     \status=\receive & \ready!
 \end{chtab}\\
 \qu \ENDATOM \\
 \qu \LAZY\ \ATOM\ \ACons\ \CONTROLS\ \doneC, \msgC\ \READS\ \status, \doneC, \msgR\\
 \qu \UPDATE\\
 \qu \begin{chtab}
     \status=\consume & \doneC!; \msgC' := \msgR
 \end{chtab}\\
 \qu \ENDATOM \\
 \ENDMODULE \\
}
\caption{Synchronous message-passing protocol}
\label{fig:syncmsg}
\end{figure}

The modules $\SSender$ and $\SReceiver$ of Figure~\ref{fig:syncmsg}
communicate via events in order to transmit a stream of messages.

%We write $\var : \EVENT$ to declare $\var$ to be a boolean variable that is
%used for modeling events.
%To issue an event represented by~$\var$, we write $\var!$, which stands for
%the assignment $\var':=\NOT\var$.
%To check if an event represented by $\var$ is present, we write $\var?$,
%which stands for the predicate $\var' = \NOT \var$.

\mypar
The private variable $\status$ of the sender indicates if it is producing a
message ($\status=\produce$), or attempting to send a message
($\status=\send$).
The private variable $\status$ of the receiver indicates if it is waiting to
receive a message ($\status=\receive$), or consuming a message
($\status=\consume$).
Messages are produced by the atom $\AProd$, which requires an unknown number
of rounds to produce a message.
Once a message is produced, the event $\doneP$ is issued, and the message is
shown as~$\msgP$
(the actual value of message is chosen non-deterministically from the finite
type $\msgType$).
Once a message has been produced, the sender is ready to send the message,
and $\status$ is updated.
When ready to send a message, the sender sleeps until the receiver becomes
ready to receive, and when ready to receive a message, the receiver sleeps
until the sender transmits a message.

\mypar
The synchronization of both agents is achieved by two-way handshaking in
three subrounds within a single update round.
The first subround belongs to the receiver.
If the receiver is ready to receive a message, it issues the interface event
$\ready$ to signal its readiness to the sender.
The second subround belongs to the sender.
If the sender sees the external event $\ready$ and is ready to send a
message, it issues the interface event $\transmit$ to signal a transmission.
The third subround belongs to the receiver.
If the receiver sees the external event $\transmit$, it copies the message
from the external variable $\msgS$ to the private variable~$\msgR$.
The sender goes on to wait for the production of another message, and the
receiver goes on to consume~$\msgR$.
Messages are consumed by the atom $\ACons$, which requires an unknown number
of rounds to consume a message.
Once a message is consumed, the event $\doneC$ is issued, the consumed
message is shown as~$\msgC$, and the receiver waits to receive another
message.

\subsection{A train controller}
\label{subsec:train}

\begin{figure}
{\small\tt
  \MODULE\ \Train\\
  \qu \INTERFACE\ \pc\ : \set{\away, \wait, \bridge}; \arrive, \leave\ : \EVENT\\
  \qu \EXTERNAL\ \signal\ : \set{\green, \red}\\
  \qu \LAZY\ \ATOM\ \CONTROLS\ \pc, \arrive, \leave\ \READS\ \pc, \arrive, \leave, \signal\\
  \qu \INIT\\
  \qu \begin{chtab}
      \TRUE & \pc' := \away
  \end{chtab}\\
  \qu \UPDATE\\
  \qu \begin{chtab}
      \pc=\away &\arrive!; \pc' := \wait\\
      \pc=\wait\ \AND\ \signal=\green & \pc' := \bridge\\
      \pc=\bridge & \leave!; \pc' := \away
  \end{chtab} \\
  \qu \ENDATOM\\
  \ENDMODULE\\
  \TrainW\ := \Train[\pc,\arrive,\signal,\leave\ := \pcW,\arriveW,\signalW,\leaveW]\\
  \TrainE\ := \Train[\pc,\arrive,\signal,\leave\ := \pcE,\arriveE,\signalE,\leaveE]\\
  \\
  \MODULE\ \Controller\\
  \qu \PRIVATE\ \nearW, \nearE\ : \bool\\
  \qu \INTERFACE\ \signalW, \signalE\ : \set{\green, \red}\\
  \qu \EXTERNAL\ \arriveW, \arriveE, \leaveW, \leaveE\ : \EVENT\\
  \qu \ATOM\ \CONTROLS\ \nearW\ \READS\ \nearW, \arriveW, \leaveW\ \AWAITS\ \arriveW, \leaveW\\
  \qu \INIT\\
  \qu \begin{chtab}
      \TRUE & \nearW' := \FALSE
  \end{chtab}\\
  \qu \UPDATE\\
  \qu \begin{chtab}
      \arriveW? & \nearW' := \TRUE\\
      \leaveW? & \nearW' := \FALSE
  \end{chtab}\\
  \qu \ENDATOM\\
  \qu \ATOM\ \CONTROLS\ \nearE\ \READS\ \nearE, \arriveE, \leaveE\ \AWAITS\ \arriveE, \leaveE\\
  \qu \INIT\\
  \qu \begin{chtab}
      \TRUE & \nearE' := \FALSE
  \end{chtab}\\
  \qu \UPDATE\\
  \qu \begin{chtab}
      \arriveE? & \nearE' := \TRUE \\
      \leaveE? & \nearE' := \FALSE
  \end{chtab}\\
  \qu \ENDATOM\\
  \qu \LAZY\ \ATOM\ \CONTROLS\ \signalW, \signalE\ \READS\ \nearW, \nearE, \signalW, \signalE\\
  \qu \INIT\\
  \qu \begin{chtab}
      \TRUE & \signalW' := \red; \signalE' := \red
  \end{chtab}\\
  \qu \UPDATE\\
  \qu \begin{chtab}
      \nearW\AND\signalE=\red & \signalW' := \green\\
      \nearE\AND\signalW=\red & \signalE' := \green\\
      \NOT\nearW & \signalW' := \red\\
      \NOT\nearE & \signalE' := \red
  \end{chtab}\\
  \qu \ENDATOM\\
  \ENDMODULE\\
  \\
  \TrainSystem\ := \HIDE\ \arriveW, \arriveE, \leaveW, \leaveE\ \IN\ \TrainW\ \pppar\ \TrainE\ \pppar\ \Controller\ \ENDHIDE
}
\caption{Railroad controller}
\label{fig:rail}
\end{figure}

Consider a railway system with two circular railroad
tracks, one for the train traveling clockwise, and the other for the
train traveling counter-clockwise.  At one point of the circle, there
is a bridge that is not wide enough to accommodate both tracks.  The
two tracks merge on the bridge, and for controlling the access to the
bridge, there is a signal at either entrance.  If the signal at the
western entrance is green, then a train coming from the west may enter
the bridge; if the signal is red, the train must wait.  The signal at
the eastern entrance to the bridge controls trains coming from the
east in the same fashion.

\mypar
Figure~\ref{fig:rail} shows the Reactive Modules description for the a
generic train {\tt Train}. The module has three interface variables:
{\tt arrive} and {\tt leave} of type event and
{\tt pc} of enumerative type \{$\away, \wait, \bridge$\};
and one external variable {\tt signal}.
The module has only one atom which controls all the
interface variables. By declaring the atom to be lazy, we are able to
model the assumption regarding independence of the speeds of different
modules.

%The keyword {\bf lazy} indicates that the atom may choose
%not to update the variables, in which case the variables retain
%their old values. This is useful to model the assumption regarding
%independence of the speeds of different modules.
%The keyword {\bf reads} indicates the old values of the
%read variables ({\tt pc, arrive, leave, signal}) are used for updating
%the controlled variables.

\mypar
The module does the following: when the train approaches the bridge,
it sends the event $\arrive$ to the railroad controller and checks the
signal at the entrance to the bridge ($\pc=\wait$).  When the signal
is red, the train stops and keeps checking the signal. When the
signal is green, the train proceeds onto the bridge ($\pc=\bridge$).
When the train exits from the bridge, it sends the event $\leave$ to
the controller and travels around the circular track ($\pc=\away$).
Multiple copies of the {\tt Train} are created by variable renaming.
$\TrainW$, which represents the train traveling clockwise, is
constructed by renaming variables $\pc$ to $\pcW$, $\arrive$ to
$\arriveW$, $\signal$ to $\signalW$ and $\leave$ to $\leaveW$.
$\TrainE$, which represents the train traveling
counter-clockwise, is constructed in a similar fashion.

\mypar
Figure~\ref{fig:rail} also shows a controller controlling the signals to
prevent collisions of the two trains. The complete railway system is represented by the module {\tt RailroadSystem},
which is the composition of the trains with the controller, with
variables $\arriveW, \arriveE, \leaveW$ and $\leaveE$ hidden.

\subsection{A resource manager}

\begin{figure}
\label{fig:rmanager}
\begin{center}
\resizebox{!}{35ex}{type indexType:  (0..3)
type regType: array indexType of bool

-------------------------------------------------------------------------------------
--
--   SPECIFICATION
--
-------------------------------------------------------------------------------------
{\small

\begin{figure}
\begin{mtab}{l}

\MODULE \Spec\\
\qu \INTF \grant, \half_empty:\bool; \grant_index:indexType; \alloc:regType \\
\qu \EXTL \req, \free, \high_priority:\bool; \free_index:indexType \\
  
\qu \ATOM \ALLOC \CONTROLS \alloc 
	\READS  \alloc 
	\AWAITS \req, \grant, \grant_index, \free, \free_index \\
\qqu \INIT\\
\qqu \begin{chtab}
	\true -> \forall i \alloc'[i] := \false
     \end{chtab} \\
\qqu \UPDATE \\
\qqu \begin{chtab}
	\true -> \forall i 
		   \alloc'[i] := 
		     \if (\grant' & \grant_index' = i) \then \true
		     \else \if (\free' & \free_index' = i) \then \false
		     \else \alloc[i] \fi \fi 
     \end{chtab} \\
\qu \ENDA \\
\qu \ATOM \HALF_EMPTY \CONTROLS \half_empty
	\READS \alloc \\
\qqu \INIT \\
\qqu \begin{chtab}
	\true -> \half_empty' := \true
     \end{chtab} \\
\qqu \update \\
\qqu \begin{chtab}
	~\alloc[0]  -> \half_empty' := \if (~\alloc[1] | ~\alloc[2] | ~\alloc[3]) \then \true
					 \else \false \fi \\
 	~\alloc[1]  -> \half_empty' := \if (~\alloc[0] | ~\alloc[2] | ~\alloc[3]) \then \true
					 \else \false \fi \\
	~\alloc[2]  -> \half_empty' := \if (~\alloc[0] | ~\alloc[1] | ~\alloc[3]) \then \true
					 \else \false \fi \\
	~\alloc[3]  -> \half_empty' := \if (~\alloc[0] | ~\alloc[1] | ~\alloc[2]) \then \true
					 \else \false \fi \\
	\default    -> \half_empty' := \false
     \end{chtab} \\
\qu \ENDA \\
\qu \ATOM \GRANT_INDEX \CONTROLS \grant_index \\
\qqu \INIT \UPDATE \\
\qqu \begin{chtab}
	true -> grant_index' := nondet
\end{chtab} \\
\qu \ENDA \\
\qu \ATOM \GRANT \CONTROLS \grant
       \READS \alloc
       \AWAITS \req, \high_priority, \half_empty, \grant_index \\
\qqu \INIT \\
\qqu \begin{chtab}
	true -> grant' := false
\end{chtab} \\
\qqu \UPDATE \\
\qqu \begin{chtab} 
	\req' & \high_priority' & ~\alloc[\grant_index'] -> \grant' := \true \\
	\req'  & \half_empty'    & ~\alloc[\grant_index'] -> \grant' :=
	\true \\
	\default -> \grant' := \false
\end{chtab} \\
\qu \ENDA \\
\ENDM \\

\MODULE \Impl \\
\qu \EXTL  \req, \free, \high_priority:\bool; \free_index:\indexType \\
\qu \INTF \grant,\half_empty:bool; grant_index:indexType;
	alloc:regType \\
\qu \PRIV \sum: \sindexType \\
\qu \ATOM \ALLOC \CONTROLS \alloc 
	\READS  \alloc 
	\AWAITS \req, \grant, \grant_index, \free, \free_index \\
\qqu \INIT\\
\qqu \begin{chtab}
	\true -> \forall i \alloc'[i] := \false
     \end{chtab} \\
\qqu \UPDATE \\
\qqu \begin{chtab}
	\true -> \forall i 
		   \alloc'[i] := 
		     \if (\grant' & \grant_index' = i) \then \true
		     \else \if (\free' & \free_index' = i) \then \false
		     \else \alloc[i] \fi \fi 
     \end{chtab} \\
\qu \ENDA \\
\qu \ATOM \SUM \CONTROLS \sum
  	\READS \sum, \alloc
	\AWAITS \grant, \free, \free_index \\
	
\qqu \INIT \\
\qqu \begin{chtab}
	true -> sum' := 0 
\end{chtab}
\qqu \UPDATE \\
\qqu \begin{chtab}
	\grant' & ~(\free' & \alloc[\free_index']) -> \sum' := \sum + 1	
	~\grant' & (\free' & \alloc[\free_index']) -> \sum' := \sum - 1
\end{chtab}
\qu \ENDA
\qu \ATOM \HALF_EMPTY \CONTROLS \half_empty
	\READS \sum \\
\qqu \INIT \\
\qqu \begin{chtab}
	-- complete this
\end{chtab} \\
\qqu \UPDATE \\
\qqu \begin{chtab}
	-- complete this
\end{chtab} \\
\qu \ENDA \\
\qu \ATOM \GRANT_INDEX \CONTROLS \grant_index
	\READS \alloc \\
\qqu \INIT \\
\qqu \begin{chtab}
	\true -> \grant_index' := 0
\end{chtab} \\
\qqu \UPDATE \\
\qqu \begin{chtab}
	~\alloc[0] -> \grant_index' := 0
	\alloc[0] & ~\alloc[1] -> \grant_index' := 1
	\alloc[0] &  \alloc[1] & ~\alloc[2] -> \grant_index' := 2
	\alloc[0] &  \alloc[1] &  \alloc[2] -> \grant_index' := 3
\end{chtab} \\
\qu \ENDA \\
\qu \ATOM \GRANT \CONTROLS \grant
       \READS \alloc, \sum
       \AWAITS \req, \high_priority \\
\qqu \INIT \\
\qqu \begin{chtab}
	-- complete this 
\end{chtab} \\
\qqu \UPDATE \\
\qqu \begin{chtab}
	-- complete this
\end{chtab} \\
\qu \ENDA \\
\ENDM

\end{mtab}

\caption{Resource manager}
\label{fig:rmanager}
\end{figure}

}







 



}
\caption{Resource Manager}
\end{center}
\end{figure}

\begin{figure}
\small
\begin{verbatim}
type indexType : (0..3)
type sindexType : (0..4)
type regType : array indexType of bool

module Rmanager
  external req : bool; free : bool; free_index : indexType;
           high_priority : bool
  interface grant : bool; grant_index : indexType; alloc : regType;
            half_empty : bool
  atom ALLOC controls alloc
       reads alloc awaits req, grant, grant_index, free, free_index
  init
    [] true -> forall i alloc'[i] := false
  update
    [] true -> forall i alloc'[i] :=
       if(grant' & grant_index'=i) then true else
       if (free' & free_index'=i) then false else alloc[i] fi fi
  endatom
  atom HALF_EMPTY controls half_empty reads alloc
  init
    [] true -> half_empty' := true
  update
    [] ~alloc[0] -> half_empty' :=
       if (~alloc[1] | ~alloc[2] | ~alloc[3]) then true else false fi
    [] ~alloc[1] -> half_empty' :=
       if (~alloc[0] | ~alloc[2] | ~alloc[3]) then true else false fi
    [] ~alloc[2] -> half_empty' :=
       if (~alloc[0] | ~alloc[1] | ~alloc[3]) then true else false fi
    [] ~alloc[3] -> half_empty' :=
       if (~alloc[0] | ~alloc[1] | ~alloc[2]) then true else false fi
    [] default -> half_empty' := false
  endatom
  atom GRANT_INDEX controls grant_index
  init update
    [] true -> grant_index' := nondet
  endatom
  atom GRANT controls grant
       reads alloc awaits req, high_priority, half_empty, grant_index
  init
    [] true -> grant' := false
  update
    [] req' & high_priority' & ~alloc[grant_index'] -> grant' := true
    [] req' & half_empty' & ~alloc[grant_index'] -> grant' := true
    [] default -> grant' := false
  endatom
endmodule
\end{verbatim}
\caption{Resource Manager (Specification)}
\label{fig:rmanagerspec}
\end{figure}

Consider a resource manager used to resolve accesses to instances of a
shared resource. A block diagram of a resource manager is given in
Figure~\ref{fig:rmanager}. There are four instances of the resource
available. The environment interacts with the resource manager to
allocate and free resources. If a resource is available, the resource
manager responds by granting an instance of the resource.


\mypar
Figure~\ref{fig:rmanagerspec} shows a reactive modules description of
the resource manager. The module $\Rmanager$ receives allocate and free
requests through the boolean external variables $\req$ and $\free$
respectively. The environment also sets the external variable
$\highpriority$ to $\TRUE$ in the case of a high-priority request.
For a normal-prioroty request, $\highpriority$ is set to $\FALSE$.
The manager may grant a high-priority request, even if only one
instance of the resource is available. A normal-priority request, on
the other hand may be granted only if at least two instances of the
resource are available.
The module $\Rmanager$ signals grants to the environment through the
interface variable $\grant$ of type $\bool$. The interface variable
$\grantindex$ is set to the index of the resource that is granted.
While freeing the resource, the environment is expected to set
the external variable $\freeindex$ to the appropriate index of the
resouce that is being freed.

\mypar
Private variable \alloc (a boolean array) is used to maintain a bit
for each resource to indicate if it is free or allocated. The
history-free private variable $\halfempty$ of type $\bool$ is
set to true if at least two instances of the resource are free.
It is used in handling normal-priority requests.

\begin{figure}
\small
\begin{verbatim}
module RManagerImpl
  external req : bool; free : bool; free_index : indexType;
           high_priority : bool
  interface grant : bool; grant_index : indexType; alloc : regType;
           half_empty : bool
  private sum : sindexType

  atom ALLOC controls alloc reads alloc
             awaits grant, grant_index, free, free_index
  init
    [] true -> forall i alloc'[i] := false
  update
    [] true -> forall i alloc'[i] :=
       if (grant' & grant_index'=i) then true else
       if (free' & free_index'=i) then false else alloc[i] fi fi
  endatom
  atom SUM controls sum reads sum, alloc awaits grant, free, free_index
  init
    [] true -> sum' := 0
  update
    [] grant' & ~(free' & alloc[free_index']) -> sum' := sum + 1
    [] ~grant' & (free' & alloc[free_index']) -> sum' := sum - 1
  endatom
  atom HALF_EMPTY controls half_empty reads sum
  init
    [] true -> half_empty' := true
  update
    [] sum > 2 -> half_empty' := false
    [] default -> half_empty' := true
  endatom
  atom GRANT_INDEX controls grant_index reads alloc
  init
    [] true -> grant_index' := 0
  update
    [] ~alloc[0] -> grant_index' := 0
    [] alloc[0] & ~alloc[1] -> grant_index' := 1
    [] alloc[0] & alloc[1] & ~alloc[2] -> grant_index' := 2
    [] alloc[0] & alloc[1] & alloc[2] -> grant_index' := 3
  endatom
  atom GRANT controls grant reads alloc, sum awaits req, high_priority
  init
    [] true -> grant' := false
  update
    [] req' & high_priority' & sum <= 3 -> grant' := true
    [] req' & sum <= 2 -> grant' := true
    [] default -> grant' := false
  endatom
endmodule
\end{verbatim}
\caption{Resource Manager (Implementation)}
\label{fig:rmanagerimpl}
\end{figure}

\mypar
Module $\RmanagerImpl$ in Figure~\ref{fig:rmanagerimpl} is a
``less-abstract'' lower-level description of the resource allocator,
Note that the modules $\Rmanager$ and $\RmanagerImpl$ have same
set of external and interface variables.
However, some design decisions have been taken to implement the
non-deterministic choices allowed in $\Rmanager$.
The $\grantindex$ variable always points to the available resource
with the least index, if one is available (as opposed to the
non-deterministic choice in $\Rmanager$) and there is a new private
variable $\total$ that keeps track of the number of resources that
have been allocated. We will later use \mocha\ to show that
$\RmanagerImpl$ is a valid implementation of the module $\Rmanager$.




