\label{chap:tut}
\section{Introduction}
%\mocha\ is a new interactive verification environment for the 
%modular verification of heterogeneous systems. The
%modeling formalism of \mocha---{\em Reactive Modules}---is rich
%enough to simultaneously model heterogeneous components: synchronous,
%asynchronous, speed-independent or real-time, finite or infinite
%state, etc. \mocha\ supports the checking of systems modeled by
%reactive modules against specification in a new logic called
%Alternating-time Temporal Logic (ATL). ATL is more general than CTL
%and is better suited for the specification of the requirements of
%modules, making it a better logic for modular reasoning and analysis
%of systems. Considerable effort has gone into giving \mocha\ a
%substantial GUI making it very easy to use. 

\rem\ is the modeling formalism and input language to \mocha. \rem\ is rich
enough to model systems with heterogeneous components: synchronous,
asynchronous, speed-independent or real-time, finite or infinite
state, etc. In this tutorial chapter we illustrate the facilities of \mocha\
by considering two simple examples, one from hardware and the other
from software. 

The hardware example is a simple counter adapted from a
similar example in the Symbolic Model Verifier ({\sc SMV}) example suite~\cite{};
the tricks developed here should enable the reader to translate any design in the commonly
used subsets of hardware description languages (HDLs) such as
{\sc Verilog}~\cite{} or {\sc Vhdl}~\cite{} into \rem. 
The software example we consider is Peterson's mutual-exclusion
protocol. In addition to modeling these examples in \rem, specifying and verifying correctness
requirements, we walk the reader through an interactive session with
the \mocha\ tool.

\section{$3$-bit counter}
Consider a counter that counts the number of 1's in a binary
input stream, modulo~8. We construct this out of three 1-bit
counter cells. 

\begin{figure}
\centerline{\includegraphics[totalheight=3.0in]{figs/countercell.idraw}}
\caption{{\tt counterCell}: Mealy FSM}
\label{fig:countercell_ckt}
\end{figure}

The Mealy Finite State Machine (FSM)
shown in Figure~\ref{fig:countercell_ckt}, {\tt counterCell} implements the
1-bit counter-cell.  {\tt counterCell} receives a {\tt carryIn} bit as
input, maintains a {\tt sumBit}, and outputs a {\tt carryOut}
bit. {\tt sumBit} is
stored in a register. The register assumes an initial value~$0$ (the
initialization circuitry is not shown in Figure~\ref{fig:countercell_ckt}).
During each clock cycle, the combinational
logic computes the values of the next-state of the register and the
carry-out bit as 
\begin{equation}
\label{eqn:countercell}
{\tt (carryOut, sumBit') = (sumBit + carryIn) }
\end{equation}
where {\tt carryOut} is the high order bit of the sum and {\tt sumBit} is the low order
bit of the sum. Notice, the next-state of {\tt sumBit}, being a latch, is
denoted by {\tt sumBit'}, whereas the other signals are outputs of
combinational gates. It is easy to see that the FSM of Figure~\ref{fig:countercell_ckt}
implements~(\ref{eqn:countercell}). 

The 3-bit counter is produced by connecting up three of the
counter-cells as shown in Figure~\ref{fig:3bit_ckt}. 
\begin{figure}
\centerline{\includegraphics[totalheight=0.75in]{figs/3bitcounter.idraw}}
\caption{$3$-bit counter}
\label{fig:3bit_ckt}
\end{figure}
The {\tt carryOut} bit of the first (second) cell is the {\tt carryIn} bit of the second
cell (third). This is achieved by physically connecting a wire between the
appropriate terminals.  The {\tt sumBit}'s of the three counter-cells
store the three bit sum of the input stream. 


\section{Modeling the $3$-bit counter in \rem}
A \rem\ specification consists of one or more {\em modules}. A module
roughly corresponds to a module in a HDL or a function in a
programming language. A module consists of set of variables and a set
of rules to define the evolution in time of the subset of variables it
controls. The input variables are called the {\em external} variables and
the output variables are called the {\em interface} variables.
The other variables of the module that are not exchanged with
its environment are called {\em private} variables. The private and
interface variables of a module are controlled by it. 

For the {\tt counterCell}, {\tt sumBit} is a private variable, {\tt carryIn} is an
external variable, and {\tt carryOut} is an interface variable, as in Figure~\ref{fig:counter_cell_rm}. 

In synchronous digital hardware, in each cycle, the primary inputs
(PIs) and the present state values of latches propagate to determine
the next state values of latches and the outputs (including the
primary outputs). Then, all the latches simultaneously assume their
next state values, and the process repeats in this manner every clock cycle. 

Similarly, with \rem, in every {\em round} (like cycle) the variables
are updated. No apriori distinction is made between variables for
latches and variables for combinational gates. Instead, the semantic
distinction between these is captured in their update rules specified
by {\em atoms}. 

If the value of a variable $x$---the present value---at the beginning
of a round is denoted by $x$, the update round computes the next or
primed value $x'$. The initial value computation that happens in the
init round is specified by the init section of the atom, and the
subsequent updates in the update rounds are specified by the update
sections of the atoms.

An atom specifies the next state function as a set of guarded
commands: when the guard is satisfied the action is taken.  The
next-state function of the variable an atom controls may be a function
of the present value of variables (i.e., unprimed version) it {\em
reads} and primed value of variables it {\em awaits}. In particular,
both the guards and the commands may only involve expressions over the
unprimed variables that are read and the primed variables that are
awaited.  For instance, the next value of {\tt carryOut} (i.e. {\tt carryOut'}),
is a function of the present value of {\tt sumBit} (being a latch) and the
next value of {\tt carryIn} (being a primary input): its atom, in
Figure~\ref{fig:counter_cell_rm}, reads {\tt sumBit} and awaits {\tt
carryIn}. 

Also, this means that the next value of a variable will be computed
after the next values of the variables it awaits are computed.
Note that cyclic await dependencies are disallowed.
In fact, the partial order induced by the await dependencies among all atoms
must be completable to a linear order.

Figure~\ref{fig:counter_cell_rm} shows the \rem\ description of the
counter-cell. In each round, the next value of carryIn is first
computed, and then sumBit and carryOut may be computed in each order. 
A sample execution of the module is given in the table below:

\begin{center}
\begin{tabular}{r|rrrrrc}
Round: 
 & 0 (initial) & 1 & 2 & 3 & 4 & $\cdots$ \\ \hline
{\tt carryIn}:  & {\sc t} & {\sc t} & {\sc t} & {\sc f} & {\sc t} & $\cdots$ \\
{\tt sumBit}:  & {\sc f} & {\sc t} & {\sc f} & {\sc f} & {\sc t} & $\cdots$ \\
{\tt carryOut}:   & {\sc f} & {\sc f} & {\sc t} & {\sc f} & {\sc f} & $\cdots$ \\ \hline
\end{tabular}
\end{center}



\begin{figure}
\begin{verbatim}
module counterCell
  private sumBit : bool
  external carryIn : bool
  interface carryOut : bool

-- this is a comment; a comment line starts with --

atom controls sumBit reads sumBit awaits carryIn 
init
  [] true -> sumBit' := false
update
  [] ~sumBit -> sumBit' := carryIn'
  [] sumBit  -> sumBit' := if (~carryIn') then true else false fi
endatom

atom controls carryOut reads sumBit awaits carryIn
init
  [] true -> carryOut' := false
update
  [] true -> carryOut' := sumBit & carryIn'
endatom

endmodule  -- end counterCell

\end{verbatim}
\caption{1-bit counter-cell}
\label{fig:counter_cell_rm}
\end{figure}


From the exercise of modeling the counter-cell in \rem, we can derive
the rules in Table~\ref{tab:recipe_mealy_fsm_rm} to translate a
Mealy FSM module to \rem. 

\begin{table}\boxed{
If a signal $x$ is a function of signal $y$, then in the atom
controlling variable $x$:
\bigskip

\begin{enumerate}

\item \label{item:latch}
if $y$ is a latch-output, the variable $y$ should be read, i.e.,
its unprimed version should be used.

\item if $y$ is a gate output, the variable $y$ should be awaited,
i.e., its primed version should be used. 

\item \label{item:pi}
if $y$ is an input signal, the variable $y$ should be awaited,
i.e., its primed version should be used. 
\end{enumerate}}
\caption{Recipe for variable assignment in translating a Mealy FSM
module to \rem}
\label{tab:recipe_mealy_fsm_rm}
\end{table}
\subsection{Module instantiation and composition}
New modules can be created from previously defined modules by
instantiating them with renaming as well as by composing together modules. The
desired interconnection of the counter-cells in Figure~\ref{fig:3bit_ckt} is done as follows: 


\begin{verbatim}
cell0 := counterCell[carryIn, carryOut := input, out0]
cell1 := counterCell[carryIn, carryOut := out0, out1]
cell2 := counterCell[carryIn, carryOut := out1, out2]
threebitcounter := hide out0, out1 in (cell0 || cell1 || cell2) endhide
\end{verbatim}

Three instances of {\tt counterCell} are created by renaming the
external and interface variables using common names for the desired
interconnections. For instance, {\tt out0} is the {\tt carryOut} of
{\tt cell0} and the {\tt carryIn} of {\tt cell1}. 

The actual interconnection is achieved by the parallel composition of
the three instances: {\tt cell0}, {\tt cell1}, and {\tt cell2}, and
hiding the {\tt carryOut} variables of {\tt cell0} and {\tt cell1},
making these private variables of {\tt threebitcounter}. The module \\
{\tt threebitcounter} has one external variable {\tt input} and one
interface variable {\tt out2}. 

The order of {\tt cell0}, {\tt cell1}, and {\tt cell2}
in the parallel composition statement is not important.
For example, the previous parallel composition statement is equivalent to the following line:

\begin{verbatim}
threebitcounter := hide out0, out1 in (cell1 || cell0 || cell2) endhide
\end{verbatim}

\section{Other issues in modeling hardware}

\subsection{Connecting a latch output to a primary input}

At times we might need to compose modules and connect a latch from one
module to the input of another module. This would lead to a violation
of item~\ref{item:latch} of Table~\ref{tab:recipe_mealy_fsm_rm} since
we decreed that an input variable is to be awaited
(by item~\ref{item:pi} of Table~\ref{tab:recipe_mealy_fsm_rm}). 

To work around this we have to insert a unit-delay non-inverting
buffer with the appropriate initial value between the latch-output and
the input of the module it connects to. A unit-delay non-inverting
buffer with initial value~$0$ is modeled in \rem\ as follows:

\begin{verbatim}
module noninvertingbuffer0
  external latchoutput
  interface onedelayed

--unit delay non inverting buffer with initial value 0 
atom controls onedelayed reads latchoutput
init 
  [] true -> onedelayed' := false
update
  [] true -> onedelayed' := latchoutput
endatom 
endmodule
\end{verbatim}

We have one additional rule to invoke in translating a network of
Mealy FSMs to \rem, and it is stated in Table~\ref{tab:addtl_rule}. 

\begin{table}\boxed{
If $y$ is a latch-output and the output of a module $A$ and is being
connected to $x$ an external variable of module $B$, introduce a
unit-delay non-inverting buffer with the appropriate initial value
between $y$ and $x$. 
}
\caption{Rule for connecting the latch-output of one module to the
input of another module}
\label{tab:addtl_rule}
\end{table}

\subsection{Non-determinism}
Non-determinism is useful in modeling systems at an abstract
level. There are two ways to model non-determinism in \rem:
\begin{enumerate}
\item With a non-deterministic assignment. 
\item By having multiple guarded commands with the same guard. 
\end{enumerate}

We illustrate both ways of having non-determinism with a simple
example. We design a module that will serve as the input to the
three-bit-counter {\tt threebitcounter} to result in a closed system,
i.e., one with no inputs (no external variables), and will
non-deterministically output a~$0$ or~$1$. The first way of modeling
this is shown below:

\begin{verbatim}
module nondetinput
  interface output : bool

atom controls output
init update
  [] true -> output' := nondet
endatom 
endmodule 
\end{verbatim}

The variable {\tt output} is assigned a value non-deterministically from the
range of values it can assume. 
The keyword \NONDET\ is used to assign to a variable a
nondeterministic element of the variable domain.

The second equivalent way of modeling the module {\tt nondetinput} is by using multiple
guarded commands with the same guard is:

\begin{verbatim}
module nondetinput
  interface output : bool

atom controls output
init update
  [] true -> output' := false
  [] true -> output' := true
endatom 
endmodule 
\end{verbatim}

This second method is useful (and the only way) currently to model a
non-deterministic assignment to a variable from a {\em subset} of the
possible values it may assume. 

\subsection{Simple exercises}
$\clubsuit$ Modify the module {\tt threebitcounter} to output a signal
whenever the three bit sum has value~$0$. 

\noindent
$\clubsuit$ Create a new
module that is the composition of threebitcounter and nondetinput. 

\section{Running \mocha}
All the module definitions have to be entered into a single file named
typically with the suffix .rm; in our case, (say) this file is
counter.rm (Figure~\ref{fig:finalfile}). \mocha\ is invoked by typing {\tt mocha} at the shell
prompt. If you do not want to bring up \mocha\ with its GUI, start
\mocha\ in the text mode typing {\tt mocha -t}.

\begin{figure}
\begin{verbatim}
-- this is a comment
-- 3 bit counter found in the SMV example suite 
module counterCell
  private sumBit : bool
  external carryIn : bool
  interface carryOut : bool

atom controls sumBit reads sumBit awaits carryIn 
init
  [] true -> sumBit' := false
update
  [] ~sumBit -> sumBit' := carryIn'
  [] sumBit -> sumBit' := if (~carryIn') then true else false fi
endatom

atom controls carryOut reads sumBit awaits carryIn
init
  [] true -> carryOut' := false
update
  [] true -> carryOut' := sumBit & carryIn'
endatom
endmodule  -- end counterCell

cell0 := counterCell[ carryIn, carryOut := input, out0 ]
cell1 := counterCell[ carryIn, carryOut := out0, out1 ]
cell2 := counterCell[ carryIn, carryOut := out1, out2 ]

threebitcounter := hide out0, out1 in
   (cell0 || cell1 || cell2) endhide

module nondetinput
  interface output : bool

atom controls output
init update
  [] true -> output' := nondet
endatom 
endmodule 

InputModule := nondetinput[ output := input ]
closedthreebitcounter := hide input in
   (InputModule || threebitcounter) endhide
\end{verbatim}
\caption{Input file: counter.rm}
\label{fig:finalfile}
\end{figure}

The module  is read and parsed with the {\tt read\_module} command.
\mocha\ displays the names of the modules that were successfully
parsed. In the case of a parse error, an appropriate message is
displayed. 

\begin{verbatim}
mocha:   read_module counter.rm
Module counterCell is composed and checked in.
Module cell0 is composed and checked in.
Module cell1 is composed and checked in.
Module cell2 is composed and checked in.
Module threebitcounter is composed and checked in.
Module nondetinput is composed and checked in.
Module InputModule is composed and checked in.
Module closedthreebitcounter is composed and checked in.
parse successful.
\end{verbatim}

The command {\tt show\_mdls} lists the modules that have been read
in. 

\begin{verbatim}
mocha: show_mdls
closedthreebitcounter
InputModule         
cell0               
cell1               
cell2               
counterCell         
threebitcounter     
\end{verbatim}

\mocha\ provides many methods of verifying the correctness of a design:
execution (i.e., simulation), invariant checking, refinement checking,
and ATL model checking. We highlight execution and ATL model checking
in this tutorial, being the distinguishing features of \mocha, and
treat refinement checking and assume-guarantee reasoning in subsequent
chapters. 

\subsection{Executing modules}
\mocha\ allows the user to execute any module in three modes: manual,
random, and game, via a Tk-based GUI for interacting with the tool and
viewing the execution trace. To execute a module, first read it in
using the {\tt read\_module} command. Then choose it by selecting it in the
browser, that is in turn brought up by choosing under the browse
option under the ``File'' pull down menu. 

After selecting the module, press the open button. For instance,
choose the module {\tt closedthreebitcounter}. A new window will pop
up containing the\\ \rem\ definition of the module selected. Now click
the ``Execute'' button. A new window pops up giving the three options
for execution as radio buttons. For the Game execution option, the
user gets to choose the atoms for which he can specify the next state
(i.e., resolve the non-determinism); the other atoms' next states will
be chosen by \mocha; use the browse button to choose the atoms the
user wants to control. 

The upper window presents (by default) a table of the external and interface
variables for the chosen module and possible values at the current
state. The user gets to choose the particular tuple he wants to
proceed with in the case of manual or game execution.  Additional
variables to view can be selected by choosing the ``Select Variables''
option under the options pull down menu. For instance, if you chose to
execute {\tt closedthreebitcounter} under the game mode, and chose to
play the module {\tt InputModule}, you will have to choose the
input variable to be made visible. When the user's button is lit, the
user should make a choice from the upper window and then press the
``Go!'' button. When it's the system's turn pressing the ``Go!''
button, executes the system's move. Try it!

\section{ATL model checking}

ATL is a new temporal logic that subsumes CTL and is appropriate for
reasoning about open systems. The difference between ATL and CTL is
that the path quantifiers in ATL are parametrized by a set of atoms,
and a formula is true along all paths that the parametrized atoms can
take the system into, no matter how the other atoms behave. 

For instance, the CTL formula $A F (out2)$ is not true with respect to
the module {\tt closedthreebitcounter}.
We obtain a counterexample to the formula if the input to the counter is set to~$0$.
In this case the sum is always~$0$ and the carry-out from the third ({\tt out2})
counter-cell never becomes~$1$.

The formula $A F (out2)$ stated in ATL is: $$<< >> F (out2) $$ where the set
of atoms parametrizing the path quantifier (within angle brackets) is
empty. The formula is to be read as follows: no matter how the agents
(i.e, atoms) behave it is the case that the system reaches a state
where {\tt out2} is true.

Similarly, the exists ($E$) path quantifier of CTL is the ATL path
quantifier parametrized by all the atoms. For instance, the module
{\tt closedthreebitcounter} satisfies the CTL formula $E F (out2)$.
The equivalent ATL formula is $$<< U >> F (out2)$$ where $U$
stands for a list of all the atoms in the module {\tt closedthreebitcounter}. 

ATL thus let us specify games where we can divide up the set of
atoms into two teams; with some atoms in one team and the remainder in the
other team, we can pose adversarial questions as to whether a team
can enforce a condition no matter how the other team behaves.

For instance, we can pose the following ATL formula that is stronger than the CTL formula $E F
(out2)$:
\begin{equation}
\label{eqn:atl_formula}
 << InputModule >> F (out2) 
\end{equation}
This formula asks: what are the states from which the module
{\tt InputModule}, i.e. the team comprised of atoms in the module {\tt
InputModule} has a strategy to make sure that {\tt
closedthreebitcounter} reaches  a state where {\tt out2} is true no
matter how the other modules (atoms) behave. Notice that in CTL the
only two teams possible are where one is the empty set and the other
all the atoms, whereas ATL let us partition the atoms into two teams
any way we want. 

The ATL formula of (\ref{eqn:atl_formula}) is also true of {\tt
closedthreebitcounter}, because again (for instance) a winning strategy
for {\tt InputModule} is to set the input to~$1$ sometime.
The reader should be convinced that it is not
difficult to come up with instances of modules that can be
distinguished by ATL formulae but not by CTL formulae. 
Note that in ATL formulae we may use the short hand $A$ and $E$ to stand for
the parametrized path quantifier with the empty set of atoms and the set of all atoms, respectively.

The ATL formulae can be entered into a file whose name is suffixed
by {\tt .spec}, named say {\tt counter.spec}:

\begin{verbatim}
atl "atl1" A F (out2) ;
atl "atl2" E F (out2) ;
atl "atl3" << InputModule >>  F (out2); 
\end{verbatim}

Lines containing ATL formulae start with word {\tt atl}, followed by
the name of the formula, and then the formula, with a semi-colon
terminating the line. The command {\tt show\_spec} prints out the names
of ATL formulae read in (as well as invariants read in). Try it. 

The command for ATL model checking is {\tt atl\_check}. The following command:

\begin{verbatim}
atl_check closedthreebitcounter atl1
\end{verbatim}

\noindent
which says check if the module {\tt closedthreebitcounter} satisfies
the ATL formula {\tt atl1}. When you run this, you will get a message
reporting a failure. On the otherhand:

\begin{verbatim}
atl_check closedthreebitcounter atl2
\end{verbatim}

\noindent
should pass as should {\tt atl\_check closedthreebitcounter atl3}. 

%As we saw above, ATL model checking lets one separate the environment
%of a module from the module itself. It is the hoped that ATL model
%checking will form the foundation of a good modular design
%methodology.

\section{Peterson's mutual exclusion protocol}

As an example of a concurrent program consisting of processes that
communicate through read-shared variables, we consider a mutual-exclusion
protocol, which ensures that no two processes simultaneously access a common
resource.
The modules $P_1$ and $P_2$ of Figure~\ref{fig:pete} model the two processes
of Peterson's solution to the mutual-exclusion problem for shared variables.
Each process  has a program counter ($\statuso,~\statust$) and a flag ($x1,~x2$), both
of which can be observed by the other process.
The program counter indicates whether a process is outside its critical
section ($\outcs$), requesting the critical section
($\reqcs$), or occupying the critical section ($\incs$).
In each update round, a process looks at the latched values of all
variables (reads them) and, nondeterministically, either updates its controlled variables or sleeps
(i.e., leaves the controlled variables unchanged: achieved by the last
guarded command with a true guard and no actions), without waiting to see
what the other process does.
Note that each process may sleep for arbitrarily many rounds:
nondeterminism is used to ensure that there is no relationship between the
execution speeds of the two processes.

\mypar
{\bf Interleaving.}
Unlike in interleaving models, both processes may modify their variables in
the same round.
While Peterson's protocol ensures mutual exclusion even under these weaker
conditions, if one were to insist on the interleaving assumption, one would
add a third module that, in each update round, nondeterministically schedules
either or none of the two processes.
Alternatively, one could describe the complete protocol as a single module
containing a single atom whose update action is the union of the update
actions of the atoms of Figure~\ref{fig:pete}.
The guarded command that specifies a union of actions consists simply of the
union of all guarded assignments of the individual actions.
This style of describing asynchronous programs as an unstructured collection
of guarded assignments is pursued in formalisms such as
{\sc Unity}~\cite{ChandyMisra88} and {\sc Mur}$\varphi$~\cite{Dill96}.

\mypar
{\bf Write-shared variables.}
The original formulation of Peterson's protocol uses a single write-shared
boolean variable~$x$, whose value always corresponds to the value of the
predicate $x1=x2$ in our formulation.
If one were to insist on modeling $x$ as a write-shared variable, one would
add a third module with the interface variable $x$ and awaited external
variables such as {\em P$_i$\_sets\_x\_to\_}0, which is a boolean interface
variable of the $i$-th process that indicates when the process wants to set
$x$ to~0.
This style of describing write-shared memory makes explicit what happens when
several processes write simultaneously to the same location.

\section{Verification}
The two modules {\tt P1} and {\tt P2} are composed to give module {\tt Pete}. 
We verify that the version of Peterson's mutual exclusion
protocol in Figure~\ref{fig:pete} does implement mutual exclusion. 
The following invariant says that both processes are not simultaneously in their
critical sections:\\

{\noindent\tt
inv "mutex" $\sim$ (pc1 = inCS \AND\ pc2 = inCS);
}\\

The invariant can be entered into a file (say) {\tt pete.spec}.
The name of the invariant is {\tt mutex}.
Start up \mocha, and first read in the
module definitions of Figure~\ref{fig:pete} by typing:\\

{\noindent\tt
mocha: read\_module pete.rm
}\\

\noindent
The specification is read in with the {\tt read\_spec} command:\\

{\noindent\tt
mocha: read\_spec pete.spec
}\\

\noindent
The mutual-exclusion is tested with the command:\\

{\noindent\tt
mocha:  inv\_check Pete mutex
}\\

\noindent
which says check if the module {\tt Pete} satisfies the invariant {\tt
mutex}. {\tt Pete} should satisfy the invariant. 

\section{Where to find the files for the examples in this chapter}

The \rem\ files as well as the ATL specs., etc., can be found on the WWW at
{\tt http://www-cad.eecs.berkeley.edu/mocha/demo.html}

\begin{figure}
{\tt
  \MODULE\ P1\\
  \qu \INTERFACE\ \statuso : \set{\outcs,\reqcs,\incs} ;  x1 : \bool\\
  \qu \EXTERNAL\ \statust : \set{\outcs,\reqcs,\incs} ;  x2 : \bool\\
  \qu \ATOM\ \CONTROLS\ \statuso, x1\ \READS\ \statuso, \statust, x1, x2\\
  \qu \INIT\\
  \qu \begin{chtab}
      \TRUE & \statuso' := \outcs
  \end{chtab} \\
  \qu \UPDATE\\
  \qu \begin{chtab}
      \statuso=\outcs & \statuso' := \reqcs ;  x1' := x2\\
      \statuso=\reqcs\ \AND\ (\statust=\outcs\ \OR\ \NOT(x1=x2)) & \statuso' := \incs\\
      \statuso=\incs & \statuso' := \outcs\\
      \TRUE &
  \end{chtab} \\
  \qu \ENDATOM \\
  \ENDMODULE \\\\

  \MODULE\ P2\\
  \qu \INTERFACE\ \statust : \set{\outcs,\reqcs,\incs} ;  x2 : \bool\\
  \qu \EXTERNAL\ \statuso : \set{\outcs,\reqcs,\incs} ;  x1 : \bool\\
  \qu \ATOM\ \CONTROLS\ \statust, x2\ \READS\ \statuso, \statust, x1, x2\\
  \qu \INIT\\
  \qu \begin{chtab}
      \TRUE & \statust' := \outcs
  \end{chtab}\\
  \qu \UPDATE\\
  \qu \begin{chtab}
      \statust=\outcs & \statust' := \reqcs ; x2' := \NOT x1\\
      \statust=\reqcs\ \AND\ (\statuso=\outcs\ \OR\ x1=x2) & \statust' := \incs\\
      \statust=\incs & \statust' := \outcs\\
      \TRUE &
  \end{chtab} \\
  \qu \ENDATOM \\
  \ENDMODULE \\
}
{\tt
Pete:= hide x1, x2 in (P1 || P2) endhide\\
}
\caption{Asynchronous mutual-exclusion protocol}
\label{fig:pete}
\end{figure}
