
\subsection{Arrays}\index{type!array}\index{array}

\mocha\ also has array and
bitvector\index{type!bitvector}\TYPE{bitvector} types.  Arrays are
essentially as in ordinary programming languages, except that
multidimensional arrays are not provided, and other minor limitations
are present.  Each array variable has to be controlled by the same
module, but different array elements can be controlled by different
atoms: hence, you can think at arrays as a bus of wires or signals,
not necessarily controlled jointly. 
The syntax for declaring an array type is: 
%
\begin{quote} 
  \ARRAY\ \indextype\ \OF\ \elementtype
\end{quote}
%
Type \indextype\ is the type of the array index, and can be a boolean,
an enumeration type, or a range type (perhaps the most common case).
Type \elementtype\ is the type of the array elements, and can be a
boolean, integer, natural, enumeration, range, or bitvector type.
An example of declaration is 
%
\begin{quote} \tt
  interface x: array (0..10) of bool
\end{quote}
%
You can declare that an atom controls the whole array {\tt x} by
writing {\tt controls x}.  If the atom controls only some elements of
{\tt x}, you can specify these simply by listing them, as in {\tt
controls x[0], x[2], x[4]}.  You can specify which elements are read
or awaited in a similar way. 
Note that the array elements that are controlled, read, or awaited
must be specified using constants only: you cannot write {\tt controls
x[1 + 2]} or {\tt awaits x[a]}. 
If an array is controlled by the module (i.e.\ if it is declared as
{\tt interface} or {\tt private} in the module), then all array
elements must be controlled by some atom of the module.

You can refer to array elements by writing the variable name followed
by a range expression in square brackets, as in {\tt x[4]} or {\tt x[z
+ 1]}.  In an expression, you can refer to an element of an array of
type \ARRAY\ \indextype\ \OF\ \elementtype\ whenever you can refer to
a variable of type \elementtype.

You can assign values only to individual array elements, not to arrays
as a whole.  Moreover, you must specify the array element being
updated using constants only: for example, you can write {\tt x'[3] :=
true}, but not {\tt x'[y] := true} nor even {\tt x'[3+1] := true}.

forall assignment

How is the \AWAITS relation computed: on array elements? How can you
check that you are referring only to array elements that you declared
as read? Or must you read whole vectors at once? 


\subsection{Bitvectors}\index{type!bitvector}\TYPE{bitvector}

Bitvectors are essentially arrays of booleans, except that you can
perform arithmetic and logical operations on a bitvector as a whole,
and that as a consequence each bitvector variable has to be controlled
by a single atom. 
