\section{Reactive Modules}

A discrete reactive system is a collection of variables that, over time,
change their values in a sequence of rounds.
We model discrete reactive systems that may interact with each other by
mathematical objects that are called {\em reactive modules\/}
(or {\em modules}, for short).

\mypar {\bf Variables vs.\ events.} A module $\ro$ has a finite
set of typed variables, denoted~$\rovars{\ro}$. A {\em state\/} of
$\ro$ is a valuation for the variables in~$\rovars{\ro}$. Events,
such as clock ticks, are modelled by toggling boolean variables.
For example, the event that is represented by the boolean variable
$\tick$ occurs whenever the module proceeds from a state $\state$
to a state $\astate$ such that
$\state[\tick]\not=\astate[\tick]$.\footnote{
  If $\state$ is a state and $\var$ is a variable in the domain of~$\state$,
  we write $\state[\var]$ for the value assigned by $\state$ to~$\var$.}

\mypar
{\bf System vs.\ environment.}
The module $\ro$ represents a system that interacts with an environment.
Some of the variables in $\rovars{\ro}$ are updated by the system, and the
other variables in $\rovars{\ro}$ are updated by the environment.
Hence, the set $\rovars{\ro}$ is partitioned into two sets:
the set $\cvars{\ro}$ of {\em controlled variables}, and the set
$\extl{\ro}$ of {\em external variables}.

\mypar
{\bf States vs.\ observations.}
Not all controlled variables of the module $\ro$ are visible to the
environment.
Hence, the set $\cvars{\ro}$ is partitioned further into two sets:
the set $\priv{\ro}$ of {\em private variables}, and the set $\intf{\ro}$ of
{\em interface variables}.
The interface variables and the external variables are visible to the
environment, and therefore called {\em observable}.
The set of observable variables of $\ro$ is denoted~$\ovars{\ro}$.
An {\em observation\/} of $\ro$ is a valuation for the variables
in~$\ovars{\ro}$.
The various classes of module variables and their relationships are
summarized in Table~1.
The distinction between private, interface, and external variables is similar
to the distinction between internal, output, and input events in the
formalism of I/O automata~\cite{Lynch96}.

\begin{table}
\begin{center}
\begin{tabular}{|c|c|c|}
  \hline
  $\priv{\ro}$ &$\intf{\ro}$&$\extl{\ro}$\\
  \hline
  \multicolumn{2}{|c|}{$\cvars{\ro}$} & \\
  \hline
  &\multicolumn{2}{c|}{$\ovars{\ro}$}\\
  \hline
  \multicolumn{3}{|c|}{$\rovars{\ro}$}\\
  \hline
\end{tabular}
\end{center}
\caption{Module variables}
\end{table}

\mypar
{\bf Asynchrony vs.\ synchrony.}
During the execution of the module $\ro$ the variables in $\rovars{\ro}$
change their values in a sequence of rounds.
As in synchronous programming languages
\cite{BerryGonthier88,BRS93,Halbwachs93:book}, each round consists of several
subrounds, and the system and the environment take turns in executing
subrounds to update variables.
However, unlike in synchronous programming languages, the number of subrounds
per round is fixed:
each variable is updated in exactly one subround of each round.
For this purpose, the controlled variables of a module are partitioned into
groups called {\em atoms}, and the variables within a group are updated
simultaneously, in the same subround.
The atoms are partially ordered.
If atom $\atom$ precedes atom $\aatom$ in the partial order, then in each
round, the $\atom$-subround must precede the $\aatom$-subround, and the
updated values of the variables controlled by $\aatom$ may depend on the
updated values of the variables controlled by~$\atom$.

\mypar
{\bf Latched vs.\ updated values.}
In each round, every variable $\var$ has two values.
The value of $\var$ at the beginning of the round is called the
{\em latched value}, and the value of $\var$ at the end of the round is
called the {\em updated value}.
We use unprimed symbols, such as~$\var$, to refer to latched values, and
primed symbols, such as~$\var'$, to refer to the corresponding updated
values.
Given a set $\vars$ of unprimed symbols, we write $\vars'$ for the set of
corresponding primed symbols.

\mypar
{\bf Initial vs.\ update actions.}
The module $\ro$ proceeds in a sequence of rounds.
The first round is an {\em initialization round}, during which the variables
in $\rovars{\ro}$ are initialized.
Each subsequent round is an {\em update round}, during which the variables in
$\rovars{\ro}$ are updated.
The initialization and updating of variables are specified by actions.
An {\em action\/} from $\vars$ to $\avars$ is a nondeterministic guarded
command ---that is, a list of guarded assignments--- so that
(1)~all guards and all right-hand-sides of assignments are expressions over
  the variables in~$\vars$, and
(2)~all left-hand-sides of assignments are variables from~$\avars$.
The action $\act$ from $\vars$ to $\avars$ is {\em executable\/} if the
disjunction of all guards is true.
Executable actions are enabled in all states.

\mypar
{\bf Atoms vs.\ modules.}
The controlled variables of a module $\ro$ are partitioned into atoms;
that is, every controlled variable of $\ro$ is controlled by one and only one
atom of~$\ro$.
The initialization round and all update rounds consist of several subrounds,
one for the environment and one for each atom.
For each atom~$\atom$, in the $\atom$-subround of the initialization round,
all variables controlled by $\atom$ are initialized simultaneously, as
defined by an initial action.
In the $\atom$-subround of each update round, all variables controlled by
$\atom$ are updated simultaneously, as defined by an update action.

\begin{definition}{atom}\it
  {\em [{\bf Atom}]}
  Let $\vars$ be a finite set of typed variables.
  An {\em $\vars$-atom\/} $\atom$ consists of a declaration and a body.
  The atom declaration consists of a set $\cvars{\atom}\subseteq\vars$ of
  {\em controlled variables}, a set $\rvars{\atom}\subseteq\vars$ of
  {\em read variables}, and a set
  $\wvars{\atom}\subseteq\vars\bs\cvars{\atom}$ of {\em awaited variables}.
  The atom body consists of an executable {\em initial action\/}
  $\init{\atom}$ from $\wvars{\atom}'$ to $\cvars{\atom}'$ and an executable
  {\em update action\/} $\update{\atom}$ from
  $\rvars{\atom}\cup\wvars{\atom}'$ to~$\cvars{\atom}'$.
\end{definition}

\mypar
In the initialization round, the initial action of the atom $\atom$ assigns
initial values to the controlled variables as a nondeterministic function of
the initial values of the awaited variables.
In each update round, the update action of $\atom$ assigns updated values to
the controlled variables as a nondeterministic function of the latched values
of the read variables and the updated values of the awaited variables.
Hence, in each round, the $\atom$-subround can take place only after all
awaited variables have already been updated.
The variable $\var$ {\em awaits\/} the variable~$\avar$, written
$\var\awaits_\atom\avar$, if $\var\in\cvars{\atom}$ and
$\avar\in\wvars{\atom}$.
Now, we can define reactive modules.

\begin{definition}{module}\it
  {\em [{\bf Module}]}
  A {\em (reactive) module\/} $\ro$ consists of a declaration and a body.
  The module declaration is a finite set $\rovars{\ro}$ of typed variables
  that is partitioned as shown in Table~1.
  The module body is a set $\roatoms{\ro}$ of $\rovars{\ro}$-atoms such that
  (1)~$(\cup_{\atom\in\roatoms{\ro}}\,\cvars{\atom})=\cvars{\ro}$;
  (2)~for all atoms $\atom$ and $\aatom$ in~$\roatoms{\ro}$,
    $\cvars{\atom}\cap\cvars{\aatom}=\emptyset$;
    and
  (3)~the transitive closure
    $\awaits_\ro\,=(\cup_{\atom\in\roatoms{\ro}}\awaits_\atom)^+$ is
    asymmetric.
\end{definition}

\mypar
The first two conditions ensure that the atoms of $\ro$ control precisely the
variables in $\cvars{\ro}$, and that each variable in $\cvars{\ro}$ is
controlled by precisely one atom.
The third condition ensures that the await dependencies among the variables
in $\rovars{\ro}$ are acyclic.
A linear order $\atom_0,\ldots,\atom_{k-1}$ of the atoms in $\roatoms{\ro}$
is {\em consistent\/} if for all $0\le i<j<k$, the awaited variables of
$\atom_i$ are disjoint from the controlled variables of~$\atom_j$.
The asymmetry of $\awaits_\ro$ ensures that there exists a consistent order
of the atoms in~$\roatoms{\ro}$.

\mypar
{\bf Module execution.}
When executing the module~$\ro$, in each round, first the external variables
are assigned arbitrary values of the correct types, and then the atoms in
$\roatoms{\ro}$ are executed in an arbitrary consistent order
$\atom_0,\ldots,\atom_{k-1}$.
Specifically, in the initialization round, after the external variables have
been initialized, the initial action of $\atom_0$ is followed by the initial
action of $\atom_1$ etc.;
and in each update round, after the external variables have been updated, the
update action of $\atom_0$ is followed by the update action of $\atom_1$ etc.
In this manner, all awaited values are available when they are needed during
the execution, and thus every round can be completed in $k+1$ subrounds
---one subround for the environment followed by one subround for each atom.
Moreover, all nondeterminism in the completion of a round is caused by the
nondeterminism of the environment and by the nondeterminism of individual
atoms, not by the order of the atoms.


\section{Examples of Reactive Modules}

The syntax we use for specifying modules will be comprehensible once a few
conventions are explained.
Variable declarations are indicated by keywords such as \AWAITS, for the
awaited variables of an atom, or \PRIV, for the private variables of a
module.
The initial and update actions of atoms are specified by the keywords \INIT\
and \UPDATE, followed by nondeterministic guarded commands.
The combined keyword \INIT\ \UPDATE\ indicates that the guarded command that
follows specifies both the initial and update actions.
If several guards of a guarded command are true, then one of the
corresponding assignments is chosen nondeterministically;
if none of the guards are true, then all controlled variables obtain their
default values.
In the initialization round, the default initial value of a variable is
chosen nondeterministically
(this is legal only if the variable has finite type).
In each update round, the default updated value of a variable is equal to the
latched value of the variable
(i.e., the value of the variable stays unchanged).
The default values are also invoked if a controlled variable does not appear
on the left-hand-side of an assignment, or if the initial command is omitted
altogether.

\begin{figure}
\begin{mtab}{l}
  \MODULE\ \SNot\\
  \qu \EXTL\ \iin : \bool\\
  \qu \INTF\ \oout : \bool\\
  \qu \ATOM\ \CONTROLS \oout\ \AWAITS\ \iin\\
  \qqu \INIT\ \UPDATE\\
  \qqu \begin{chtab}
    \iin'=0 &\oout':=1\\
    \iin'=1 &\oout':=0
  \end{chtab}\\
  \qu \ENDA \\
  \qu \ENDM \\\\

  \MODULE\ \SAnd \\
  \qu \EXTL\ \iino,\iint : \bool \\
  \qu \INTF\ \oout : \bool \\
  \qu \ATOM\ \CONTROLS \oout\ \AWAITS\ \iino,\iint \\
  \qqu \INIT\ \UPDATE\\
  \qqu \begin{chtab}
    \iino'=0 &\oout':=0\\
    \iint'=0 &\oout':=0\\
    \iino'=1\AND\iint'=1 &\oout':=1
  \end{chtab}\\
  \qu \ENDA \\
  \qu \ENDM \\\\

  \MODULE\ \SLatch\\
  \qu \EXTL\ \iin, \reset : \bool\\
  \qu \INTF\ \oout : \bool\\
  \qu \PRIV\ \sstate:\bool\\
  \qu \ATOM\ \CONTROLS \oout\ \READS\ \sstate\\
  \qqu \UPDATE\\
  \qqu \begin{chtab}
    \true &\oout':=\sstate
  \end{chtab}\\
  \qu \ENDA \\

  \qu \ATOM\ \CONTROLS \sstate\ \AWAITS\ \iin,\reset\\
  \qqu \INIT\ \UPDATE\\
  \qqu \begin{chtab}
    \reset' & \sstate':= \false \\
    \NOT\reset' & \sstate':=\iin'\\
  \end{chtab}\\
  \qu \ENDA \\
  \qu \ENDM \\
\end{mtab}
\caption{Synchronous \Not\ gate, \And\ gate, and latch}
\label{fig:syncand}
\end{figure}


\subsection{Synchronous circuits}

Synchronous circuits are built from logic gates and memory cells that are
driven by a sequence of clock ticks.
Each logic gate computes a boolean value once per clock cycle, and each
memory cell stores a boolean value from one clock cycle to the next.
We model each logic gate and each memory cell as a reactive module so that
every update round represents a clock cycle.
All sequential circuits can be constructed from the building blocks shown in
Figure~\ref{fig:syncand}.

\mypar The module $\SNot$ models a synchronous \Not\ gate, which
takes a boolean input and produces a boolean output. The input is
modelled as an external variable, $\iin$, because it is modified
by the environment and visible to the gate. The output is modelled
as an interface variable, $\oout$, because it is modified by the
gate and visible to the environment. In the initialization round,
the \Not\ gate waits for the input value to be initialized before
computing the initial output value, by negating the initial input
value. In each update round, the \Not\ gate waits for the input
value to be updated before computing the next output value, by
negating the updated input value.

\mypar
The module $\SAnd$ models a synchronous \And\ gate that produces the boolean
output $\oout$ as a function of the two boolean inputs $\iino$ and $\iint$.
In each round, the interface variable $\oout$ is initialized or updated after
both external variables $\iino$ and $\iint$ have been initialized or
updated.

\mypar
The synchronous latch $\SLatch$ takes the two boolean inputs $\sset$ and
$\reset$, produces the boolean output $\oout$, and maintains the private bit
$\sstate$.
In each update round, the latch copies its state to the interface variable
$\oout$, without waiting for the updated values of the external variables.
In a later subround, after both external variables have been updated, the
latch updates its state:
if both updated external variables are low, then $\sstate$ stays unchanged;
if only $\sset$ is high, then $\sstate$ goes to~1;
if only $\reset$ is high, then $\sstate$ goes to~0;
if both are high, then $\sstate$ goes to an arbitrary value.
In the initialization round, the output of the latch is arbitrary, and the
state of the latch is initialized after both inputs have been initialized.
If both inputs are initially low, then the initial state of the latch is
arbitrary, but equal to the initial output.

\mypar
{\bf History-free variables in verification.}
Given a module~$\ro$, a variable $\var$ of $\ro$ is {\em history-free\/} if
$\var$ is not read by any atom of~$\ro$.
Then, the update commands of $\ro$ can refer only to the updated value
$\var'$ and not to the latched value~$\var$.
In synchronous circuits, all variables that represent wires are history-free.
Specifically, all variables of Figure~\ref{fig:syncand} except for the latch
state, $\sstate$, are history-free.
In each round, the possible updated values of a history-free variable $\var$
depend only on the latched values of variables that are not history-free, and
on the updated values of variables other than~$\var$.
In this way, history-free variables are analogous to the combinational
variables of hardware description languages, the selection variables of
{\sc Cospan}~\cite{Kurshan94}, and the pointwise functions of dataflow
languages.
During the verification of a module by explicit search through the state
space, the values of history-free variables are omitted from the search
stack.
Similarly, during the symbolic verification of a module, the history-free
variables are eliminated using existential quantification in each
image-computation step.
Hence, unlike other variables, extra history-free variables cause little
extra cost in verification.


\begin{figure}
\begin{mtab}{l}
  \MODULE\ P1\\
  \qu \INTF\ \statuso : \set{\outcs,\reqcs,\incs} ;  x1 : \bool\\
  \qu \EXTL\ \statust : \set{\outcs,\reqcs,\incs} ;  x2 : \bool\\
  \qu \ATOM\ \CONTROLS \statuso,x1\ \READS\ \statuso,\statust,x1,x2\\
  \qqu \INIT\\
  \qqu \begin{chtab}
    \true &\statuso':=\outcs
  \end{chtab} \\
  \qqu \UPDATE\\
  \qqu \begin{chtab}
     \statuso=\outcs & \statuso':=\reqcs ;  x1':=x2\\
     \statuso=\reqcs\AND(\statust=\outcs \OR \NOT(x1=x2))&\
     \statuso':=\incs\\
     \statuso=\incs &\statuso':=\outcs\\
     \true &
  \end{chtab} \\
  \qu \ENDA \\
  \qu \ENDM \\\\

  \MODULE\ P2\\
  \qu \INTF\ \statust : \set{\outcs,\reqcs,\incs} ;  x2 : \bool\\
  \qu \EXTL\ \statuso : \set{\outcs,\reqcs,\incs} ;  x1 : \bool\\
  \qu \ATOM\ \CONTROLS \statust,x2\ \READS\ \statuso,\statust,x1,x2\\
  \qqu \INIT\\
  \qqu \begin{chtab}
    \true &\statust':=\outcs
  \end{chtab}\\
  \qqu \UPDATE\\
  \qqu \begin{chtab}
    \statust=\outcs &\statust':=\reqcs ;  x2':=\NOT x1\\
    \statust=\reqcs\AND(\statuso=\outcs\OR x1=x2)&\
    \statust':=\incs\\
    \statust=\incs &\statust':=\outcs\\
    \true &
  \end{chtab} \\
  \qu \ENDA \\
  \qu \ENDM \\\\

\end{mtab}
\caption{Asynchronous mutual-exclusion protocol}
\label{fig:pete}
\end{figure}

\subsection{Asynchronous shared-memory programs}

As an example of a concurrent program consisting of processes that
communicate through read-shared variables, we consider a mutual-exclusion
protocol, which ensures that no two processes simultaneously access a common
resource.
The modules $P_1$ and $P_2$ of Figure~\ref{fig:pete} model the two processes
of Peterson's solution to the mutual-exclusion problem for shared variables.
Each process  has a program counter ($\statuso,~\statust$) and a flag ($x1,~x2$), both
of which can be observed by the other process.
The program counter indicates whether a process is outside its critical
section ($\outcs$), requesting the critical section
($\reqcs$), or occupying the critical section ($\incs$).
In each update round, a process looks at the latched values of all variables
and, nondeterministically, either updates its controlled variables or sleeps
(i.e., leaves the controlled variables unchanged), without waiting to see
what the other process does.
Note that each process may sleep for arbitrarily many rounds:
nondeterminism is used to ensure that there is no relationship between the
execution speeds of the two processes.

\mypar {\bf Interleaving.} Unlike in interleaving models, both
processes may modify their variables in the same round. While
Peterson's protocol ensures mutual exclusion even under these
weaker conditions, if one were to insist on the interleaving
assumption, one would add a third module that, in each update
round, nondeterministically schedules either or none of the two
processes. The modelling of interleaving by a scheduler module
introduces only history-free variables, and thus, does not
increase the search space during verification. Alternatively, one
could describe the complete protocol as a single module containing
a single atom whose update action is the union of the update
actions of the atoms of Figure~\ref{fig:pete}. The guarded command
that specifies a union of actions consists simply of the union of
all guarded assignments of the individual actions. This style of
describing asynchronous programs as an unstructured collection of
guarded assignments is pursued in formalisms such as {\sc
Unity}~\cite{ChandyMisra88} and {\sc Mur}$\varphi$~\cite{}.

\mypar {\bf Write-shared variables.} The original formulation of
Peterson's protocol uses a single write-shared boolean
variable~$x$, whose value always corresponds to the value of the
predicate $x1=x2$ in our formulation. If one were to insist on
modelling $x$ as a write-shared variable, one would add a third
module with the interface variable $x$ and awaited external
variables such as {\em P$_i$\_sets\_x\_to\_}0, which is a boolean
interface variable of the $i$-th process that indicates when the
process wants to set $x$ to~0. Since all of these variables with
the exception of $x$ are history-free, the modelling does not
increase the search space during verification. This style of
describing write-shared memory makes explicit what happens when
several processes write simultaneously to the same location.


\begin{figure}
\begin{mtab}{l}
 \TYPE \msgType : \bool \\
  \MODULE\ \SSender\\
  \qu \EXTL\ \ready : \event\\
  \qu \INTF\ \transmit : \event ;  \msgS,\msgP : \msgType\\
  \qu \PRIV\ \status : \set{\produce,\send} ;  \doneP : \event\\
  \qu \ATOM\ \CONTROLS \status,\transmit,\msgS\
    \READS\ \status,\transmit,\msgS,\doneP,\msgP,\ready\
    \AWAITS\ \doneP,\ready\\
  \qqu \INIT\\
  \qqu \begin{chtab}
    \true &\status':=\produce
  \end{chtab}\\
  \qqu \UPDATE\\
  \qqu \begin{chtab}
    \status=\produce\AND\doneP? &\status':=\send\\
    \status=\send\AND\ready? &\
     \transmit! ;  \msgS':=\msgP ;  \status':=\produce
  \end{chtab}\\
  \qu \ENDA \\
  \qu \ATOM\ \AProd~ \CONTROLS \doneP,\msgP\ \READS\ \status,\doneP,\msgP\\
  \qqu \UPDATE\\
  \qqu \begin{chtab}
    \status=\produce &\doneP! ;  \msgP':=\msgType\\
    \true & \\
  \end{chtab}\\
  \qu \ENDA \\
  \qu \ENDM \\\\

  \MODULE\ \SReceiver\\
  \qu \EXTL\ \transmit : \event ;  \msgS : \msgType\\
  \qu \INTF\ \ready : \event ;  \msgC : \msgType\\
  \qu \PRIV\ \status : \set{\receive,\consume} ;  \doneC : \event ;
     \msgR : \msgType\\
  \qu \ATOM\ \CONTROLS \status, \msgR\
    \READS\ \status,\transmit,\doneC\
    \AWAITS\ \transmit,\msgS,\doneC\\
  \qqu \INIT\\
  \qqu \begin{chtab}
    \true &\status':=\receive
  \end{chtab}\\
  \qqu \UPDATE\\
  \qqu \begin{chtab}
    \status=\receive\AND\transmit? &\
      \msgR':=\msgS' ;  \status':=\consume\\
    \status=\consume\AND\doneC? &\status':=\receive
  \end{chtab}\\
  \qu \ENDA \\
  \qu \ENDM \\\\

  \qu \ATOM\ \CONTROLS \ready\ \READS\ \status,\ready\\
  \qqu \UPDATE\\
  \qqu \begin{chtab}
    \status=\receive &\ready!\\
    \true &
  \end{chtab}\\
  \qu \ENDA \\
  \qu  \ATOM\ \ACons~ \CONTROLS \doneC,\msgC\ \READS\ \status,\doneC,\msgR\\
  \qqu \UPDATE\\
  \qqu \begin{chtab}
    \status=\consume &\doneC!; \msgC' := \msgR\\
    \true &
  \end{chtab}\\
  \qu \ENDA \\
  \qu \ENDM \\
\end{mtab}
\caption{Synchronous message-passing protocol}
\label{fig:syncmsg}
\end{figure}

\subsection{Synchronous message-passing protocols}

The modules $\SSender$ and $\SReceiver$ of
Figure~\ref{fig:syncmsg} communicate via events in order to
transmit a stream of messages. We write $\var : \event$ to declare
$\var$ to be a boolean variable that is used for modelling events.
To issue an event represented by~$\var$, we write $\var!$, which
stands for the assignment $\var':=\NOT\var$. To check if an event
represented by $\var$ is present, we write $\var?$, which stands
for the predicate $\var'\not=\var$.

\mypar
The private variable $\status$ of the sender indicates if it is producing a
message ($\status=\produce$), or attempting to send a message
($\status=\send$).
The private variable $\status$ of the receiver indicates if it is waiting to
receive a message ($\status=\receive$), or consuming a message
($\status=\consume$).
Messages are produced by the atom $\AProd$, which requires an unknown number
of rounds to produce a message.
Once a message is produced, the event $\doneP$ is issued, and the message is
shown as~$\msgP$
(the actual value of message is chosen nondeterministically from the finite
type $\msgType$).
Once a message has been produced, the sender is ready to send the message,
and $\status$ is updated.
When ready to send a message, the sender sleeps until the receiver becomes
ready to receive, and when ready to receive a message, the receiver sleeps
until the sender transmits a message.

\mypar
The synchronization of both agents is achieved by two-way handshaking in
three subrounds within a single update round.
The first subround belongs to the receiver.
If the receiver is ready to receive a message, it issues the interface event
$\ready$ to signal its readiness to the sender.
The second subround belongs to the sender.
If the sender sees the external event $\ready$ and is ready to send a
message, it issues the interface event $\transmit$ to signal a transmission.
The third subround belongs to the receiver.
If the receiver sees the external event $\transmit$, it copies the message
from the external variable $\msgS$ to the private variable~$\msgR$.
The sender goes on to wait for the production of another message, and the
receiver goes on to consume~$\msgR$.
Messages are consumed by the atom $\ACons$, which requires an unknown number
of rounds to consume a message.
Once a message is consumed, the event $\doneC$ is issued, the consumed
message is shown as~$\msgC$, and the receiver waits to receive another
message.

\mypar
{\bf Event variables in verification.}
Like history-free variables, event variables are also used only for
interaction within a round, and their actual values at the beginning of a
round are immaterial.
In a sense, the values of event variables behave like labels on the
transitions of a state-transition graph, unlike the values of other
variables, which behave like labels on the states.
Consequently, during explicit verification, the values of event variables are
omitted from the search stack, and during symbolic verification, event
variables are eliminated using existential quantification.

\subsection{ A train controller}

{\small

\begin{figure}
\begin{mtab}{l}
  \MODULE\ \Train\\
  \qu \INTF\ \pc : \set{\away,\wait,\bridge}\;; \arrive,\leave : \event\\
  \qu \EXTL\ \signal : \set{\green,\red}\\
  \qu \LAZY\ \ATOM\ \CONTROLS \pc,\arrive,\leave\ \READS\ \pc,\arrive,\leave,\signal\\
  \qqu \INIT\\
  \qqu \begin{chtab}
    \true &\pc':=\away
  \end{chtab}\\
  \qqu \UPDATE\\
  \qqu \begin{chtab}
    \pc=\away &\arrive!;\; \pc':=\wait\\
    \pc=\wait\AND\signal=\green &\pc':=\bridge\\
    \pc=\bridge &\leave!;\; \pc':=\away
  \end{chtab} \\
  \qu \ENDA\\
  \ENDM\\\\
    \TrainW =
    \Train[\pc,\arrive,\signal,\leave:=
    \pcW,\arriveW,\signalW,\leaveW]\\
    \TrainE =
    \Train[\pc,\arrive,\signal,\leave:=
    \pcE,\arriveE,\signalE,\leaveE]\\\\
  \MODULE\ \Controller\\
  \qu \PRIV\ \nearW,\nearE : \bool\\
  \qu \INTF\ \signalW,\signalE : \set{\green,\red}\\
  \qu \EXTL\ \arriveW,\arriveE,\leaveW,\leaveE : \event\\
  \qu \ATOM\
    \CONTROLS \nearW\
    \READS\ \nearW,\arriveW,\leaveW\
    \AWAITS\ \arriveW,\leaveW\\
  \qqu \INIT\\
  \qqu \begin{chtab}
    \true &\nearW':=\false
  \end{chtab}\\
  \qqu \UPDATE\\
  \qqu \begin{chtab}
    \arriveW? &\nearW':=\true\\
    \leaveW? &\nearW':=\false
  \end{chtab}\\
  \qu \ENDA\\
  \qu \ATOM\
    \CONTROLS \nearE\
    \READS\ \nearE,\arriveE,\leaveE\
    \AWAITS\ \arriveE,\leaveE\\
  \qqu \INIT\\
  \qqu \begin{chtab}
    \true &\nearE':=\false
  \end{chtab}\\
  \qqu \UPDATE\\
  \qqu \begin{chtab}
    \arriveE? &\nearE':=\true\\
    \leaveE? &\nearE':=\false
  \end{chtab}\\
  \qu \ENDA\\
  \qu \LAZY\ \ATOM\
    \CONTROLS \signalW,\signalE\
    \READS\ \nearW,\nearE,\signalW,\signalE\\
  \qqu \INIT\\
  \qqu \begin{chtab}
    \true &\signalW':=\red;\; \signalE':=\red
  \end{chtab}\\
  \qqu \UPDATE\\
  \qqu \begin{chtab}
    \nearW\AND\signalE=\red &\signalW':=\green\\
    \nearE\AND\signalW=\red &\signalE':=\green\\
    \NOT\nearW &\signalW':=\red\\
    \NOT\nearE &\signalE':=\red
  \end{chtab}\\
  \qu \ENDA\\
  \ENDM\\\\
  \TrainSystem = \HIDE\ \arriveW,\arriveE,\leaveW,\leaveE\ \IN
  \TrainW
  \pppar\TrainE
  \pppar\Controller
\end{mtab}

\caption{Railroad controller}
\label{fig:rail}
\end{figure}

}

Consider a railway system with two circular railroad tracks, one
for the train travelling clockwise, and the other for the train
travelling counter-clockwise.  At one point of the circle, there
is a bridge that is not wide enough to accommodate both tracks.
The two tracks merge on the bridge, and for controlling the access
to the bridge, there is a signal at either entrance.  If the
signal at the western entrance is green, then a train coming from
the west may enter the bridge; if the signal is red, the train
must wait.  The signal at the eastern entrance to the bridge
controls trains coming from the east in the same fashion.

\mypar Figure~\ref{fig:rail} shows the Reactive Modules
description for the a generic train {\it Train}. The module has
three interface variables: {\it pc} of enumerative type (\{$\away,
\wait, \bridge$\}) and {\it arrive, leave} of type event
($\event$); and one external variable {\it signal}. An event
variable $x$ is a boolean variable with restricted operations:
$x!$ stands for the assignment $x':=\NOT x$ (emission of the
event) and $x?$ stands for the condition $x'\neq x$ (checking for
presence). The module has only one atom which controls all the
interface variables. The keyword {\bf lazy} indicates that the
atom may choose not to update the variables, in which case the
variables retain their old values. This is useful to model the
assumption regarding independence of the speeds of different
modules. The keyword {\bf reads} indicates the old values of the
read variables ({\it pc, arrive, leave, signal}) are used for
updating the controlled variables.

\mypar The module does the following: when the train approaches
the bridge, it sends the event $\arrive$ to the railroad
controller and checks the signal at the entrance to the bridge
($\pc=\wait$).  When the signal is red, the train stops and keeps
checking the signal.  When the signal is green, the train proceeds
onto the bridge ($\pc=\bridge$). When the train exits from the
bridge, it sends the event $\leave$ to the controller and travels
around the circular track ($\pc=\away$). Multiple copies of the
{\it Train} are created by variable renaming. $\TrainW$, which
represents the train travelling clockwise, is constructed by
renaming variables $\pc$ to $\pcW$, $\arrive$ to $\arriveW$,
$\signal$ to $\signalW$ and $\leave$ to $\leaveW$. $\TrainE$,
which represents the train travelling counter-clockwise, is
constructed in a similar fashion.

\mypar
Figure~\ref{fig:rail} also shows a controller controlling the signals to
prevent collisions of the two trains. The complete railway system is represented by the module {\it Rail},
which is the composition of the trains with the controller, with
variables $\arriveW, \arriveE, \leaveW$ and $\leaveE$ hidden.

\subsection{ A resource manager}

\begin{figure}
\label{fig:rmanager}
\begin{center}
\resizebox{!}{35ex}{type indexType:  (0..3)
type regType: array indexType of bool

-------------------------------------------------------------------------------------
--
--   SPECIFICATION
--
-------------------------------------------------------------------------------------
{\small

\begin{figure}
\begin{mtab}{l}

\MODULE \Spec\\
\qu \INTF \grant, \half_empty:\bool; \grant_index:indexType; \alloc:regType \\
\qu \EXTL \req, \free, \high_priority:\bool; \free_index:indexType \\
  
\qu \ATOM \ALLOC \CONTROLS \alloc 
	\READS  \alloc 
	\AWAITS \req, \grant, \grant_index, \free, \free_index \\
\qqu \INIT\\
\qqu \begin{chtab}
	\true -> \forall i \alloc'[i] := \false
     \end{chtab} \\
\qqu \UPDATE \\
\qqu \begin{chtab}
	\true -> \forall i 
		   \alloc'[i] := 
		     \if (\grant' & \grant_index' = i) \then \true
		     \else \if (\free' & \free_index' = i) \then \false
		     \else \alloc[i] \fi \fi 
     \end{chtab} \\
\qu \ENDA \\
\qu \ATOM \HALF_EMPTY \CONTROLS \half_empty
	\READS \alloc \\
\qqu \INIT \\
\qqu \begin{chtab}
	\true -> \half_empty' := \true
     \end{chtab} \\
\qqu \update \\
\qqu \begin{chtab}
	~\alloc[0]  -> \half_empty' := \if (~\alloc[1] | ~\alloc[2] | ~\alloc[3]) \then \true
					 \else \false \fi \\
 	~\alloc[1]  -> \half_empty' := \if (~\alloc[0] | ~\alloc[2] | ~\alloc[3]) \then \true
					 \else \false \fi \\
	~\alloc[2]  -> \half_empty' := \if (~\alloc[0] | ~\alloc[1] | ~\alloc[3]) \then \true
					 \else \false \fi \\
	~\alloc[3]  -> \half_empty' := \if (~\alloc[0] | ~\alloc[1] | ~\alloc[2]) \then \true
					 \else \false \fi \\
	\default    -> \half_empty' := \false
     \end{chtab} \\
\qu \ENDA \\
\qu \ATOM \GRANT_INDEX \CONTROLS \grant_index \\
\qqu \INIT \UPDATE \\
\qqu \begin{chtab}
	true -> grant_index' := nondet
\end{chtab} \\
\qu \ENDA \\
\qu \ATOM \GRANT \CONTROLS \grant
       \READS \alloc
       \AWAITS \req, \high_priority, \half_empty, \grant_index \\
\qqu \INIT \\
\qqu \begin{chtab}
	true -> grant' := false
\end{chtab} \\
\qqu \UPDATE \\
\qqu \begin{chtab} 
	\req' & \high_priority' & ~\alloc[\grant_index'] -> \grant' := \true \\
	\req'  & \half_empty'    & ~\alloc[\grant_index'] -> \grant' :=
	\true \\
	\default -> \grant' := \false
\end{chtab} \\
\qu \ENDA \\
\ENDM \\

\MODULE \Impl \\
\qu \EXTL  \req, \free, \high_priority:\bool; \free_index:\indexType \\
\qu \INTF \grant,\half_empty:bool; grant_index:indexType;
	alloc:regType \\
\qu \PRIV \sum: \sindexType \\
\qu \ATOM \ALLOC \CONTROLS \alloc 
	\READS  \alloc 
	\AWAITS \req, \grant, \grant_index, \free, \free_index \\
\qqu \INIT\\
\qqu \begin{chtab}
	\true -> \forall i \alloc'[i] := \false
     \end{chtab} \\
\qqu \UPDATE \\
\qqu \begin{chtab}
	\true -> \forall i 
		   \alloc'[i] := 
		     \if (\grant' & \grant_index' = i) \then \true
		     \else \if (\free' & \free_index' = i) \then \false
		     \else \alloc[i] \fi \fi 
     \end{chtab} \\
\qu \ENDA \\
\qu \ATOM \SUM \CONTROLS \sum
  	\READS \sum, \alloc
	\AWAITS \grant, \free, \free_index \\
	
\qqu \INIT \\
\qqu \begin{chtab}
	true -> sum' := 0 
\end{chtab}
\qqu \UPDATE \\
\qqu \begin{chtab}
	\grant' & ~(\free' & \alloc[\free_index']) -> \sum' := \sum + 1	
	~\grant' & (\free' & \alloc[\free_index']) -> \sum' := \sum - 1
\end{chtab}
\qu \ENDA
\qu \ATOM \HALF_EMPTY \CONTROLS \half_empty
	\READS \sum \\
\qqu \INIT \\
\qqu \begin{chtab}
	-- complete this
\end{chtab} \\
\qqu \UPDATE \\
\qqu \begin{chtab}
	-- complete this
\end{chtab} \\
\qu \ENDA \\
\qu \ATOM \GRANT_INDEX \CONTROLS \grant_index
	\READS \alloc \\
\qqu \INIT \\
\qqu \begin{chtab}
	\true -> \grant_index' := 0
\end{chtab} \\
\qqu \UPDATE \\
\qqu \begin{chtab}
	~\alloc[0] -> \grant_index' := 0
	\alloc[0] & ~\alloc[1] -> \grant_index' := 1
	\alloc[0] &  \alloc[1] & ~\alloc[2] -> \grant_index' := 2
	\alloc[0] &  \alloc[1] &  \alloc[2] -> \grant_index' := 3
\end{chtab} \\
\qu \ENDA \\
\qu \ATOM \GRANT \CONTROLS \grant
       \READS \alloc, \sum
       \AWAITS \req, \high_priority \\
\qqu \INIT \\
\qqu \begin{chtab}
	-- complete this 
\end{chtab} \\
\qqu \UPDATE \\
\qqu \begin{chtab}
	-- complete this
\end{chtab} \\
\qu \ENDA \\
\ENDM

\end{mtab}

\caption{Resource manager}
\label{fig:rmanager}
\end{figure}

}







 



}
\caption{ Resource Manager}
\end{center}
\end{figure}

{\small

\begin{figure}
\begin{mtab}{l}
\TYPE \indexType :  (0..3) \\
\TYPE \regType : \ARRAY \indexType \OF \bool \\
\MODULE \Rmanager\\
\qu \INTF \grant\ : \bool;\; \grantindex:\indexType;\;  \\
\qu \EXTL \req, \free, \highpriority:\bool;\; \freeindex:\indexType \\
\qu \PRIV \halfempty\ :\bool;\;  \alloc:\regType \\

\qu \ATOM \ALLOC\; \CONTROLS \alloc\;
    \READS  \alloc\;
    \AWAITS \req, \grant, \grantindex, \free, \freeindex \\
\qqu \INIT\\
\qqu \begin{chtab}
    \true &  \FORALL  i\ \alloc'[i] := \false
     \end{chtab} \\
\qqu \UPDATE \\
\qqqu  \choice \true \limp  \FORALL i\  \alloc'[i] := \\
\qqqqu       \IF (\grant' \AND \grantindex' = i)\; \THEN \true\; \\
\qqqqu       \ELSE \IF (\free' \AND \freeindex' = i)\; \THEN \false\; \\
\qqqqu           \ELSE \alloc[i]\; \FI \FI \\
\qu \ENDA \\
\qu \ATOM \HALFEMPTY \CONTROLS \halfempty\;
    \READS \alloc \\
\qqu \INIT \\
\qqu \begin{chtab}
    \true & \halfempty' := \true
     \end{chtab} \\
\qqu \UPDATE \\
\qqu \begin{chtab}
    \NOT \alloc[0] & \halfempty' := \IF (\NOT ( \alloc[1] \AND \alloc[2] \AND \alloc[3]) ) \; \THEN \true\;
                     \ELSE \false\; \FI \\
    \NOT \alloc[1] & \halfempty' := \IF (\NOT ( \alloc[0] \AND \alloc[2] \AND \alloc[3]) )\; \THEN \true\;
                     \ELSE \false\; \FI \\
    \NOT \alloc[2]  & \halfempty':= \IF (\NOT ( \alloc[0] \AND \alloc[1] \AND \alloc[3]) )\; \THEN \true\;
                     \ELSE \false\; \FI \\
    \NOT \alloc[3]  & \halfempty' := \IF(\NOT ( \alloc[0] \AND \alloc[1] \AND \alloc[2]) )\; \THEN \true\;
                     \ELSE \false\; \FI \\
    \DEFAULT   & \halfempty' := \false
     \end{chtab} \\
\qu \ENDA \\
\qu \ATOM \GRANTINDEX\; \CONTROLS \grantindex \\
\qqu \INIT \UPDATE \\
\qqu \begin{chtab}
    true & grantindex' := nondet
\end{chtab} \\
\qu \ENDA \\
\qu \ATOM \GRANT\; \CONTROLS \grant
       \READS \alloc\;
       \AWAITS \req, \highpriority, \halfempty, \grantindex \\
\qqu \INIT \\
\qqu \begin{chtab}
    true & grant' := false
\end{chtab} \\
\qqu \UPDATE \\
\qqu \begin{chtab}
    \req' \AND \highpriority' \AND \NOT \alloc[\grantindex'] & \grant' := \true \\
    \req'  \AND \halfempty'    \AND \NOT \alloc[\grantindex'] & \grant' :=
    \true \\
    \DEFAULT & \grant' := \false
\end{chtab} \\
\qu \ENDA \\
\ENDM \\\\
\end{mtab}

\caption{Resource Manager (Specification)}
\label{fig:rmanagerspec}
\end{figure}
}


Consider a resource manager used to resolve accesses to instances of a
shared resource. A block diagram of a resource manager is given in
Figure~\ref{fig:rmanager}.  There are four instances of the resource
available.  The environment interacts with the resource manager by
allocate and free requests. If a resource is available, the resource
manager responds by granting an instance of the resource.


\mypar Figure~\ref{fig:rmanagerspec} shows a reactive modules
description of the resource manager.  The module $\Rmanager$
receives allocate and free requests through the boolean external
variables $\req$ and $\free$ respectively.  The environment also
sets the external variable $\highpriority$ to $\true$ in the case
of a high-priority request. For a normal-prioroty request,
$\highpriority$ is set to $\false$. The manager may grant a
high-priority request, even if only one instance of the resource
is available. A normal-priority request, on the other hand may be
granted only if at least two instances of the resource are
available. The module $\Rmanager$ signals grants to the
environment through the interface variable $\grant$ of type
$\bool$. The interface variable $\grantindex$ is set to the index
of the resource that is granted. While freeing the resource, the
environment is expected to set the external variable $\freeindex$
to the appropriate index of the resource that is being freed.

\mypar
Private variable \alloc (a boolean array) is used to maintain a bit
for each resource to indicate if it is free or allocated. The
history-free private variable $\halfempty$ of type $\bool$ is
set to true if at least two instances of the resource are free.
It is used in handling normal-priority requests.


{\small

\begin{figure}
\begin{mtab}{l}

\MODULE \RmanagerImpl \\
\qu \EXTL  \req, \free, \highpriority:\bool;\; \freeindex:\indexType \\
\qu \INTF \grant,\halfempty:bool;\; \grantindex:indexType;
    alloc:regType \\
\qu \PRIV \halfempty\:\bool;\;  \alloc:\regType;\; \total\:\sindexType \\
\qu \ATOM \ALLOC\; \CONTROLS \alloc\;
    \READS  \alloc
    \AWAITS \req, \grant, \grantindex, \free, \freeindex \\
\qqu \INIT\\
\qqu \begin{chtab}
    \true & \FORALL i\ \alloc'[i] := \false
     \end{chtab} \\
\qqu \UPDATE \\
\qqu \UPDATE \\
\qqqu  \choice \true \limp  \FORALL i\  \alloc'[i] := \\
\qqqqu       \IF (\grant' \AND \grantindex' = i)\; \THEN \true\; \\
\qqqqu       \ELSE \IF (\free' \AND \freeindex' = i)\; \THEN \false\; \\
\qqqqu           \ELSE \alloc[i]\; \FI \FI \\
\qu \ENDA \\
\qu \ATOM \TOTAL\; \CONTROLS \total\;
    \READS \total, \alloc\;
    \AWAITS \grant, \free, \freeindex \\
\qqu \INIT \\
\qqu \begin{chtab}
    \true & \total' := 0
\end{chtab}
\qqu \UPDATE \\
\qqu \begin{chtab}
  \grant' \AND \NOT (\free' \AND \alloc[\freeindex']) & \total' := \total + 1 \\
  \NOT \grant' \AND (\free' \AND \alloc[\freeindex']) & \total' := \total - 1
\end{chtab}\\
\qu \ENDA \\
\qu \ATOM \HALFEMPTY\; \CONTROLS \halfempty\;
    \READS \total \\
\qqu \INIT \\
\qqu \begin{chtab}
    \true  & \halfempty' := \true \\
\end{chtab} \\
\qqu \UPDATE \\
\qqu \begin{chtab}
    \total > 2   &  \halfempty' := \false \\
    \DEFAULT     &  \halfempty' := \true
\end{chtab}\\
\qu \ENDA\\

\qu \ATOM \GRANTINDEX\; \CONTROLS \grantindex\;
    \READS \alloc \\
\qqu \INIT \\
\qqu \begin{chtab}
    \true & \grantindex' := 0
\end{chtab} \\
\qqu \UPDATE \\
\qqu \begin{chtab}
    \NOT\alloc[0] & \grantindex' := 0 \\
    \alloc[0] \AND \NOT \alloc[1] &  \grantindex' := 1 \\
    \alloc[0] \AND  \alloc[1] \AND \NOT \alloc[2] &  \grantindex' := 2 \\
    \alloc[0] \AND  \alloc[1] \AND  \alloc[2] & \grantindex' := 3
\end{chtab}\\
\qu \ENDA\\
\qu \ATOM \GRANT\; \CONTROLS \grant\;
    \READS \alloc , \total\;
    \AWAITS \req, \highpriority \\
\qqu \INIT \\
\qqu \begin{chtab}
    \true & \grant' := \false \\
\end{chtab} \\
\qqu \UPDATE \\
\qqu \begin{chtab}
    \req' \AND \highpriority' \AND \total = 3 & \grant' := \true \\
     \req'  \AND  \total <= 2 & \grant' := \true\\
     \DEFAULT & \grant' := \false
\end{chtab} \\
\qu \ENDA \\
\ENDM
\end{mtab}

\caption{Resource Manager (Implementation)}
\label{fig:rmanagerimpl}
\end{figure}
}

\mypar
Module $\RmanagerImpl$ in Figure~\ref{fig:rmanagerimpl} is a
``less-abstract'' lower-level description of the resource allocator,
First, note that the modules $\Rmanager$ and $\RmanagerImpl$ have same
set of external and interface variables.
However, some design decisions have been taken to implement the
non-deterministic choices allowed in $\Rmanager$.
The $\grantindex$ variable always points to the available resource
with the least index, if one is available (as opposed to the
non-deterministic choice in $\Rmanager$) and there is a new private
variable $\total$ that keeps track of the number of resources that
have been allocated. We will later use \mocha to show that
$\RmanagerImpl$ is a valid implementation of the module $\Rmanager$.


\section{Semantics of Reactive Modules}

The execution of a module results in a trace of observations.
Reactive modules are related via a trace semantics:
roughly speaking, one module {\em implements\/} (or {\em refines\/}) another
module if all possible traces of the former, more detailed module are also
possible traces of the latter, more abstract module.


\subsection{The trace language of a module}

Let $\ro$ be a reactive module.
As indicated earlier, a state of $\ro$ is a valuation for the set
$\rovars{\ro}$ of module variables.
We write $\states_\ro$ for the set of states of~$\ro$.

\mypar
A state $\state$ of the module $\ro$ is {\em initial\/} if it can be obtained
by executing all initial actions of $\ro$ in a consistent order.
We write $\init\ro$ for the set of initial states of the module~$\ro$.
The set $\init\ro$ is nonempty, because all initial actions are executable.
For two states $\state$ and $\astate$ of~$\ro$, the state $\astate$ is a
{\em successor\/} of~$\state$, written $\state\trans_\ro\astate$, if
$\astate$ can be obtained from $\state$ by executing all update actions of
$\ro$ in a consistent order.
The binary relation $\trans_\ro$ over the state space $\states_\ro$ is called
the {\em transition relation\/} of the module~$\ro$.
The transition relation $\trans_\ro$ is serial
(i.e., every state has at least one successor), because all update actions
are executable.
Moreover, a module does not constrain the behavior of the external variables
and interacts with its environment in a nonblocking way.


\mypar
In this way, the module $\ro$ defines a state-transition graph with the state
space~$\states_\ro$, the initial states $\init\ro$, and the transition
relation~$\trans_\ro$.
The initialized paths of this graph are called the trajectories of the
module:
a {\em trajectory\/} of $\ro$ is a finite sequence $\state_0\ldots\state_n$
of states of $\ro$ such that
(1)~the first state $\state_0$ is initial and
(2)~for all $0\le i<n$, the state $\state_{i+1}$ is a successor
  of~$\state_i$.
If $\stateseq=\state_0\ldots\state_n$ is a trajectory of~$\ro$, then the
corresponding sequence
$\stateseq[\ovars{\ro}]=\state_0[\ovars{\ro}]\ldots\state_n[\ovars{\ro}]$ of
observations is called a {\em trace\/} of~$\ro$.\footnote{
  If $\state$ is a state and $\vars$ is a set of variables in the domain
  of~$\state$, we write $\state[\vars]$ for the valuation for $\vars$ that
  assigns to each variable $\var\in\vars$ the value~$\state[\var]$.}
Thus, a trace records the sequence of observations that may result
from executing the module for finitely many steps. The {\em trace
language\/} of the module~$\ro$, denoted~$\lang{\ro}$, is the set
of traces of~$\ro$. By definition, every prefix of a trajectory is
also a trajectory, and hence, every prefix of a trace is also a
trace. Since the set of initial states is nonempty, and the
transition relation is serial, every trajectory of a module, and
hence also every trace, can be extended. It follows that a module
cannot deadlock. In modelling, therefore, a deadlock situation
must be represented by a special state with a single outgoing
transition back to itself.


\subsection{The implementation preorder between modules}

The semantics of the module $\ro$ consists of the trace
language~$\lang{\ro}$, as well as all information that is necessary for
describing the possible interactions of $\ro$ with the environment:
the set $\intf{\ro}$ of interface variables, the set $\extl{\ro}$ of external
variables, and the await dependencies
$\awaits_\ro\cap\,(\intf{\ro}\times\ovars{\ro})$ between interface variables
and observable variables
(there cannot be any await dependencies between external variables and other
variables).

\begin{definition}{implement}\it
  {\em [{\bf Implementation}]}
  The module $\ro$ {\em implements\/} the module~$\aro$, written
  $\ro\refines\aro$, if the following conditions are met:
  (1)~every interface variable of $\aro$ is an interface variable of~$\ro$;
  (2)~every external variable of $\aro$ is an observable variable of~$\ro$;
  (3)~for all observable variables $\var$ of ${\aro}$ and all interface
    variables $\avar$ of~${\aro}$, if $\avar\awaits_\aro\var$, then
    $\avar\awaits_\ro\var$;
    and
  (4)~if $\stateseq$ is a trace of~$\ro$, then the projection
    $\stateseq[\ovars{\aro}]$ is a trace of~$\aro$.
\end{definition}


\mypar
The first three conditions ensure that the compatibility constraints imposed
by $\ro$ on its environment are at least as strong as those imposed
by~$\aro$.
The fourth condition is conventional trace containment.
Intuitively, if $\ro\refines\aro$, then the module $\ro$ is
{\it as detailed as\/} the module~$\aro$:
the implementation $\ro$ has possibly more interface and external variables
than the specification~$\aro$;
some external variables of $\aro$ may be interface variables of~$\ro$, and
thus are more constrained in~$\ro$;
the implementation $\ro$ has possibly more await dependencies among its
observable variables than the specification~$\aro$;
and $\ro$ has possibly fewer traces than~$\aro$, and thus more constraints on
its execution.
It is easy to check that every module $\ro$ implements itself, and that if a
module $\ro$ implements another module $\aro$, which, in turn, implements a
third module $\bro$, then $\ro$ also implements~$\bro$.
Hence, the implementation relation $\refines$ is a preorder
(i.e., reflexive and transitive).

\mypar
We write $\ro\refeq\aro$ if $\ro$ implements $\aro$ and $\aro$
implements~$\ro$.
It follows that $\refeq$ is an equivalence relation on modules.
The {\em meaning\/} of a module $\ro$ is the $\refeq$-equivalence class
of~$\ro$.


\subsection{Special classes of atoms and modules}

{\bf Combinational vs.\ sequential atoms.}
An atom $\atom$ is {\em combinational\/} if it has
(1)~no read variables and
(2)~identical initial and update actions:
$\rvars{\atom}=\emptyset$, and $\init{\atom}=\update{\atom}$.
In each update round, the updated values of the controlled variables of a
combinational atom depend only on the updated values of other variables, and
not on any latched values.
Furthermore, a combinational atom cannot distinguish between the
initialization round and later rounds.
If an atom is not combinational, then it is called {\em sequential}.
For example, in Figure~\ref{fig:syncand}, the atoms of the modules $\SNot$
and $\SAnd$ and the atom that controls the variable $\sstate$ of the module
$\SLatch$ are combinational;
the atom that controls the variable $\oout$ of $\SLatch$ is sequential.
Note that a combinational atom may control some variables that are not
history-free, and an atom may be sequential despite controlling only
history-free variables.
The two cases apply to the two atoms of the module $\SLatch$.


\mypar
For further illustration of the difference between combinational and
sequential atoms, consider the following example.
Given two variables $x$ and $y$ of the same type, we want $x$ to duplicate
the behavior of~$y$.
The combinational atom
\begin{mtab}{l}
    \qu \ATOM \CCopy~\CONTROLS\ x\ \AWAITS y'\\
    \qqu \INIT \UPDATE\\
    \qqu \begin{chtab}
      \true &\ x':=y'
    \end{chtab}\\
    \qu \ENDA \\
\end{mtab}

copies $y$ into $x$ without delay, and ensures that both $x$ and $y$ have the
same value at the end of each round.
The sequential atom
  \begin{mtab}{l}
    \qu \ATOM\ \SCopy~ \CONTROLS\ x\ \READS\ y\\
    \qqu \UPDATE\\
    \qqu \begin{chtab}
      \true &\ x':=y
    \end{chtab} \\
    \qu \ENDA \\
\end{mtab}
copies $y$ into $x$ with a delay of one round.
In each update round, $x$ is assigned the value of~$y$ at the beginning of
the round
(the initial command is irrelevant for the purposes of this example).

\mypar
{\bf Lazy vs.\ eager atoms.}
An atom {\em sleeps\/} in an update round if the values of all controlled
variables stay unchanged.
An $\vars$-atom $\atom$ is {\em lazy\/} if it may sleep in every update
round.
A sufficient syntactic condition for laziness is the presence of the guarded
assignment ``$\true\ra$'' in the update command, which leaves the values of
all controlled variables unchanged.
For example, the atoms of the modules $P_1$ and $P_2$ from
Figure~\ref{fig:pete} are lazy.
Typically, lazy atoms are nondeterministic and sequential, with all
controlled variables being read in order to keep their values unchanged.

\mypar
If an atom is not lazy, then it is called {\em eager}.
Both atoms $\CCopy$ and $\SCopy$ are eager.
In the first case, all updates of $x$ follow immediately, within the same
round, the corresponding updates of~$y$;
in the second case, the updates of $x$ are delayed by exactly one round.
By contrast, the lazy atom
\begin{mtab}{l}
  \qu \ATOM\   \LCopy\ \CONTROLS\ x\ \READS\ x\ \AWAITS\ y'\\
  \qqu \INIT\ \UPDATE\\
  \qqu \begin{chtab}
    \true &\ x':=y'\\
    \true &
  \end{chtab} \\
  \qu \ENDA \\
\end{mtab}
copies $y$ into $x$ at arbitrary times.
In each update round, either the value of $x$ stays unchanged, or it is set
to the updated value of~$y$.
Consequently, some values of $y$ may not be copied into~$x$.

\mypar
{\bf Event-driven vs.\ round-driven atoms. }
In each update round, an atom can notice changes in the values of variables
only for variables that are both read and awaited.
For these variables, which are called {\em watched}, the atom can compare the
latched values with the updated values.
Therefore, each change in the value of a watched variable is an observable
event.
An $\vars$-atom $\atom$ is {\em event-driven\/} if it may sleep in every
update round in which no observable event occurs;
that is, the atom may sleep whenever the values of all watched variables stay
unchanged.
A sufficient syntactic condition for being event-driven is the presence of a
conjunct of the form $x?$ in each guard of the update command.
Also, every lazy atom is event-driven.
For example, all atoms of the modules $\SSender$ and $\SReceiver$ from
Figure~\ref{fig:syncmsg} are event-driven.

\mypar
While the progress of a lazy atom cannot be enforced at all, the progress of
an event-driven atom can be enforced by other atoms that modify watched
variables.
However, the progress of an event-driven atom cannot be enforced solely by
the expiration of rounds.
Hence, if an atom is not event-driven, then it is called {\em round-driven}.
Both atoms $\CCopy$ and $\SCopy$ are round-driven.
The behavior of $\CCopy$ can also be defined in an event-driven manner by
the atom
\begin{mtab}{l}
  \qu \ATOM\   \ECopy\ \CONTROLS\ x\ \READS\ x,y\ \AWAITS\ y'\\
  \qqu \INIT\\
  \qqu \begin{chtab}
    \true &\ x':=y'
  \end{chtab}\\
  \qqu \UPDATE\\
  \qqu \begin{chtab}
    y'\not=y &\ x':=y'
  \end{chtab} \\
  \qu \ENDA \\
\end{mtab}
because the value of $x$ needs to be modified only when the value of $y$
changes.
The event-driven specification of immediate copying, however, reads both $x$
and~$y$, and is no longer combinational
($y$~is read to check if the value of $y$ changes, and $x$ is read to keep
the value of $x$ unchanged).

\mypar
{\bf Asynchronous modules.}
A module {\em stutters\/} in an update round if the values of all interface
variables stay unchanged.
Asynchronous modules are defined so that they may stutter in every update
round.

\begin{definition}{async}\it
  {\em [{\bf Asynchrony}]}
  A module $\ro$ is {\em asynchronous\/} if all interface variables of $\ro$
  are controlled by lazy atoms.
\end{definition}

\mypar
It follows that the environment cannot enforce the observable progress of an
asynchronous module.
While an asynchronous module can privately record all changes in the values
of external variables, all updates of interface variables proceed at a speed
that is independent of the environment speed.
For example, the modules $P_1$ and $P_2$ from Figure~\ref{fig:pete} are
asynchronous.

\mypar
{\bf Round-insensitive modules.}
A module {\em sleeps\/} in an update round if the values of all controlled
variables stay unchanged, and the environment {\em stutters\/} in an update
round if the values of all external variables stay unchanged.
Round-insensitive modules are defined so that they may sleep in every update
round in which the environment stutters.

\begin{definition}{round-insensitive}\it
  {\em [{\bf Round-insensitivity}]}
  A module $\ro$ is {\em round-insensitive\/} if all atoms of $\ro$ are
  combinational or event-driven.
\end{definition}

\mypar
It follows that the trace language of a round-insensitive module $\ro$ is
closed under the insertion of stutter steps:
if $\obs_0\ldots\obs_n$ is a trace of~$\ro$, then so are the observation
sequences $\obs_0\ldots\obs_i\obs_i\ldots\obs_n$ for all $0\le i\le n$.
%%% WHY IS THIS RELEVANT ?
%Therefore, for round-insensitive modules $\ro$ and~$\aro$, we have
%$\lang{\ro}\subseteq\lang{\aro}$ iff the stutter closure of $\lang{\ro}$ is
%a subset of the stutter closure of~$\lang{\aro}$.
For example, the modules $\SSender$ and $\SReceiver$ from
Figure~\ref{fig:syncmsg} are round-insensitive.

\begin{figure}
  \begin{mtab}{l}
    \MODULE\ \CCount\\
    \qu \INTF\ \cnt : \nat\\
    \qu \ATOM\ \CONTROLS\ \cnt\ \READS\ \cnt\\
    \qqu \INIT\\
    \qqu \begin{chtab}
      \true &\ \cnt':=0
    \end{chtab}\\
    \qqu \UPDATE\\
    \qqu \begin{chtab}
      \true &\cnt':=\cnt+1
      \end{chtab}\\
    \qu \ENDA \\
    \qu \ENDM \\\\

    \MODULE\ \SCount\\
    \qu \EXTL\ \tick : \event\\
    \qu \INTF\ \cnt : \nat\\
    \qu \ATOM\ \cnt\ \READS\ \cnt,\tick\ \AWAITS\ \tick'\\
    \qqu \INIT\\
    \qqu \begin{chtab}
      \true &\ \cnt':=0
    \end{chtab}\\
    \qqu \UPDATE\\
    \qqu \begin{chtab}
      \tick? &\cnt':=\cnt+1\\
      \end{chtab}\\
    \qu \ENDA \\
    \qu \ENDM \\\\


    \MODULE\ \ACount\\
    \qu \INTF\ \cnt : \nat\\
    \qu \ATOM\ \CONTROLS\ \cnt\ \READS\ \cnt\\
    \qqu \INIT\\
    \qqu \begin{chtab}
      \true &\ \cnt':=0
    \end{chtab}\\
    \qqu \UPDATE\\
    \qqu \begin{chtab}
      \true &\cnt':=\cnt+1\\
      \true &
    \end{chtab}\\
    \qu \ENDA \\
    \qu \ENDM \\

  \end{mtab}
\caption{Three counters}
\label{fig:count}
\end{figure}

\mypar
Asynchrony and round-insensitivity are independent:
a module may be synchronous and round-sensitive, asynchronous and
round-sensitive, synchronous and round-insensitive, or asynchronous and
round-insensitive.
The difference between asynchrony and round-insensitivity is illustrated by
the three counters shown in Figure~\ref{fig:count}.
While the environment cannot enforce observable progress of an asynchronous
module, it can enforce observable progress of a round-insensitive module by
modifying external variables.
A round-insensitive module, on the other hand, cannot count rounds, but only
changes in the values of external variables.
In our example, the round-sensitive synchronous counter $\CCount$ is
incremented in each round, the round-insensitive synchronous counter
$\SCount$ is incremented with every occurrence of the external event~$\tick$,
and the round-insensitive asynchronous counter $\ACount$ is incremented
nondeterministically.


\section{Operations on Reactive Modules}

We create complex modules from simple modules using the five operations of
variable renaming, parallel composition, variable hiding, round abstraction,
and triggering.
%%% MOVED THIS HERE
All five operations $f$ are compositional in the sense that the
equivalence relation $\refeq$ is a congruence with respect to~$f$:
if $\ro\refines\aro$, then $f(\ro)\refines f(\aro)$.
Thus, if we prove that module $\ro$ is $\refeq$-equivalent to module~$\aro$,
then $\ro$ can be substituted for $\aro$ in every context without affecting
the meaning of the compound module.
Furthermore, in order to prove that $f(\ro)$ implements~$f(\aro)$, it
suffices to prove that $\ro$ implements~$\aro$.
In this way, reasoning about compound modules can be reduced to reasoning
about the component modules.


\subsection{Variable renaming}

The renaming operation is useful for creating different instances of a
module, and for avoiding name conflicts.
Let $\ro$ be a module, and let $\var$ and $\avar$ be two variables of the
same type such that $\avar$ is not in~$\rovars{\ro}$.
Then the module $\ro[\var:=\avar]$ results from $\ro$ by renaming $\var$
to~$\avar$.
Henceforth, whenever we write $\ro[\var:=\avar]$, we assume that $\var$ and
$\avar$ have the same type, and that $\avar$ is not a module variable
of~$\ro$.
We furthermore assume that for any two modules, a private variable of one
module is not a module variable of the other module.
This can always be achieved by renaming private variables, which does not
change the meaning of a module.
It is obvious that the renaming operation is compositional:
for all modules $\ro$ and~$\aro$, if $\ro\refines\aro$, then
$\ro[\var:=\avar]\refines\aro[\var:=\avar]$.


\subsection{Parallel composition}

The composition operation combines two modules into a single module whose
behavior captures the interaction between the two component modules.
The two modules $\ro$ and $\aro$ are {\em compatible\/} if
(1)~the interface variables of $\ro$ and $\aro$ are disjoint, and
(2)~the await dependencies among the observable variables of $\ro$ and $\aro$
  are acyclic
  ---that is, the transitive closure $(\awaits_\ro\cup\awaits_\aro)^+$ is
  asymmetric.
It follows that if $\ro$ and $\bro$ are compatible modules, and
$\ro\refines\aro$, then $\aro$ and $\bro$ are also compatible.

\begin{definition}{compose}\it
  {\em [{\bf Composition\/}]}
  If $\ro$ and $\aro$ are two compatible modules, then the
  {\em composition\/} $\ro\|\aro$ is the module with the set
  $\priv{\ro\|\aro}=\priv{\ro}\cup\priv{\aro}$ of private variables, the set
  $\intf{\ro\|\aro}=\intf{\ro}\cup\intf{\aro}$ of interface variables, the set
  $\extl{\ro\|\aro}=(\extl{\ro}\cup\extl{\aro})\bs\intf{\ro\|\aro}$ of
  external variables, and the set
  $\roatoms{\ro\|\aro}=\roatoms{\ro}\cup\roatoms{\aro}$ of atoms.
\end{definition}

\mypar
It is easy to check that for two compatible modules $\ro$ and~$\aro$, the
composition $\ro\|\aro$ is again a module.
The composition $\ro\|\aro$ is asynchronous iff both $\ro$ and $\aro$ are
asynchronous, and $\ro\|\aro$ is round-insensitive iff both $\ro$ and $\aro$
are round-insensitive.
Henceforth, whenever we write $\ro\|\aro$, we assume that the modules $\ro$
and $\aro$ are compatible.
The composition operation on modules is commutative and associative.
We therefore omit parentheses when writing $\ro\|\aro\|\bro$.

\mypar
Parallel composition behaves like language intersection.
Let $\ro$ and $\aro$ be two compatible modules, and let $\obsseq$ be a finite
sequence of observations of the compound module $\ro\|\aro$.
Then, $\obsseq$ belongs to the language $\linsem{\ro\|\aro}$ iff the
projection $\obsseq[\ovars\ro]$ belongs to $\linsem\ro$ and the projection
$\obsseq[\ovars\aro]$ belongs to~$\linsem\aro$.
In particular, if $\ro$ and $\aro$ have identical observations, then
$\lang{\ro\|\aro}=\lang\ro\cap\lang\aro$.
It follows that, up to projection, the trace language of a compound module is
a subset of the trace language of each component.
Hence, the composition of two modules creates a module that is equally or
more detailed than its components.
This is captured by the first part of the following proposition.
The second part asserts that the composition operation is compositional.

\begin{proposition}{comp2}
  Let $\ro$, $\aro$, and $\bro$ be three modules such that $\ro$ and $\bro$
  are compatible.
  Then
  {\rm (1)}~$\ro\|\bro$ $\refines$ $\ro$, and
  {\rm (2)}~$\ro\refines\aro$ implies $\ro\|\bro$ $\refines$ $\aro\|\bro$.
\end{proposition}

\mypar
It follows that, in order to prove that a complex compound module
$\ro_1\|\ro_2$ (with a large state space) implements a simpler compound
module $\aro_1\|\aro_2$ (with a small state space), it suffices to prove
(1)~$\ro_1$ implements~$\aro_1$ and
(2)~$\ro_2$ implements~$\aro_2$.
We call this the {\em compositional proof rule\/} for reactive modules.
It is valid, because parallel composition and implementation behave like
language intersection and language containment, respectively.

\mypar
{\bf Assume-guarantee reasoning.}
While the compositional proof rule decomposes the verification task of
proving implementation between compound modules into subtasks, it may not
always be applicable.
In particular, $\ro_1$ may not implement $\aro_1$ for all environments, but
only if the environment behaves like $\ro_2$, and vice versa.
For such cases, an assume-guarantee proof rule is needed
\cite{Stark,AbadiLamport,GrumbergLong94,Reactive95}.
The {\em assume-guarantee proof rule\/} for reactive modules asserts that in
order to prove that $\ro_1\|\ro_2$ implements $\aro_1\|\aro_2$, it suffices
to prove
(1)~$\ro_1\|\aro_2$ implements $\aro_1$, and
(2)~$\aro_1\|\ro_2$ implements $\aro_2$.
Both proof obligations (1) and (2) typically involve smaller state spaces
than the original proof obligation, because the complex compound module
$\ro_1\|\ro_2$ usually has the largest state space involved.
The assume-guarantee proof rule is circular;
unlike the compositional proof rule, it does not simply follow from the
fact that parallel composition and implementation behave like language
intersection and language containment.
Rather the proof of the validity of the assume-guarantee proof rule proceeds
by induction on the length of traces.
For this, it is crucial that every trace of a module can be extended.

\begin{proposition}{ag}
  Let $\ro_1$ and $\ro_2$ be two compatible modules, and let $\aro_1$ and
  $\aro_2$ be two compatible modules such that every external variable of
  $\aro_1\|\aro_2$ is an observable variable of $\ro_1\|\ro_2$.
  If $\ro_1\|\aro_2\linimp\aro_1$ and $\aro_1\|\ro_2\linimp\aro_2$, then
  $\ro_1\|\ro_2\linimp\aro_1\|\aro_2$.
\end{proposition}


\subsection{Variable hiding}

The hiding of interface variables allows us to construct module abstractions
of varying degrees of detail.
For instance, after composing two modules, it may be appropriate to convert
some interface variables to private variables, so that they are used only for
the interaction of the component modules, and are no longer visible to the
environment of the compound module.

\begin{definition}{hiding}\it
  {\em [{\bf Hiding}]}
  Given a module $\ro$ and a variable~$\var$, by {\em hiding\/} $\var$ in
  $\ro$ we obtain the module $\HIDE\ \var\ \IN\ \ro$.
  If $\var$ is an interface variable of~$\ro$, then $\HIDE\ \var\ \IN\ \ro$
  has the set $\priv{\ro}\cup\set{\var}$ of private variables, the set
  $\intf{\ro}\bs\set{\var}$ of interface variables, the set $\extl{\ro}$ of
  external variables, and the set $\roatoms{\ro}$ of atoms.
  If $\var$ is not an interface variable of~$\ro$, then
  $\HIDE\ \var\ \IN\ \ro$ is identical to~$\ro$.
\end{definition}

\mypar
We write $\HIDE\ \var_1,\var_2\ \IN\ \ro$ short for the module
$\HIDE\ \var_1\ \IN\ (\HIDE\ \var_2\ \IN\ \ro)$, which is identical to the
module $\HIDE\ \var_2\ \IN\ (\HIDE\ \var_1\ \IN\ \ro)$.
The hiding of a variable creates in a module that is equally or less
detailed, and the hiding operation is compositional.
Both facts are stated in the following proposition.

\begin{proposition}{hide}
  For all modules $\ro$ and~$\aro$, and every variable~$\var$,
  (1)~$\ro$ $\refines$ $(\HIDE\ \var\ \IN\ \ro)$, and
  (2)~if $\ro\refines\aro$, then
    $(\HIDE\ \var\ \IN\ \ro)$ $\refines$ $(\HIDE\ \var\ \IN\ \aro)$.
\end{proposition}

\mypar
{\bf Spatial scaling.}
The three operations of renaming, composition, and hiding allow us to
construct a space hierarchy of modules.
This can be illustrated by the example of synchronous circuits.

\begin{figure}
\begin{mtab}{l}
  \MODULE\ \SOrSpec\\
  \qu \EXTL\ \iino,\iint : \bool \\
  \qu \INTF\ \oout : \bool \\
  \qu \ATOM\ \CONTROLS\ \oout\ \AWAITS\ \iino',\iint' \\
  \qqu \INIT\ \UPDATE\\
  \qqu \begin{chtab}
    \iino'=0 &\oout':=0\\
    \iint'=0 &\oout':=0\\
    \iino'=1\OR\iint'=1 &\oout':=1
  \end{chtab}\\
  \qu \ENDA \\
  \qu \ENDM \\\\

  \SOrImp\ =\ \HIDE\ z_1,z_2,z_3\ \IN\\
  \qu \pppar\SAnd[\iino,\iint,\oout:=z_1,z_2,z_3]\\
  \qu \pppar\SNot[\iin,\oout:=\iino,z_1]\\
  \qu \pppar\SNot[\iin,\oout:=\iint,z_2]\\
  \qu \pppar\SNot[\iin,\oout:=z_3,\oout]
\end{mtab}
\caption{Two definitions of a synchronous \Or\ gate}
\label{fig:syncor}
\end{figure}

%%% EXTENDED EXAMPLE
\mypar
Figure~\ref{fig:syncor} shows two module definitions for a synchronous \Or\
gate.
The module $\SOrSpec$ specifies the input-output behavior of an \Or\ gate
similar to the definition of the synchronous \And\ gate from
Figure~\ref{fig:syncand}.
The module $\SOrImp$ builds an \Or\ gate from an \And\ gate and three
inverters.
First, we rename the variables of the modules from Figure~\ref{fig:syncand}
that define synchronous \And\ and \Not\ gates in order to create three
instances of a \Not\ gate and connect, for example, the output of the \And\
gate with the input of the third \Not\ gate.
Second, we compose the four modules representing the \And\ gate and the three
\Not\ gates.
Third, we hide the variables that represent internal wires, for example, the
wire $z_3$ that connects the output of the \And\ gate with the input of the
third \Not\ gate.
It is easy to check that the two modules $\SOrSpec$ and $\SOrImp$ are
$\refeq$-equivalent;
in particular, they have the same traces.

%\begin{figure}
%\strut\centerline{\input synccount.eepic}
%\caption{Block diagram for the three-bit binary counter}
%\label{fig:synccount}
%\end{figure}

\begin{figure}
\begin{mtab}{l}
  \SXor\ =\ \HIDE\ \za, \zb, \zc, \zd\ \IN \\
  \qu \comment\EXTL\ \iino,\iint \\
  \qu \comment\INTF\ \oout \\
  \qu \SNot[\iin, \oout := \iino, \za] \\
  \qu \pppar \SNot[\iin, \oout := \iint, \zb] \\
  \qu \pppar \SAnd[\iint, \oout := \zb, \zc] \\
  \qu \pppar \SAnd[\iino, \oout :=  \za, \zd] \\
  \qu \pppar \SOr[\iino, \iint := \zc, \zd] \\\\
  \SCountOne\ =\ \HIDE\ \za\ \IN \\
  \qu \comment\EXTL\ \start,\inc\\
  \qu \comment\INTF\ \oout,\carry\\
  \qu \SLatch[\reset, \iin, \oout := \start, \oout, \za] \\
  \qu \pppar \SAnd[\iino, \iint, \oout := \inc, \za, \carry] \\
  \qu \pppar \SXor[\iino, \iint := \inc, \za] \\\\
%  \SCountOne\ =\ \HIDE\ \sset,\reset,z\ \IN\\
%  \qu \comment\EXTL\ \start,\inc\\
%  \qu \comment\INTF\ \oout,\carry\\
%  \qu \pppar\SLatch[\sset,\reset,\oout]\\
%  \qu \pppar\SAnd[\iino,\iint,\oout:=\oout,\inc,\carry]\\
%  \qu \pppar\SOr[\iino,\iint,\oout:=\carry,\start,\reset]\\
%  \qu \pppar\SNot[\iin,\oout:=\reset,z]\\
%  \qu \pppar \SAnd[\iino,\iint,\oout:=\inc,z,\sset]\\\\
  \SCountThree\ =\ \HIDE\ \carrya,\carryb,\carryc\ \IN\\
  \qu \comment\EXTL\ \start,\inc\\
  \qu \comment\INTF\ \oouta,\ooutb,\ooutc\\
  \qu \pppar\SCountOne[\start,\inc,\oout,\carry:=
    \start,\inc,\oouta,\carrya]\\
  \qu \pppar\SCountOne[\start,\inc,\oout,\carry:=
    \start,\carrya,\ooutb,\carryb]\\
  \qu \pppar\SCountOne[\start,\inc,\oout,\carry:=
    \start,\carryb,\ooutc,\carryc]
\end{mtab}
\caption{Three-bit binary counter}
\label{fig:synccount1}
\end{figure}

\mypar
Using the definitions of synchronous gates and latches from
Figure~\ref{fig:syncand}, and the three operations of renaming, composition,
and hiding, we can build sequential circuits whose clock cycles correspond to
rounds.
As an example, we design a three-bit binary counter.
The counter takes two boolean inputs, represented by the external variables
$\start$ and $\inc$, for starting and incrementing the counter.
The counter value ranges from 0 to~7, and is represented by three bits.
We do not make any assumption about the initial counter value.
A start command resets the counter value to 0 and overrides any increment
command that is issued in the same round.
An increment command increases the counter value by~1.
If the counter value is~7, the increment command changes the counter value
to~0.
In each round, the counter issues its value as output
---the low bit on the interface variable $\oout_0$, the middle bit on the
interface variable $\oout_1$, and the high bit on the interface variable
$\oout_2$.

\mypar Figure~\ref{fig:synccount} shows a possible design of the
three-bit counter from three one-bit counters. This design is
defined by the module \SCountThree\ of Figure~\ref{fig:synccount1}
(for clarity, we sometimes annotate both component and compound
modules with the names of the observable variables). Note that
$\carry_0$ waits for both $\start$ and $\inc$, that $\carry_1$
waits for $\carry_0$, and that $\carry_2$ waits for $\carry_1$. It
follows that all three bits of the counter are updated within a
single round. By contrast, a combinational cycle would result in
cyclic await dependencies, and thus cannot be modelled in this
way, by identifying each clock cycle with a round. Rather,
circuits that contain combinational cycles need to be modelled by
identifying a single clock cycle with several rounds
---say, as many rounds as it takes for the external event $\tick$ to happen.
%%% WOULDN'T THIS MAKE A GOOD EXAMPLE FOR ROUND ABSTRACTION ?

\mypar
{\bf From synchrony to asynchrony.}
Hiding preserves asynchrony, round-sensitivity, and round-insensitivity, but
not synchrony.
Hence, hiding is useful for constructing asynchronous modules from
synchronous modules.
Consider, for example, an asynchronous module $\Clock$ that
nondeterministically issues the interface event $\tick$:
\begin{mtab}{l}
    \MODULE\ \Clock\\
    \qu \INTF\ \tick : \event\\
    \qu \ATOM\ \CONTROLS\ \tick\ \READS\ \tick\\
    \qqu \UPDATE\\
    \qqu \begin{chtab}
      \true&\tick!\\
      \true&
      \end{chtab}\\
    \qu \ENDA \\
    \qu \ENDM \\

\end{mtab}
Then, given the synchronous counter $\SCount$ from Figure~\ref{fig:count},
we can implement the asynchronous counter $\ACount$ using hiding:
  \[\ACount\ \refeq\ \HIDE\ \tick\ \IN\ \SCount\ppar\Clock\]

\begin{figure}
\begin{mtab}{l}
  \MODULE\ \SRSpec\\
  \qu \INTF\ \msgP,\msgC:\msgType\\
  \qu \PRIV\ \status : \set{\prodcons,\prods,\cons}\; \msgO : \msgType\\
  \qu \ATOM\ \CONTROLS \status,\msgP,\msgC,\msgO\ \READS\ \\
  \qqu \INIT\\
  \qqu \begin{chtab}
    \true &\status':=\prods
  \end{chtab}\\
  \qqu \UPDATE\\
  \qqu \begin{chtab}
    \status=\prods&
      \msgP':=\msgType\;\status':=\prodcons\\
    \status=\prodcons&
      \msgC':=\msgP\;\msgP':=\msgType\;\status':=\prodcons\\
    \status=\prodcons&
      \msgO':=\msgP\;\msgP':=\msgType\;\status':=\cons\\
    \status=\prodcons&
      \msgC':=\msgP\;\status':=\prods\\
    \status=\cons&
      \msgC':=\msgO\;\status':=\prodcons\\
    \true &
  \end{chtab} \\
  \qu \ENDA \\
  \qu \ENDM \\
\end{mtab}
\caption{Asynchronous message-passing specification}
\label{fig:asyncmsg}
\end{figure}

%%% EXTENDED EXAMPLE
\mypar
For a more elaborate example, recall the synchronous message-passing protocol
from Figure~\ref{fig:syncmsg}:
  \[\SR\ =\ \SSender\ppar\SReceiver\]
After hiding the communication events, so that only the streams of produced
and consumed messages remain visible, we obtain an asynchronous module:
  \[\SRImpl\ =\ \HIDE\ \ready,\transmit,\msgS\ \IN\ \SR\]
This module implements the asynchronous module \SRSpec\ of
Figure~\ref{fig:asyncmsg}, which contains a single lazy atom:
  \[\SRImpl\ \refines\ \SRSpec\]
Alternatively, \SRSpec\ could be implemented by sender and receiver processes
that communicate via asynchronous, rather than synchronous, handshaking.
