\section{Alternating-time Temporal Logic}
In this section, we discuss model checking 
Alternating Temporal Logic formulas in \mocha.  The reader is
refered to \cite{ATL} for details. Here we only give a brief introduction
to this logic.

ATL is a temporal logic designed to write
requirements of {\em open\/} systems \cite{ATL}. For instance, let
$\Sigma$ be a set of agents corresponding to different components of
the system and the external environment.  Then, the logic ATL admits
formulas of the form $\ors{A}F p$, where $p$ is a state predicate and
$A$ is a subset of agents.  The formula $\ors{A}F p$ means that the
agents in the set $A$ can cooperate to reach a $p$-state no matter how
the remaining agents resolve their choices.  This is formalized by
defining games, and satisfaction of ATL formulas corresponds to
existence of winning strategies in such games.

The model checking problem for ATL is to determine whether a given module
satisfies a given ATL formula. In Reactive Modules, each agent corresponds
to an atom. For each external variable, there is an extra agent which
controls it. 


