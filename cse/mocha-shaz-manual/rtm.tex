\section{Real-Time Modules}
\label{sec:rtm}

\subsection{Describing Systems with Real-Time Modules}
{\sc Mocha} supports real-time systems that are
described in the form of {\em timed reactive modules} as defined in
\cite{timed-modules}. In addition to the discrete-valued variables of
reactive modules, a timed module makes use real-valued ($\reals$) {\em clock
variables}. The purpose of clock variables is to keep track of
elapsed time. Guarded commands can assign integer values to clock
variables, and guards can depend on values of clock variables. 

$\clock$ denotes the clock variable type. Clock variables
are initialized to $0$ by default\footnote{
An error is signal-led if there is a non-zero initial assignment to a
clock variable.}. For each clock $x$, guards can include positive
Boolean combinations of inequalities of the form $x \le C$, $x \ge C$
and $x=C$, where $C \in \nats$. Assignments can assign integer values
to clocks in the usual fashion: $x := C$, where $C \in \nats$.

In addition to $\INIT$ and $\UPDATE$ commands, a timed module has a
set of $\WAIT$ commands which describe the passage of time. When a
$\WAIT$ command is executed, all non-clock variables remain the same,
and all clock variables are incremented by the same amount. A typical
$\WAIT$ command has the following form

$$ \mbox{{\tt Untimed\_Guard}} \;\land\; x_1 \le C_1 \;\land\; x_2 \le C_2 \;\land\;
... \;\land\; x_n \le C_n \;\;\rightarrow\;\; x_1':\le C_1 ;\;\; x_2' :\le C_2 ;\;\;
... ;\;\; x_n :\le C_n $$

\noindent where $x_1,...,x_n$ are clocks, $C_1,...,C_n$ are positive
integers, and ``{\tt Untimed\_Guard}'' is a guard on non-clock
variables. The interpretation of such a $\WAIT$ statement is as
follows: If {\tt Untimed\_Guard} holds and the values of the clocks
satisfy the inequalities $x_i \le C_i$ for all $i \le n$, then a time
period of $\delta$ can elapse while the values of all non-clock
variables remain constant. $\delta$ must satisfy the following condition:
$x_i + \delta = x_i' \le C_i$ for all $i$. In this way, $\WAIT$
statements are used to specify upper bounds on the time elapsed in a
given state. The guards of $\WAIT$ commands are sometimes referred to
as ``clock invariants'' because the invariants on the clocks specified
by the guard must hold for the module to remain in that state.

A timed module makes progress by executing either an $\UPDATE$ command
or a $\WAIT$ command. If no $\WAIT$ command can be executed, then this
forces an $\UPDATE$ command to be executed\footnote{For a precise
treatment of the semantics of timed modules, refer to
\cite{AlurHenzinger97}}. 

\begin{figure}[h]
\begin{mtab}{l}
  \MODULE\ \RTTrain\\
  \qu \INTF\ \pc:\set{\far,\near,\gate};\; \arrive:\bool\\
  \qu \PRIV\ x:\clock\\
  \qu \ATOM\ \CONTROLS \pc,x,\arrive\ \READS\ \pc,x,\arrive\\
  \qqu \INIT\\
  \qqu \begin{chtab}
    \true &\pc':=\far;\; \arrive':=\false
  \end{chtab}\\
  \qqu \UPDATE\\
  \qqu \begin{chtab}
    \pc=\far &\pc':=\near;\; \arrive':=\true;\; x':=0\\
    \pc=\near\land x\ge 3 &\pc':=\gate;\; x':=0\\
    \pc=\gate\land x\ge 1 &\pc':=\far;\; \arrive':=\false\\
  \end{chtab}\\
  \qqu \WAIT\\
  \qqu \begin{chtab}
    \pc=\far & \\
    \pc=\near\land x\le 5 & x :\le 5 \\
    \pc=\gate\land x\le 2 & x :\le 2 \\
  \end{chtab}\\\\
\end{mtab}
\caption{Real-time module for the train }
\label{fig:rttrain}
\end{figure}

A simple example of a real-time module is presented in
Figure~\ref{fig:rttrain}. This module describes the behavior of a
train approaching a gate. The interface variable $\pc$ initially has
the value $\far$, indicating that the train is far from the gate, and
the variable $\arrive$ is set to $\false$. $\pc$ can have the value
$\far$ indefinitely, as indicated by the first $\WAIT$ command,
or can take on the value $\near$. When the train moves
to $\near$, the timer $x$ is reset to $0$ and the interface variable 
$\arrive$ is set to $\true$. The second $\WAIT$ command puts an
upper bound of 5 on $x$ while $\pc = \near$. While $\pc = \near$, if $x \ge 3$, then the guard of the second $\UPDATE$ command is satisfied, which means
that after spending 3 time units at $\pc = \near$, $\pc$ can move to $\far$.
After 5 time units at $\pc = \near$, the guard of the second $\WAIT$ statement 
is no longer satisfied, which forces the second update command to be executed.
After 5 time units, $\pc$ {\em must} move to $\far$. 

\mypar

\begin{figure}
\begin{mtab}{l}
  \MODULE\ \RTGate\\
  \qu \EXTL\ \arrive:\bool\\
  \qu \INTF\ \pc:\set{\open,\toclose,\closed}\\
  \qu \PRIV\ y:\clock\\
  \qu \ATOM\ \CONTROLS \pc,y\ \READS\ \pc,y,\arrive \AWAITS\ \arrive' \\
  \qqu \INIT\\
  \qqu \begin{chtab}
    \true &\pc':=\open
  \end{chtab}\\
  \qqu \UPDATE\\
  \qqu \begin{chtab}
    \pc=\open\land\arrive' &\pc':=\toclose\; y':=0\\
    \pc=\toclose\land y\ge 1 &\pc':=\closed\; y':=0\\
    \pc=\closed\land y\ge 7 &\pc':=\open\\
 \end{chtab}\\
  \qqu \WAIT\\
  \qqu \begin{chtab}
    \pc=\open\land\neg\arrive' & \\
    \pc=\toclose\land y\le 2 & y:\le 2\\
    \pc=\closed\land y\le 7 & y :\le 7
  \end{chtab}\\\\
\end{mtab}
\caption{Real-time railroad gate controller}
\label{fig:rtctrl}
\end{figure}

The timed reactive module description for a gate controller given in 
Figure~\ref{fig:ctrl} operates in a similar fashion. The system
consisting of the train and the controller is then given by
$$RealTimeTrainSystem = \HIDE\ \;\arrive\; \IN \;\RTTrain \;\pppar\; \RTGate$$

\subsection{Verification with Real-Time Modules}

{\sc Mocha} supports the following commands that operate on real-time
modules 
	\begin{itemize}
	\item {rtm\_trans} [-v] [-h] $\langle$module\_name$\rangle$:
	\item {rtm\_init} [-v] [-h] $\langle$module\_name$\rangle$:
	\item {rtm\_search} [-v] [-i] [-h] $\langle$module\_name$\rangle$:
	\item {rtm\_static\_order} [[-h] -f $\langle$file\_name$\rangle$ $\langle$module\_name$\rangle$:
	\item {rtm\_dynamic\_var\_ordering} [-d] [-e
$\langle$method$\rangle$ ] [ -f $\langle$method$\rangle$ ]  [-h]:
	\end{itemize}


\subsection{Transition Relations of Real-Time Modules}

{\sc Mocha} restricts guards on clocks and clock invariants to be
positive Boolean combinations of inequalities of the form $x \le c$
and $x \ge c$, where $c \in \nats$. This is adequate for modeling the
behavior of any physical system, as argued in \cite{serdar-cav-submit}. In
\cite{digital-clocks}, it is proven that, with this restriction, for
each trace $\trace$ of a timed module, there exists a trace
$\inttrace$ such that (i) the sequence values that discrete variables
take on is the same for $\trace$ and $\inttrace$ and (ii) all updates
of discrete variables take place at integer-valued points in
time. This enables clocks to be modeled as integer-valued variables
that increase at the same rate. Timed modules are converted by {\sc
Mocha} into (untimed) modules, equivalent to the original ones in the
sense described above.




