\documentclass{slides}
\usepackage[dvips]{color}
\usepackage{epsfig}

\newcommand{\spacer}{\rule[-3mm]{0mm}{8mm}}

\title{Security Analysis Of Handhelds}
\author{Vaibhav Bhandari\\
Department of Computer Science, UCSC\\
vaibhav@soe.ucsc.edu
}

\begin{document}
%%%%%%%%%%%%%%%%%%%%%%%%===Slide 0
\maketitle

%%%%%%%%%%%%%%%%%%%%%%%===Slide 1
\begin{slide}
\begin{center}
\underline{Embeded Devices are everywhere}
\end{center}

We can't built secure applications on these insecure devices.

\end{slide}

%%%%%%%%%%%%%%%%%%%%%%%%===Slide 2
\begin{slide}
\begin{center}
\underline{Introduction}
\end{center}
\begin{itemize}
\item Threat : Causual / Determined
\item Security : Restricted / Confidential
\end{itemize}
\end{slide}

%%%%%%%%%%%%%%%%%%%%%%%%===Slide 3
\begin{slide}
\begin{center}
\underline{Exemplary Handhelds}
\end{center}
\begin{itemize}
\item Palm OS device: Monolithic OS
\item Simputer: Mass market PC with full-fledged OS support 
\end{itemize}
\end{slide}

%%%%%%%%%%%%%%%%%%%%%%%%===Slide 4
\begin{slide}
\begin{center}
\underline{Security Analysis}
\end{center}

Evaluating the robustness of security policy to handle the
various threats.\\

Making certain the robustness of various security mechanisms
and understanding there limitations.
\end{slide}

%%%%%%%%%%%%%%%%%%%%%%%%===Slide 5
\begin{slide}
\begin{center}
\underline{Questions to answer}
\end{center}
\small
\begin{itemize}
\item Does it have a power-on password and is the password verification robust?
\item Is there access control for local/ remote operation?
\item Is there a way to check Malignant programs?
\item Is Strong Encryption available for both wired and wireless links?
\item Is Memory properly secured?
\item Is there a cyrpto-processor and how vulnerable is it?
\item Are the device protocols time/tool tested?
\item Is there any audit mechanism available?
\item Does the device have access to updated information?
\item Finally, how robust are the business process?
\end{itemize}
\end{slide}

%%%%%%%%%%%%%%%%%%%%%%%%===Slide 6
\begin{slide}
\begin{center}
\underline{Secure Password Retrieval}
\end{center}
\small
\begin{itemize}\addtolength{\itemsep}{-0.5\baselineskip}
\item Challenge/ Response mechanism on the network.
\item Encrypt and Salt Credentials stored on the system.
\item Implementation of {\bf power-on} passsword.
\item Alternative passwords like signatures, graphical passowords.
\end {itemize}
\end{slide}

%%%%%%%%%%%%%%%%%%%%%%%%===Slide 7
\begin{slide}
\begin{center}
\underline{Access Control}
\end{center}
\small
\begin{itemize}
\item Palm: password protected data -records and beam bit.
\item Simputer: Session database.
\begin {itemize}
\item Name
\item Magic: read(r), exclusive(e), secure(s), transient(t)
\item Value
\item An example of a table is given below:
\begin{center}
\begin{tabular}[!h]{|lll|}
     \hline
      Name  &   Magic   &     Value \\
     \hline
     name   &   s       & Manohar \\
     address &  s       & CSA Department\\
     DOB     &  s       & June 21 1960 \\
     Status  &  e567rty & Married \\
     phone   & ru87iii89 & 3092368 \\
     choice  & t345tyu  &  pizza \\
     \hline
\end{tabular}
\end{center}
\end{itemize}
\pagebreak
\item IML code segment using it:
\begin{center}
\begin{tabular}{|l|}
\hline
    \verb+<page>+\\
    \verb+ <tr><td> Name: </td><td><input type="text" width="15"+\\
    \verb+		height="1"+\\
    \verb+var="var0" value="_name" magic="s"/></td></tr>+\\
    \verb+ <tr><td> Occupation: </td><td><input type="text" +\\
    \verb+		width="15" height= 1"+\\
    \verb+var="var0" value="_occupation" magic="s"/></td></tr>+\\
    \verb+</page>+\\
\hline
\end{tabular}
\end{center}
\item  Transfer of data to an application (especially remote) is through express approval of user.
\end{itemize}
\end{slide}

%%%%%%%%%%%%%%%%%%%%%%%%===Slide 8
\begin{slide}
\begin{center}
\underline{Smart Card Attacks}
\end{center}
\small
\begin{itemize}
\item Invasive attack on Hardware:Using devices like scanning capacitance microscope can fuel low-cost attacks.
\item Non-invasive hardware attack: like analysis of power consumption will fade due to counter attacks like randomised clocking.
\item API-level attacks: Chained processing commands in unusual way based on:
\begin{itemize}
\item Data Repeats: exploiting X-OR vulnerability
\item Timing
\item Man in middle
\end{itemize}
\end{itemize}
\end{slide}

%%%%%%%%%%%%%%%%%%%%%%%%===Slide 9
\begin{slide}
\begin{center}
\underline{An Authentication Protocol}
\end{center}

\begin{center}
\begin{tabular}{|lll|}
\hline
 & & \\
U & $\Rightarrow $ & $ SC: PIN $ \\
SC & $\Rightarrow $ & $ D: User Valid $\\
D & $\Rightarrow $ & $ SC: K_{PD}, D_{SN}$ \\
SC & $\Rightarrow $ &  $ D: \{N, SC, K_{PD}, D_{SN}\}_{K_{SU}}, K_{PU}$ \\
D & $\Rightarrow $ & $ CA: \{N, SC, K_{PD}, D_{SN}\}_{K_{SU}}, K_{PU}$ \\
CA & $\Rightarrow $ & $ D: \{Good D\}_{K_{PU}}$ \\
D & $\Rightarrow $ & $ SC: \{Good D\}_{K_{PU}}$ \\
 & & \\
\hline
\end{tabular}
\end{center}

In {\bf BAN} Logic:
\begin{center}
\small
\begin{tabular}{c}
(SC believes good User) (SC believes CA said good Device)\\
\hline
SC believes good Device \\
\end{tabular}
\end{center}
\end{slide}

%%%%%%%%%%%%%%%%%%%%%%%%===Slide 10
\begin{slide}
\begin{center}
\underline{Acknowledgements}
\end{center}
Thanks every one, especially Prof. Martin Abadi and KS Vivek of PicoPeta Simputers for support and ideas.
\end{slide}

\end{document}
